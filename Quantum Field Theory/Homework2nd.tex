% XeLaTeX can use any Mac OS X font. See the setromanfont command below.
% Input to XeLaTeX is full Unicode, so Unicode characters can be typed directly into the source.

% The next lines tell TeXShop to typeset with xelatex, and to open and save the source with Unicode encoding.

%!TEX TS-program = xelatex
%!TEX encoding = UTF-8 Unicode

\documentclass[12pt]{article}
\usepackage{geometry}                % See geometry.pdf to learn the layout options. There are lots.
\geometry{letterpaper}                   % ... or a4paper or a5paper or ... 
%\geometry{landscape}                % Activate for for rotated page geometry
%\usepackage[parfill]{parskip}    % Activate to begin paragraphs with an empty line rather than an indent
\usepackage{graphicx}
\usepackage{amssymb}
\usepackage{amsmath,bm}           % 数学符号与特殊字符,其中\bm表示黑体公式
\usepackage{ragged2e}          % 两端对齐
\usepackage{color}             	% 字体颜色
\usepackage{indentfirst}		% 首行缩进
\geometry{left=1.5cm,right=2.5cm,top=2.0cm,bottom=2.5cm}
\usepackage{setspace}		%使用间距宏包
\usepackage{simplewick}		% 场论中收缩符号专用宏包,非必要

% Will Robertson's fontspec.sty can be used to simplify font choices.
% To experiment, open /Applications/Font Book to examine the fonts provided on Mac OS X,
% and change "Hoefler Text" to any of these choices.

\usepackage{fontspec,xltxtra,xunicode}
\defaultfontfeatures{Mapping=tex-text}
\setromanfont[Mapping=tex-text]{Hoefler Text}
\setsansfont[Scale=MatchLowercase,Mapping=tex-text]{Gill Sans}
\setmonofont[Scale=MatchLowercase]{Andale Mono}

\title{\textbf{Quantum Feild Theory Homework}}
\author{\emph{Surilige Dhalenguite} \and \emph{Bingkai Sheng}}
%\date{}                                           % Activate to display a given date or no date


\begin{document}
\maketitle
\setlength{\parindent}{0pt}	% 取消首行缩进,如果需要缩进,则将0pt设置为所需数量的pt,一般为2pt
\begin{spacing}{1.5}		% 行间距变为double-space
% For many users, the previous commands will be enough.
% If you want to directly input Unicode, add an Input Menu or Keyboard to the menu bar 
% using the International Panel in System Preferences.
% Unicode must be typeset using a font containing the appropriate characters.
% Remove the comment signs below for examples.

% \newfontfamily{\A}{Geeza Pro}
% \newfontfamily{\H}[Scale=0.9]{Lucida Grande}
% \newfontfamily{\J}[Scale=0.85]{Osaka}
\newcommand*{\dif}{\mathop{}\!\mathrm{d}} 		% \mathop{}用来输出直立黑体的微分算子,如\mathop{d},为了简略我们使用\dif代替\mathop{d}


% 第一题
\textbf{1. Consider $\psi^{\top}(x)C\Gamma\psi(x)$, which $C=i\gamma_{0}\gamma_{2}$. Analysis $\psi^{\top}(x)C\Gamma\psi(x)$ under the Lorentz transformation.} \\		% \top是转置符号
~\\
\textbf{Answer\,:} $\displaystyle\dif{x'}^{\mu}=\frac{\partial{x'^{\mu}}}{\partial{x^{\nu}}}\equiv\Lambda^{\mu}_{\;\nu}\dif{x}^{\nu}$, $\det(\Lambda^{\mu}_{\;\nu})=1$, $(\gamma^{0})^{\top}=\gamma^{0}$, $(\gamma^{1})^{\top}=-\gamma^{1}$, $(\gamma^{2})^{\top}=\gamma^{2}$, $(\gamma^{3})^{\top}=-\gamma^{3}$; 	
				% \displaystyle使各字母或符号达到适当大小
$\psi'(x')=S\psi(x)$, $S^{-1}\gamma^{\mu}S=\Lambda^{\mu}_{\;\nu}\gamma^{\nu}$, $S^{\dagger}=\gamma^{0}S^{-1}\gamma^{0}$, $S^{*}=\gamma^{0}\gamma^{1}\gamma^{3}S\gamma^{3}\gamma^{1}\gamma^{0}$.\\
~\\
(1) $\Gamma=I_{4\times4}$\\
We can write the Lorentz transformation below :
\begin{equation}
\displaystyle\psi'^{\top}(x')C\psi'(x')=\psi^{\top}(x)S^{\top}i\gamma_{0}\gamma_{2}S\psi(x)=-\psi^{\top}(x)S^{\top}i\gamma^{0}\gamma^{2}S\psi(x)
	\tag{1.1.1}
\end{equation}
So we have :\\
\begin{equation}
\begin{aligned} 	% 依照等号对齐,需要对齐的等号前加上&即可
S^{\dagger} &= {S^{*}}^{\dagger} =(\gamma^{0}\gamma^{1}\gamma^{3}S\gamma^{3}\gamma^{1}\gamma^{0})^{\top}=(\gamma^{0})^{\top}(\gamma^{1})^{\top}(\gamma^{3})^{\top}S(\gamma^{3})^{\top}(\gamma^{1})^{\top}(\gamma^{0})^{\top} \\
&=\gamma^{0}(-\gamma^{1})(-\gamma^{3})S^{\top}(-\gamma^{3})(-\gamma^{1})\gamma^{0} \\
&=\gamma^{0}S^{-1}\gamma^{0}
\end{aligned}
	\tag{1.1.2}
\end{equation}
So we can get $\gamma^{1}\gamma^{3}S^{\top}\gamma^{3}\gamma^{1}=S^{-1}$. Then we'll get $S^{\top}=\gamma^{3}\gamma^{1}S^{-1}\gamma^{1}\gamma^{3}$.\\
Substitute Equ.1.1.2 into Equ.1.1.1:
\begin{equation}
\begin{aligned}
\displaystyle \psi'^{\top}(x')C\psi'(x') &= -\psi^{\top}(x)\gamma^{3}\gamma^{1}S^{-1}\gamma^{1}\gamma^{3}i\gamma^{0}\gamma^{2}S\psi(x)=\psi^{\top}(x)\gamma^{3}\gamma^{1}S^{-1}i\gamma^{0}\gamma^{1}\gamma^{2}\gamma^{3}S\psi(x) \\
&=\psi^{\top}(x)\gamma^{3}\gamma^{1}S^{-1}\gamma^{5}S\psi(x)
\end{aligned}
	\tag{1.1.3}
\end{equation}
For
\begin{equation}
S^{-1}\gamma^{5}S=\det(\Lambda^{\mu}_{\;\nu})\gamma^{5}=\gamma^{5}
	\tag{1.1.4}
\end{equation}
So we have $\psi'^{\top}(x')C\psi'(x')=\psi^{\top}(x)\gamma^{3}\gamma^{1}\gamma^{5}\psi(x)$, of which
\begin{equation}
\begin{aligned} 
\displaystyle\gamma^{3}\gamma^{1}\gamma^{5} &= \gamma^{3}\gamma^{1}(i\gamma^{0}\gamma^{1}\gamma^{2}\gamma^{3})=-i\gamma^{0}(\gamma^{1})^{2}\gamma^{2}(\gamma^{3})^{2}\\
&=-i\gamma^{0}\gamma^{2}=i\gamma_{0}\gamma_{2}=C
\end{aligned}	\tag{1.1.5}
\end{equation}
We can finally get $\psi'^{\top}(x')C\psi'(x')=\psi^{\top}(x)C\psi(x)$ when $\Gamma=I_{4\times4}$.\\
~\\
~\\
(2) $\Gamma=\gamma^{\mu}$\\
\begin{equation}
\begin{aligned}
\displaystyle \psi'^{\top}(x')\gamma^{\mu}C\psi'(x')  &= \psi^{\top}(x)S^{\top}\gamma^{\mu}CS\psi(x)=\psi^{\top}(x)\gamma^{3}\gamma^{1}S^{-1}\gamma^{1}\gamma^{3}\gamma^{\mu}i\gamma_{0}\gamma_{2}\psi(x) \\
&= -\psi^{\top}(x)\gamma^{3}\gamma^{1} S^{-1}\gamma^{1}\gamma^{3}\gamma^{\mu} i\gamma^{0}\gamma^{2} S\psi(x)
\end{aligned}	\tag{1.2.1}
\end{equation}
When $\mu=0, 2$ \\
\begin{equation}
\begin{aligned}
\displaystyle \psi'^{\top}(x')\gamma^{\mu}C\psi'(x')  &= -\psi^{\top}(x)\gamma^{3}\gamma^{1}S^{-1}\gamma^{\mu}i\gamma^{1}\gamma^{3}\gamma^{0}\gamma^{2}S\psi(x) \\
&= \psi^{\top}(x)\gamma^{3}\gamma^{1}S^{-1}\gamma^{\mu}i\gamma^{0}\gamma^{1}\gamma^{2}\gamma^{3}S\psi(x) \\
&= \psi^{\top}(x)\gamma^{3}\gamma^{1}S^{-1}\gamma^{\mu}\gamma^{5}S\psi(x) \\
&= -\psi^{\top}(x)\gamma^{3}\gamma^{1}S^{-1}\gamma^{5}\gamma^{\mu}S\psi(x) \\
&= -\psi^{\top}(x)\gamma^{3}\gamma^{1}\gamma^{5}\Lambda^{\mu}_{\;\nu}\gamma^{nu}S\psi(x) \\
&= -\Lambda^{\mu}_{\;\nu}\psi^{\top}(x)C\gamma^{\nu}\psi(x)
\end{aligned}	\tag{1.2.2}
\end{equation}
For \;$C\gamma^{\nu}=-i\gamma^{0}\gamma^{2}\gamma^{\nu}$\; and $\{\gamma^{\mu},\gamma^{\nu}\}=\gamma^{\mu}\gamma^{\nu}+\gamma^{\nu}\gamma^{\mu}=2g^{\mu\nu}I_{4\times4}$\;,\\
We have
\begin{equation}
\begin{aligned}
\displaystyle  C\gamma^{\nu} &= -i\gamma^{0}(2g^{2\nu}I_{4\times4}-\gamma^{\nu}\gamma^{2})=-2i g^{2\nu}\gamma^{0}+i(2g^{0\nu} I_{4 \times 4}-\gamma^{\nu}\gamma^{0})\gamma^{2} \\
&=i\cdot2(g^{0\nu}\gamma^{2}-g^{2\nu}\gamma^{0})+\gamma^{\nu}C		% \cdot为小圆点
\end{aligned}	\tag{1.2.3}
\end{equation}
So we can say
\begin{equation}
\psi'^{\top}(x')\gamma^{\mu}C\psi'(x')=-\Lambda^{\mu}_{\;\nu}\left[i\cdot2\psi^{\top}(x)(g^{0\nu}\gamma^{2}-g^{2\nu}\gamma^{0})\psi(x)+\psi^{\top}(x)\gamma^{\mu}C\psi(x)\right]		\tag{1.2.4}
\end{equation}
As the same goes, when $\mu=1, 3$ \\
\begin{equation}
\psi'^{\top}(x')\gamma^{\mu}C\psi'(x')=\Lambda^{\mu}_{\;\nu}[i\cdot2\psi^{\top}(x)(g^{0\nu}\gamma^{2}-g^{2\nu}\gamma^{0})\psi(x)+\psi^{\top}(x)\gamma^{\mu}C\psi(x)]	
	\tag{1.2.5}
\end{equation}
We can get the flowing conclusion :
\begin{equation}
\psi'^{\top}(x')\gamma^{\mu}C\psi'(x')=\left\{		% 大括号公式的写法
\begin{aligned}
-\Lambda^{\mu}_{\;\nu}\left[i\cdot2\psi^{\top}(x)(g^{0\nu}\gamma^{2}-g^{2\nu}\gamma^{0})\psi(x)+\psi^{\top}(x)\gamma^{\mu}C\psi(x)\right] \quad \mu=0, 2\\
\Lambda^{\mu}_{\;\nu}\left[i\cdot2\psi^{\top}(x)(g^{0\nu}\gamma^{2}-g^{2\nu}\gamma^{0})\psi(x)+\psi^{\top}(x)\gamma^{\mu}C\psi(x)\right] \quad \mu=1, 3\\
\end{aligned}
\right.	% 注意这个点不要拉下\right.
	\tag{1.2.6}
\end{equation}
~\\
Because $C=i\gamma_{0}\gamma_{2}$, \\
\begin{equation}
\begin{aligned}
\psi'^{\top}(x')C\gamma^{\mu}\psi'(x') &= \psi^{\top}(x)S^{\top}C\gamma^{\mu}S\psi(x) =\psi^{\top}(x)\gamma^{3}\gamma^{1}S^{-1}\gamma^{1}\gamma^{3}i\gamma^{0}\gamma^{2}\gamma^{\mu}S\psi(x)\\
&= \psi^{\top}(x)\gamma^{3}\gamma^{1}S^{-1}\gamma^{5}\gamma^{\mu}\psi(x) \\
&= \psi^{\top}(x)\gamma^{3}\gamma^{1}(S^{-1}\gamma^{5}S)(S^{-1}\gamma^{\mu}S)\psi(x) \\
&= \Lambda^{\mu}_{\;\nu}\psi^{\top}(x)C\gamma^{\nu}\psi(x)
\end{aligned}	\tag{1.2.7}
\end{equation}
So we can finally get $\psi'^{\top}(x')\gamma^{\mu}C\psi'(x')=\Lambda^{\mu}_{\;\nu}\psi^{\top}(x)C\gamma^{\nu}\psi(x)$ when $\Gamma=\gamma^{\mu}$.\\
~\\
~\\
(3) $\displaystyle \Gamma=\sigma^{\mu\nu}=\frac{i}{2}[\gamma^{\mu},\gamma^{\nu}]$ \\ 
\begin{equation}
\begin{aligned}
\displaystyle \psi'^{\top}(x')C\sigma^{\mu\nu}\psi'(x') &= \psi^{\top}(x)S^{\top}C\sigma^{\mu\nu}S\psi(x)=\frac{i}{2}\psi^{\top}(x)S^{\top}C(\gamma^{\mu}\gamma^{\nu}-\gamma^{\nu}\gamma^{\mu})S\psi(x) \\
&= \frac{i}{2}(\psi^{\top}(x)S^{\top}C\gamma^{\mu}\gamma^{\nu}S\psi(x)-\psi^{\top}(x)S^{\top}C\gamma^{\nu}\gamma^{\mu}S\psi(x))
\end{aligned}	\tag{1.3.1}
\end{equation}
We should calculate $\psi^{\top}(x)S^{\top}C\gamma^{\mu}\gamma^{\nu}S\psi(x)$ and $\psi^{\top}(x)S^{\top}C\gamma^{\nu}\gamma^{\mu}S\psi(x)$ first :
\begin{equation}
\begin{aligned}
\displaystyle \psi^{\top}(x)S^{\top}C\gamma^{\mu}\gamma^{\nu}S\psi(x) &= \psi^{\top}(x)\gamma^{3}\gamma^{1}S^{-1}i\gamma^{0}\gamma^{1}\gamma^{2}\gamma^{3}\gamma^{\mu}\gamma^{\nu}S\psi(x) \\
&=\Lambda^{\mu}_{\;\alpha}\Lambda^{\nu}_{\;\beta}\psi^{\top}(x)C\gamma^{\alpha}\gamma^{\beta}\psi(x)
\end{aligned}	\tag{1.3.2}	
\end{equation}
\begin{equation}
\begin{aligned}
\displaystyle \psi^{\top}(x)S^{\top}C\gamma^{\nu}\gamma^{\mu}S\psi(x) &= -\psi^{\top}(x)\gamma^{3}\gamma^{1}S^{-1}\gamma^{1}\gamma^{3}i\gamma^{0}\gamma^{2}\gamma^{\mu}\gamma^{\nu}S\psi(x) \\
&=\Lambda^{\mu}_{\;\alpha}\Lambda^{\nu}_{\;\beta}\psi^{\top}(x)C\gamma^{\beta}\gamma^{\alpha}\psi(x)
\end{aligned}	\tag{1.3.3}	
\end{equation}
Then substitute Equ.1.3.2 and Equ.1.3.3 into Equ.1.3.1, we'll get :
\begin{align*}
\psi'^{\top}(x')C\sigma^{\mu\nu}\psi'(x') &= \frac{i}{2}(\Lambda^{\mu}_{\;\alpha}\Lambda^{\nu}_{\;\beta}\psi^{\top}(x)C\gamma^{\alpha}\gamma^{\beta}\psi(x)-\Lambda^{\mu}_{\;\alpha}\Lambda^{\nu}_{\;\beta}\psi^{\top}(x)C\gamma^{\beta}\gamma^{\alpha}\psi(x))\\
&=\frac{i}{2}\Lambda^{\mu}_{\;\alpha}\Lambda^{\nu}_{\;\beta}\psi^{\top}(x)C(\gamma^{\alpha}\gamma^{\beta}-\gamma^{\beta}\gamma^{\alpha})\psi(x)\\
&=\Lambda^{\mu}_{\;\alpha}\Lambda^{\nu}_{\;\beta}\psi^{\top}(x)C\sigma^{\alpha\beta}\psi(x)
	\tag{1.3.4}	
\end{align*}
We can finally get $\displaystyle \psi'^{\top}(x')C\sigma^{\mu\nu}\psi'(x')=\Lambda^{\mu}_{\;\alpha}\Lambda^{\nu}_{\;\beta}\psi^{\top}(x)C\sigma^{\alpha\beta}\psi(x)$ when $\displaystyle \Gamma=\sigma^{\mu\nu}=\frac{i}{2}[\gamma^{\mu},\gamma^{\nu}]$\\
~\\
~\\
(4) $\Gamma=\gamma^{\mu}\gamma^{5}$\\
\begin{equation}
\begin{aligned}
\displaystyle \psi'^{\top}(x')C\gamma^{\mu}\gamma^{5}\psi'(x') &=\psi^{\top}(x)S^{\top}C\gamma^{\mu}\gamma^{5}S\psi(x)=-\psi^{\top}(x)\gamma^{3}\gamma^{1}S^{-1}\gamma^{1}\gamma^{3}i\gamma^{0}\gamma^{2}\gamma^{\mu}\gamma^{5}S\psi(x) \\
&=\psi^{\top}(x)\gamma^{3}\gamma^{1}S^{-1}\gamma^{s}\gamma^{\mu}\gamma^{5}S\psi(x)=\psi^{\top}(x)\gamma^{3}\gamma^{1}\gamma^{5}\Lambda^{\mu}_{\;\nu}\gamma^{\nu}\gamma^{5}S\psi(x)
\end{aligned}		\tag{1.4.1}
\end{equation}
Then we have
\begin{equation}
\psi'^{\top}(x')C\gamma^{\mu}\gamma^{5}\psi'(x')=\Lambda^{\mu}_{\;\nu}\psi^{\top}(x)C\gamma^{\nu}\gamma^{5}\psi(x)
		\tag{1.4.2}
\end{equation}
We can finally get $\psi'^{\top}(x')C\gamma^{\mu}\gamma^{5}\psi'(x')=\Lambda^{\mu}_{\;\nu}\psi^{\top}(x)C\gamma^{\nu}\gamma^{5}\psi(x)$ when $\Gamma=\gamma^{\mu}\gamma^{5}$.\\
~\\
~\\
(5) $\Gamma=\gamma^{5}$
\begin{equation}
\begin{aligned}
\displaystyle \psi'^{\top}(x')C\gamma^{5}\psi'(x') &=\psi^{\top}(x)S^{\top}C\gamma^{5}S\psi(x)=-\psi^{\top}(x)\gamma^{3}\gamma^{1}S^{-1}\gamma^{1}\gamma^{3}i\gamma^{0}\gamma^{2}\gamma^{5}S\psi(x) \\
&=\psi^{\top}(x)\gamma^{3}\gamma^{1}S^{-1}\gamma^{5}\gamma^{5}S\psi(x)= \psi^{\top}(x)\gamma^{3}\gamma^{1}\gamma^{5}\gamma^{5}S\psi(x)\\
&=\psi^{\top}(x)C\gamma^{5}\psi(x)
\end{aligned}	\tag{1.5.1}
\end{equation}
We can finally get $\psi'^{\top}(x')C\gamma^{5}\psi'(x')=\psi^{\top}(x)C\gamma^{5}\psi(x)$ when $\Gamma=\gamma^{5}$.\\
~\\
~\\
~\\




% 第二题
% \iffalse 	% 多行注释,\iffalse和\fi之间被注释掉
% \fi			% 多行注释,\iffalse和\fi之间被注释掉
\textbf{2. Analyze the spin of Dirac field .}\\
~\\
\textbf{Answer\,:} We make $S_{i}$ to represent the component of spin angular momentum on the $\bm{\hat e_{i}}$-axis, so we have 	% \bm表示黑体公式
\begin{equation}
\displaystyle 
s_{3}=\pi_{a}S^{12}_{ab}\psi_{b}=\pi_{a}(\frac{1}{2}\gamma^{1}\gamma^{2})\psi_{b}=-\frac{i}{2}\pi_{a}(\Sigma_{3})_{ab}\psi_{b}
	\tag{2.1.1}
\end{equation}
Equ.2.1.1 describes the the spin density of fermions.
~\\
~\\

(1) Fermion \\
For $\pi_{a}=i\psi^{*}_{a}$, we have $\displaystyle s_{3}=\frac{1}{2}\psi^{*}_{a}(\Sigma_{3})_{ab}\psi_{b}$ , then we'll get $S_{3}$ :
\begin{equation}
\displaystyle 
S_{3}=\int\dif^{3}xs_{3}=\int\dif^{3}x\frac{1}{2}\psi^{*}_{a}(\Sigma_{3})_{ab}\psi_{b}
	\tag{2.1.2}
\end{equation}
For quantized Dirac field, we have
\begin{equation}
\displaystyle 
\hat S_{3}=\int\dif^{3}x\frac{1}{2}\hat\psi^{\dagger}_{a}(\Sigma_{3})_{ab}\hat\psi_{b}
	\tag{2.1.3}
\end{equation}
which
\begin{equation}
\displaystyle 
\hat\psi_{a}(\bm x,0)=\int \frac{\dif^{3} p}{(2\pi)^{3}}\frac{1}{\sqrt{2E_{\bm p}}}e^{i\bm p \cdot \bm x}\sum\limits_{s=1}^{2}\left[\hat a^{s}_{\bm p}u^{s}_{a}(\bm p)+\hat b^{s\dagger}_{\bm{-p}}v^{s}_{a}(\bm{-p})\right]
	\tag{2.1.4}
\end{equation}
and
\begin{equation}
\displaystyle 
\hat\psi^{\dagger}_{a}(\bm x,0)=\int \frac{\dif^{3} p}{(2\pi)^{3}}\frac{1}{\sqrt{2E_{\bm p}}}e^{-i\bm p \cdot \bm x}\sum\limits_{s=1}^{2}\left[\hat a^{s\dagger}_{\bm p}u^{s*}_{a}(\bm p)+\hat b^{s}_{\bm{-p}}v^{s*}_{a}(\bm{-p})\right]
	\tag{2.1.5}
\end{equation}
for $t=0$, $\bm p$ is 3-momentum, then we can factor $e^{\pm i\bm p \cdot \bm x}$ out before the sum.\\
Substitute  Equ.2.4 and Equ.2.5 into Equ.2.3, we'll get the expansion of $S_{3}$ :	% 注意此处我对过长的公式的操作,网上方法很多,但我觉得这是最简便的一种
\begin{equation}
\begin{aligned}	
\hat S_{3} &= \int\dif^{3}x \int \frac{\dif^{3} p'}{(2\pi)^{3}}\frac{1}{\sqrt{2E_{\bm p'}}}e^{i\bm p' \cdot \bm x}\sum\limits_{s'=1}^{2}\left[\hat a^{s'}_{\bm p}u^{s'}_{a}(\bm p')+\hat b^{s'\dagger}_{\bm{-p'}}v^{s'}_{a}(\bm{-p'})\right] \\ & \times \frac{1}{2}(\Sigma_{3})_{ab} \int \frac{\dif^{3} p}{(2\pi)^{3}}\frac{1}{\sqrt{2E_{\bm p}}}e^{-i\bm p \cdot \bm x}\sum\limits_{s=1}^{2}\left[\hat a^{s\dagger}_{\bm p}u^{s*}_{b}(\bm p)+\hat b^{s}_{\bm{-p}}v^{s*}_{b}(\bm{-p})\right]\\
&= \int\dif^{3}x \int \frac{\dif^{3} p'\dif^{3} p}{(2\pi)^{6}}\frac{1}{\sqrt{2E_{\bm p'}2E_{\bm p}}} e^{i\bm p' \cdot \bm x} e^{-i\bm p \cdot \bm x} \sum\limits_{s,s'}\left[\hat a^{s'}_{\bm p}u^{s'}_{a}(\bm p')+\hat b^{s'\dagger}_{\bm{-p'}}v^{s'}_{a}(\bm{-p'})\right]\\ & \times \frac{1}{2}(\Sigma_{3})_{ab} \left[\hat a^{s\dagger}_{\bm p}u^{s*}_{b}(\bm p)+\hat b^{s}_{\bm{-p}}v^{s*}_{b}(\bm{-p})\right] \\
\end{aligned}
	\tag{2.1.6}
\end{equation}
Suppose $\displaystyle \textcircled{1}=\sum\limits_{s,s'}\left[\hat a^{s'}_{\bm p}u^{s'}_{a}(\bm p')+\hat b^{s'\dagger}_{\bm{-p'}}v^{s'}_{a}(\bm{-p'})\right] \times \frac{1}{2}(\Sigma_{3})_{ab} \left[\hat a^{s\dagger}_{\bm p}u^{s*}_{b}(\bm p)+\hat b^{s}_{\bm{-p}}v^{s*}_{b}(\bm{-p})\right]$ ,\\
so
\begin{equation}
\begin{aligned}
\textcircled{1} &= \frac{1}{2}(\Sigma_{3})_{ab} \sum\limits_{s,s'}\left[\hat a^{s'}_{\bm p}u^{s'}_{a}(\bm p')+\hat b^{s'\dagger}_{\bm{-p'}}v^{s'}_{a}(\bm{-p'})\right]  \left[\hat a^{s\dagger}_{\bm p}u^{s*}_{b}(\bm p)+\hat b^{s}_{\bm{-p}}v^{s*}_{b}(\bm{-p})\right] \\
&= \frac{1}{2}(\Sigma_{3})_{ab} \sum\limits_{s,s'} \left[ { \hat a^{s'}_{\bm p}u^{s'}_{a}(\bm p')\hat a^{s\dagger}_{\bm p}u^{s*}_{b}(\bm p)+\hat b^{s'\dagger}_{\bm{-p'}}v^{s'}_{a}(\bm{-p'})\hat a^{s\dagger}_{\bm p}u^{s*}_{b}(\bm p)+ } \right. \\ & \left. { \hat a^{s'}_{\bm p}u^{s'}_{a}(\bm p')\hat b^{s}_{\bm{-p}}v^{s*}_{b}(\bm{-p})+\hat b^{s'\dagger}_{\bm{-p'}}v^{s'}_{a}(\bm{-p'})\hat b^{s}_{\bm{-p}}v^{s*}_{b}(\bm{-p}) } \right] \\
\end{aligned}
	\tag{2.1.7}
\end{equation}
When $\bm p=\bm0$
\begin{equation}
\hat S_{3} \hat a^{s\dagger}_{\bm0} \left|0\right> = \int\dif^{3}x\int \frac{\dif^{3} p'}{(2\pi)^{3}}\frac{1}{\sqrt{2E_{\bm p'}2E_{\bm p}}}e^{i\bm p' \cdot \bm x}e^{-i\bm p \cdot \bm x} \textcircled{1} \hat a^{s\dagger}_{\bm0} \left|0\right>
	\tag{2.1.8}
\end{equation}
and
\begin{equation}
\begin{aligned}
\textcircled{1}\hat a^{s\dagger}_{\bm0} \left|0\right> 
&= \frac{1}{2}(\Sigma_{3})_{ab} \sum\limits_{s,s'} \left[ { \hat a^{s'}_{\bm p}u^{s'}_{a}(\bm p')\hat a^{s\dagger}_{\bm p}u^{s*}_{b}(\bm p)\hat a^{s\dagger}_{\bm0} \left|0\right> +\hat b^{s'\dagger}_{\bm{-p'}}v^{s'}_{a}(\bm{-p'})\hat a^{s\dagger}_{\bm p}u^{s*}_{b}(\bm p)\hat a^{s\dagger}_{\bm0} \left|0\right> } \right. \\ & \left. { + \hat a^{s'}_{\bm p}u^{s'}_{a}(\bm p')\hat b^{s}_{\bm{-p}}v^{s*}_{b}(\bm{-p})\hat a^{s\dagger}_{\bm0} \left|0\right> +\hat b^{s'\dagger}_{\bm{-p'}}v^{s'}_{a}(\bm{-p'})\hat b^{s}_{\bm{-p}}v^{s*}_{b}(\bm{-p})\hat a^{s\dagger}_{\bm0} \left|0\right> } \right] \\
\end{aligned}
	\tag{2.1.9}
\end{equation}
of which
\begin{equation}
\{ \hat a^{r}_{\bm p}, \hat a^{s\dagger}_{\bm q} \}=\{ \hat b^{r}_{\bm p}, \hat b^{s\dagger}_{\bm q} \}=(2\pi)^{3}\delta^{3}(\bm p-\bm q)\delta_{rs}
	\tag{2.1.10}
\end{equation}
So 
\begin{equation}
\hat S_{3}\hat a^{s\dagger}_{\bm0} \left|0\right> = \frac{1}{2} \int \frac{\dif^{3} p}{(2\pi)^{3}}\frac{1}{2E_{\bm p}} (-1)^{s+1}2E_{\bm p}(2\pi)^{3}\delta^{3}(\bm p) \hat a^{s\dagger}_{\bm p} \left|0\right> = \frac{(-1)^{s+1}}{2}\hat a^{s\dagger}_{\bm 0} \left|0\right>
	\tag{2.1.11}
\end{equation}
For positive fermion field, we have $\displaystyle \hat S_{3}\hat a^{1\dagger}_{\bm0} \left|0\right> = \frac{1}{2} \hat a^{1\dagger}_{\bm0} \left|0\right> $ , and $\displaystyle \hat S_{3}\hat a^{2\dagger}_{\bm0} \left|0\right> = -\frac{1}{2} \hat a^{2\dagger}_{\bm0} \left|0\right> $ .\\
~\\
~\\

(2) Antifermion \\
As the same goes, we can get\\
\begin{equation}
\hat S_{3}\hat b^{s\dagger}_{\bm0} \left|0\right> = \int\dif^{3}x\int \frac{\dif^{3} p'}{(2\pi)^{3}}\frac{1}{\sqrt{2E_{\bm p'}2E_{\bm p}}}e^{i\bm p' \cdot \bm x}e^{-i\bm p \cdot \bm x} \textcircled{1} \hat b^{s\dagger}_{\bm0} \left|0\right>
	\tag{2.2.1}
\end{equation}
and
\begin{equation}
\begin{aligned}
\textcircled{1}\hat b^{s\dagger}_{\bm0} \left|0\right>  
&= \frac{1}{2}(\Sigma_{3})_{ab} \sum\limits_{s,s'} \left[ { \hat a^{s'}_{\bm p}u^{s'}_{a}(\bm p')\hat a^{s\dagger}_{\bm p}u^{s*}_{b}(\bm p)\hat b^{s\dagger}_{\bm0} \left|0\right>  +\hat b^{s'\dagger}_{\bm{-p'}}v^{s'}_{a}(\bm{-p'})\hat a^{s\dagger}_{\bm p}u^{s*}_{b}(\bm p)\hat b^{s\dagger}_{\bm0} \left|0\right> } \right. \\ & \left. { + \hat a^{s'}_{\bm p}u^{s'}_{a}(\bm p')\hat b^{s}_{\bm{-p}}v^{s*}_{b}(\bm{-p})\hat b^{s\dagger}_{\bm0} \left|0\right> +\hat b^{s'\dagger}_{\bm{-p'}}v^{s'}_{a}(\bm{-p'})\hat b^{s}_{\bm{-p}}v^{s*}_{b}(\bm{-p})\hat b^{s\dagger}_{\bm0} \left|0\right> } \right] 
\end{aligned}
	\tag{2.2.2}
\end{equation}
So we have
\begin{equation}
\begin{aligned}
\hat S_{3}\hat b^{s\dagger}_{\bm0} \left|0\right> &= -\frac{1}{2} \int \frac{\dif^{3} p}{(2\pi)^{3}}\frac{1}{2E_{\bm p}} (-1)^{s+1}2E_{\bm p}(2\pi)^{3}\delta^{3}(\bm p) \hat b^{s\dagger}_{\bm{-p}} \left|0\right> = -\frac{(-1)^{s+1}}{2}\hat b^{s\dagger}_{\bm 0} \left|0\right> \\
&=\frac{(-1)^{s}}{2}\hat b^{s\dagger}_{\bm 0} \left|0\right>
\end{aligned}
	\tag{2.2.3}
\end{equation}
For negative fermion field, we have $\displaystyle \hat S_{3}\hat b^{1\dagger}_{\bm0} \left|0\right> = -\frac{1}{2} \hat b^{1\dagger}_{\bm0} \left|0\right>$ , and $\displaystyle \hat S_{3}\hat b^{2\dagger}_{\bm0} \left|0\right> = \frac{1}{2} \hat b^{2\dagger}_{\bm0} \left|0\right> $ .\\
~\\
~\\
~\\



\textbf{3. Prove : (1) $\displaystyle C\bar\psi\gamma^{\mu}\psi C = -\bar\psi\gamma^{\mu}\psi$; \quad (2) $\displaystyle C\bar\psi\gamma^{\mu}\gamma^{5}\psi C = \bar\psi\gamma^{\mu}\gamma^{5}\psi$; \quad (3) $\displaystyle C\bar\psi\sigma^{\mu\nu}\psi C = -\bar\psi\sigma^{\mu\nu}\psi$, which $C$ is unitary linear operator  to represent charge conjugation.} \\		% \bar为字母正上方加一条短线
~\\
\textbf{Proof\,:} We have $\displaystyle \bar\psi=\psi^{\dagger}\gamma^{0}$, and $\displaystyle (\gamma^{5})^{\top}=\gamma^{5}$\\	% \top是转置符号
~\\
For bilinear Dirac field under charge conjugation transformation, we have $C\psi C = (-i\bar\psi\gamma^{0}\gamma^{2})^{\top}$, and $C\bar\psi C = (-i\gamma^{0}\gamma^{2}\psi)^{\top}$
\begin{align*}
C\bar\psi\psi C &= C\bar\psi CC \psi C = (-i\bar\psi\gamma^{0}\gamma^{2})^{\top} (-i\gamma^{0}\gamma^{2}\psi)^{\top} \\
&= - (\gamma^{0}\gamma^{2}\psi)^{\top}_{a} (\bar\psi\gamma^{0}\gamma^{2})^{\top}_{a} = - \gamma^{0}_{ab}\gamma^{2}_{bc}\psi_{c} \bar\psi_{d}\gamma^{0}_{de}\gamma^{2}_{ea} \\
&= - \gamma^{0}_{ab}\gamma^{2}_{bc}\psi_{c} \psi_{e}^{\dag}\gamma^{2}_{ea} \\
&= - \gamma^{0}_{ab}\gamma^{2}_{bc}\gamma^{2}_{ea} \left[ \{\psi_{c}(x),\psi_{e}^{\dag}(x)\} - \psi_{e}^{\dag}(x)\psi_{c}(x) \right] \\
&= - \gamma^{0}_{ab}\gamma^{2}_{bc}\gamma^{2}_{ca}\delta^{(3)}(\bm0) + \gamma^{2}_{ea}\psi^{\dag}_{e}(x)\gamma^{0}_{ab}\gamma^{2}_{bc}\psi_{c}(x) \\
&= \gamma^{0} \delta^{(3)}(\bm0) + \bar\psi(x)\gamma^{0}\gamma^{2}\gamma^{0}\gamma^{2}\psi(x)
\end{align*}
Therefore $\displaystyle C\bar\psi\psi C = \bar\psi\psi$. As the same goes, we have\\
~\\
(1) 
\begin{align*}
	C\bar\psi\gamma^{\mu}\psi C &= (-i\gamma^{0}\gamma^{2}\psi)^{\top}\gamma^{\mu}(-i\bar\psi\gamma^{0}\gamma^{2})^{\top}\\
	&= -\psi_{b} (\gamma^{2})^{\top}_{bc}(\gamma^{0})^{\top}_{ca} \gamma^{\mu}_{ab} (\gamma^{2})^{\top}_{bd} \psi^{\dagger}_{d} \\
	&= \psi^{\dagger}_{d} \gamma^{2}_{db} (\gamma^{\mu})^{\top}_{ba} \gamma^{0}_{ac}\gamma^{2}_{cb} \psi_{b} \\
	&= \psi^{\dagger}_{d} \gamma^{0}_{de}\gamma^{0}_{ef} \gamma^{2}_{fb} (\gamma^{\mu})^{\top}_{ba} \gamma^{0}_{ac}\gamma^{2}_{cb} \psi_{b} \\
	&= \bar\psi_{e} \gamma^{0}_{ef} \gamma^{2}_{fb} (\gamma^{\mu})^{\top}_{ba} \gamma^{0}_{ac}\gamma^{2}_{cb} \psi_{b} \\
	&= \bar\psi \gamma^{0}\gamma^{2} (\gamma^{\mu})^{\top} \gamma^{0}\gamma^{2} \psi\\
	\tag{3.1.1}
\end{align*}
When $\mu=0, 2$, $(\gamma^{\mu})^{\top} = \gamma^{\mu}$
\begin{equation}
\bar\psi \gamma^{0}\gamma^{2} (\gamma^{\mu})^{\top} \gamma^{0}\gamma^{2} = \bar\psi (-1)\gamma^{\mu} \psi = -\bar\psi \gamma^{\mu} \psi
	\tag{3.1.2}
\end{equation}
and if $\mu=1, 3$, $(\gamma^{\mu})^{\top} = -\gamma^{\mu}$
\begin{equation}
\bar\psi \gamma^{0}\gamma^{2} (\gamma^{\mu})^{\top} \gamma^{0}\gamma^{2} = \bar\psi (-\gamma^{\mu}) \psi = -\bar\psi \gamma^{\mu} \psi
	\tag{3.1.3}
\end{equation}
So we can say $\displaystyle C\bar\psi\gamma^{\mu}\psi C = -\bar\psi \gamma^{\mu} \psi$ . \\
Q.E.D.\\
~\\
~\\
(2) 
\begin{equation}
\begin{aligned}
	C\bar\psi\gamma^{\mu}\psi C &= (-i\gamma^{0}\gamma^{2}\psi)^{\top}\gamma^{\mu}\gamma^{5}(-i\bar\psi\gamma^{0}\gamma^{2})^{\top}\\
	&= -\psi_{f} (\gamma^{2})^{\top}_{fe}(\gamma^{0})^{\top}_{ea} \gamma^{\mu}_{ab}\gamma^{5}_{bc} (\gamma^{2})^{\top}_{cd} \psi^{\dagger}_{d} \\
	&= -\psi^{\dagger}_{d} \gamma^{2}_{dc} (\gamma^{5})^{\top}_{cb}(\gamma^{\mu})^{\top}_{ba} \gamma^{0}_{ae}\gamma^{2}_{ef} \psi_{f} \\
	&= -\bar\psi_{d}  \gamma^{0}_{de}\gamma^{2}_{ec} \gamma^{5}_{cb}(\gamma^{\mu})^{\top}_{ba} \gamma^{0}_{ae}\gamma^{2}_{ef} \psi_{f} \\
	&= -\bar\psi \gamma^{0}\gamma^{2} \gamma^{5}(\gamma^{\mu})^{\top} \gamma^{0}\gamma^{2} \psi
\end{aligned}
	\tag{3.2.1}
\end{equation}
When $\mu=0, 2$, $(\gamma^{\mu})^{\top} = \gamma^{\mu}$
\begin{equation}
-\bar\psi \gamma^{0}\gamma^{2} \gamma^{5}(\gamma^{\mu})^{\top} \gamma^{0}\gamma^{2} \psi = -\bar\psi \gamma^{5}\gamma^{\mu} \psi = \bar\psi \gamma^{\mu}\gamma^{5} \psi
	\tag{3.2.2}
\end{equation}
and if $\mu=1, 3$, $(\gamma^{\mu})^{\top} = -\gamma^{\mu}$
\begin{equation}
-\bar\psi \gamma^{0}\gamma^{2} \gamma^{5}(\gamma^{\mu})^{\top} \gamma^{0}\gamma^{2} \psi = -\bar\psi (-\gamma^{\mu})\gamma^{5} \psi = \bar\psi \gamma^{\mu}\gamma^{5} \psi
	\tag{3.2.3}
\end{equation}
So we can say $\displaystyle C\bar\psi\gamma^{\mu}\gamma^{5}\psi C = \bar\psi \gamma^{\mu}\gamma^{5} \psi$ . \\
Q.E.D.\\
~\\
~\\
(3) For $\displaystyle \sigma^{\mu\nu}=\frac{i}{2} [\ \gamma^{\mu}, \gamma^{\nu} ]\ = \frac{i}{2}(\gamma^{\mu}\gamma^{\nu}-\gamma^{\nu}\gamma^{\mu})$ ,\\
we have \\
\begin{equation}
\begin{aligned}
C \bar\psi\gamma^{\mu}\gamma^{\nu}\psi C &= (-i\gamma^{0}\gamma^{2}\psi)^{\top}\gamma^{\mu}\gamma^{\nu}(-i\bar\psi\gamma^{0}\gamma^{2})^{\top} \\
&= -\psi_{f} (\gamma^{2})^{\top}_{fe}(\gamma^{0})^{\top}_{ea} \gamma^{\mu}_{ab}\gamma^{\nu}_{bc} (\gamma^{2})^{\top}_{cd} \psi^{\dagger}_{d} \\
&= -\psi^{\dagger}_{d} \gamma^{2}_{dc} (\gamma^{\nu})^{\top}_{cb}(\gamma^{\mu})^{\top}_{ba} \gamma^{0}_{ae}\gamma^{2}_{ef} \psi_{f} \\
&= -\bar\psi \gamma^{0}\gamma^{2} (\gamma^{\nu})^{\top}(\gamma^{\mu})^{\top} \gamma^{0}\gamma^{2} \psi\\
\end{aligned}
	\tag{3.3.1}
\end{equation}
$\bullet$ When $\mu, \nu = 0, 2$ , or $\mu, \nu = 1, 3$ . We have
\begin{equation}
-\bar\psi \gamma^{0}\gamma^{2} (\gamma^{\nu})^{\top}(\gamma^{\mu})^{\top} \gamma^{0}\gamma^{2} \psi = \bar\psi \gamma^{\nu}\gamma^{\mu} \psi = -\bar\psi \gamma^{\mu}\gamma^{\nu} \psi
	\tag{3.3.2}
\end{equation}
$\bullet$ Or $\mu,\nu = 0,2$ , or $\mu,\nu = 1,3$ . We have
\begin{equation}
-\bar\psi \gamma^{0}\gamma^{2} (\gamma^{\nu})^{\top}(\gamma^{\mu})^{\top} \gamma^{0}\gamma^{2} \psi = \bar\psi (-\gamma^{\nu})(-\gamma^{\mu}) \psi = -\bar\psi \gamma^{\mu}\gamma^{\nu} \psi
	\tag{3.3.3}
\end{equation}
From Equ.3.3.2 and Equ.3.3.3, we can get $C \bar\psi\gamma^{\mu}\gamma^{\nu}\psi C = -\bar\psi \gamma^{\mu}\gamma^{\nu} \psi$ . \\
And as the same goes, we have $C \bar\psi\gamma^{\nu}\gamma^{\mu}\psi C = -\bar\psi \gamma^{\nu}\gamma^{\mu} \psi$ . \\
Then we obtain
\begin{equation}
\begin{aligned}
C\bar\psi\sigma^{\mu\nu}\psi C &= \frac{i}{2}C\bar\psi[\gamma^{\mu},\gamma^{\nu}]\psi C\\
&= \frac{i}{2}C\bar\psi(\gamma^{\mu}\gamma^{\nu}-\gamma^{\nu}\gamma^{\mu})\psi C\\
&= \frac{i}{2}\bar\psi(\gamma^{\mu}\gamma^{\nu}-\gamma^{\nu}\gamma^{\mu})\psi \\
&= \bar\psi\sigma^{\nu\mu}\psi =-\bar\psi\sigma^{\mu\nu}\psi \\
\end{aligned}
	\tag{3.3.3}
\end{equation}
So we can say $\displaystyle C\bar\psi\sigma^{\mu\nu}\psi C = -\bar\psi\sigma^{\mu\nu}\psi$ .\\
Q.E.D.\\
~\\
~\\
~\\

% 第四题
\textbf{4. Prove the commutation relation of creation-annihilation operators of photon field.} \\		% \top是转置符号
~\\
\textbf{Proof\,:} With the expressions of vector field operator in Equ.4.1.1 and conjugate momentum in Equ.4.1.2 below
\begin{equation}
\bm{A}(x) = \int \frac{\dif^{3}k}{(2\pi)^{3}} \frac{1}{\sqrt{2E_{\bm k}}} \sum\limits_{\lambda=1,2} \bm\epsilon^{(\lambda)}(\bm k) \left( a_{\lambda}(\bm k)e^{ik \cdot x}+a_{\lambda}^{\dagger}(\bm k)e^{-ik \cdot x} \right) 
% 利用\left( ... \right)使括号显示更美观,类似有\left\{ ... \right\}
% 对于长公式换行,如果涉及到从长公式括号中的某一处截断,请参考https://blog.csdn.net/i10630226/article/details/44700361的方法
	\tag{4.1.1}
\end{equation}

\begin{equation}
\bm{\pi}(x) = i \int \frac{\dif^{3}k}{(2\pi)^{3}} \sqrt{\frac{E_{\bm k}}{2}} \sum\limits_{\lambda=1,2} \bm\epsilon^{(\lambda)}(\bm k) \left(a_{\lambda}(\bm k)e^{ik \cdot x}-a_{\lambda}^{\dagger}(\bm k)e^{-ik \cdot x}\right) 	% 利用\left( ... \right)使括号显示更美观
	\tag{4.1.2}
\end{equation}
which polarization vector $\bm\epsilon$ satisfies $\bm k \cdot \bm\epsilon^{(\lambda)}(\bm k)=0$ , and three space-like polarization vectors $\bm\epsilon^{(\lambda)}(\bm k)$ are orthogonal to $\bm k$
\begin{equation}
\bm\epsilon^{(\lambda)}(\bm k) \cdot \bm\epsilon^{(\lambda')}(\bm k) = -\delta_{\lambda\lambda'} \qquad \sum\limits_{\lambda=1}^{3} \epsilon^{(\lambda)}_{\rho}(k)\epsilon^{(\lambda)}_{\nu}(k) = -\left( g_{\rho\nu}-\frac{k_{\rho}k_{\nu}}{\mu^{2}} \right)
	\tag{4.1.3}
\end{equation}
Now we can work out the expression of annihilation operator
\begin{equation}
a_{\lambda=1,2}(\bm k) = \int \dif^{3}x e^{-i\bm k \cdot \bm x}
	\tag{4.1.4}
\end{equation}
And creation operator $a_{\lambda}^{\dagger}(\bm k)$ is
\begin{equation}
a_{\lambda=1,2}^{\dagger}(\bm k) = \int \dif^{3}x e^{i\bm k \cdot \bm x}
	\tag{4.1.5}
\end{equation}
Then we have
\begin{equation}
\begin{aligned}
\left[a_{\lambda}(\bm k), a_{\lambda'}^{\dagger}(\bm k')\right] &= -\int \dif^{3}x\dif^{3}x'e^{-ik \cdot x}e^{ik' \cdot x'} \\		% 利用\left[ ... \right]使括号显示更美观
&= \\
&= \delta_{\lambda\lambda'}\delta^{(3)}(\bm k - \bm k')
\end{aligned}
	\tag{4.1.6}
\end{equation}
So we verify that creation-annihilation operators of vector field have commutation relation which is $\left[a_{\lambda}(\bm k), a_{\lambda'}^{\dagger}(\bm k')\right] = \delta_{\lambda\lambda'}\delta^{(3)}(\bm k - \bm k')$





\end{spacing} 		% 结束行距 

\end{document}  