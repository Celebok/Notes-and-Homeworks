% XeLaTeX can use any Mac OS X font. See the setromanfont command below.
% Input to XeLaTeX is full Unicode, so Unicode characters can be typed directly into the source.

% The next lines tell TeXShop to typeset with xelatex, and to open and save the source with Unicode encoding.

%!TEX TS-program = xelatex
%!TEX encoding = UTF-8 Unicode

\documentclass[12pt]{article}
\usepackage{geometry}                % See geometry.pdf to learn the layout options. There are lots.
\geometry{letterpaper}                   % ... or a4paper or a5paper or ... 
%\geometry{landscape}                % Activate for for rotated page geometry
%\usepackage[parfill]{parskip}    % Activate to begin paragraphs with an empty line rather than an indent
\usepackage{graphicx}
\usepackage{amssymb}

\usepackage{xeCJK} 			% 中日韩语对应宏包
\usepackage{amsmath,bm}           % 数学符号与特殊字符
\usepackage{ragged2e}          % 两端对齐
\usepackage{color}             	% 字体颜色
\usepackage{indentfirst}		% 首行缩进
\geometry{left=1.5cm,right=2.5cm,top=2.0cm,bottom=2.5cm}
\usepackage{setspace}		%使用间距宏包
\usepackage{mathrsfs}		% 花体字母,然后使用\mathscr{A}命令
\usepackage{simplewick}		% 场论中收缩符号专用宏包,非必要
\usepackage{float}
% Will Robertson's fontspec.sty can be used to simplify font choices.
% To experiment, open /Applications/Font Book to examine the fonts provided on Mac OS X,
% and change "Hoefler Text" to any of these choices.

\usepackage{fontspec,xltxtra,xunicode}
\defaultfontfeatures{Mapping=tex-text}
\setromanfont[Mapping=tex-text]{Hoefler Text}
\setsansfont[Scale=MatchLowercase,Mapping=tex-text]{Gill Sans}
\setmonofont[Scale=MatchLowercase]{Andale Mono}
\setCJKmainfont{华文宋体}		% 输出中文,我的中文字体是系统自带的华文宋体

\title{\Huge{高量往年考试题}}
\author{梁天笑\textsuperscript{1} \and 盛炳开\textsuperscript{2} \and 梁浩\textsuperscript{3}}	% \textsuperscript{n}为第n作者
\date{2019年1月14日}

\numberwithin{equation}{section}	 % 我们可使用amsmath 宏包提供的numberwithin命令来实现公式与章节的关联
\allowdisplaybreaks[4]			 % 允许多行公式跨页


\begin{document}
\maketitle
\setlength{\parindent}{0pt}	% 取消首行缩进,如果需要缩进,则将0pt设置为所需数量的pt,一般为2pt
\thispagestyle{empty}		% 封面页不显示页码


\newpage
\thispagestyle{empty}		% 目录页不显示页码
\renewcommand\contentsname{\centering{目\qquad 录}}
\tableofcontents


\begin{spacing}{1.5}			% 行间距变为double-space
\newcommand*{\dif}{\mathop{}\!\mathrm{d}} 		% \mathop{}用来输出直立黑体的微分算子,如\mathop{d},为了简略我们使用\dif代替\mathop{d}


\newpage
\setcounter{page}{1}			% 从下面开始编页码
\section{2004年习题~2010年习题}
%%%%%%%%%%%%%%%%%%%%%%%%%%%%%%%%%%%%%%%%%%%%%%%%%%
%%%%%%%%%%%%%%%%%%%% 2004年期末题 %%%%%%%%%%%%%%%%%%%%%%%
%%%%%%%%%%%%%%%%%%%%%%%%%%%%%%%%%%%%%%%%%%%%%%%%%%
\subsection{2004年习题}
~\\
~\\
\textbf{1 \quad 试证明幺正算符$U$与复共轭算符$K$的乘积为反幺正算符。}\\
证明:$\because$		 % 表示因为,\therefore表示所以 \nabla为∇
\begin{equation}
(UK)\sum_{n}c_{n}\psi_{n} = U\sum_{n}c^{*}_{n}K\psi_{n} = \sum_{n}c^{*}_{n}(UK)\psi_{n}
\end{equation}
$\therefore UK$具有反线性。下面需要证明$(UK)^{-1} = (UK)^{\dagger}$。\\
$\because$
\begin{equation}\nonumber 		% \nonumber使该公式不编号
(UK)(UK)^{\dagger} = UK \cdot KU^{\dagger} = UK^{2}U^{\dagger} = UU^{\dagger} = 1,\text{其中}K^{2}=1
\end{equation}
$\therefore$ 可得$(UK)^{-1} = KU^{\dagger}$。\\
又$\because$
\begin{equation}
\begin{aligned}
\int \psi^{*}(UK)\phi\dif\tau &= \int \psi^{*}U\phi^{*}\dif\tau = \int (U^{\dagger}\psi)^{*}\phi^{*}\dif\tau \\
&= \int \psi^{*}K(U^{\dagger}\phi)\dif\tau
\end{aligned}
\end{equation}
$\therefore$ 可得$(UK)^{\dagger} = KU^{\dagger}$,即有$(UK)^{\dagger} = (UK)^{-1}$,$UK$为反幺正算符。\\
~\\
~\\
\textbf{2 \quad 已知在$\hat{s}_{z}$表象中,$\displaystyle \hat{s}_{x}=\frac{\hbar}{2} \begin{bmatrix} 0 & 1 \\ 1 & 0 \end{bmatrix}$,$\displaystyle \hat{s}_{y}=\frac{\hbar}{2} \begin{bmatrix} 0 & -i \\ i & 0 \end{bmatrix}$,求$\hat{s}_{y}$在$\hat{s}_{x}$表象中的矩阵表示。}\\
~\\
解:$\hat{s}_{x}$在$\hat{s}_{z}$表象中对应的本征基矢为$\displaystyle \frac{1}{\sqrt{2}}\begin{pmatrix} 1 \\ 1 \end{pmatrix} \text{和} \frac{1}{\sqrt{2}}\begin{pmatrix} 1 \\ -1 \end{pmatrix}$,变换矩阵的表达形式为$\displaystyle U=\frac{1}{\sqrt{2}} \begin{bmatrix} 1 & 1 \\ 1 & -1 \end{bmatrix}$,\\
~\\
$\therefore$
\begin{equation}
\hat{s}_{y} = U \;\frac{\hbar}{2} \begin{bmatrix} 0 & -i \\ i & 0 \end{bmatrix}U^{-1} = \frac{\hbar}{2} \begin{bmatrix} 0 & i \\ -i & 0 \end{bmatrix}
\end{equation}
$\hat{s}_{y}$在$\hat{s}_{x}$表象中的矩阵表示为$\displaystyle \frac{\hbar}{2} \begin{bmatrix} 0 & i \\ -i & 0 \end{bmatrix}$。\\
~\\
~\\
\textbf{3 \quad 试证明在空间转动变换下$\displaystyle I=\sum_{m} Y^{*}_{lm}(\theta,\phi)Y_{lm}(\theta,\phi)$保持不变,式中$Y_{lm}(\theta,\phi)$为球谐函数。}\\
证明:在空间转动变换下\\
\begin{equation}\nonumber 		% \nonumber使该公式不编号
\begin{aligned}
\sum_{m} Y^{*}_{lm}(\theta',\phi')Y_{lm}(\theta',\phi') &= \sum_{m}\sum_{m'm''}D^{l*}_{m'm}(\alpha\beta\gamma)Y^{*}_{lm}(\theta,\phi)D^{l}_{m''m}(\alpha\beta\gamma)Y_{lm}(\theta,\phi)\\
&= \sum_{m}\sum_{m'm''}\delta_{m'm''}Y^{*}_{lm}(\theta,\phi)Y_{lm}(\theta,\phi)\\
&= \sum_{m} Y^{*}_{lm}(\theta,\phi)Y_{lm}(\theta,\phi)
\end{aligned}
\end{equation}
$\therefore$在空间转动变换下$\displaystyle I=\sum_{m} Y^{*}_{lm}(\theta,\phi)Y_{lm}(\theta,\phi)$保持不变。\\
~\\
~\\
\textbf{4 \quad 试利用一阶张量投影定理计算电子磁矩在$\left|jm\right>={\displaystyle\sum\limits_{m_{l}m_{s}}C^{jm}_{lm_{l}\frac{1}{2}m_{s}}}\left|lm_{l}\right>\left|\frac{1}{2}m_{s}\right>$}态上的平均值。电子的磁矩算符为$\mu=\mu_{0}(g_{L}\bm{l}+g_{S}\bm{s})$,其中$\displaystyle \mu_{0}=-\frac{e}{2m_{e}c}$;$g_{L}=1$和$g_{S}=2$分别为电子的轨道和自旋朗德因子,$m_{e}$为电子静止质量。\\
解:由$Wignet-Eckart$定理
\begin{equation}
\left<jm\right| \mu \left|jm\right> = C^{jm}_{lm1M}\left< j \right\| \mu_{0} \left\| j \right> \\
\end{equation}
其中$m=m+M$,$\therefore$ $M=0$。\\
$\therefore$由投影定理
\begin{equation}
\begin{aligned}
\left<jm\right| \mu \left|jm\right> &= \left< jm \right| \mu_{0} \left| jm \right> = \frac{\left< jm \right| \hat{J}_{0} \left| jm \right> \left< jm \right| \hat{J}\cdot\hat{\mu} \left| jm \right>}{j(j+1)\hbar^{2}}\\
&= m\hbar\frac{\left< jm \right| \hat{J}\cdot\hat{\mu} \left| jm \right>}{j(j+1)\hbar^{2}}
\end{aligned}
\end{equation}
又$\because$
\begin{align}\nonumber 		% \nonumber使该公式不编号
\hat{J}\cdot\hat{\mu} &= \mu_{0}(g_{L}\hat{J}\cdot\hat{L}+g_{S}\hat{J}\cdot\hat{S}) \notag \\
&= \frac{1}{2}\mu_{0} [ g_{L}(\hat{J}^{2}+\hat{L}^{2}-\hat{S}^{2})+g_{S}(\hat{J}^{2}+\hat{S}^{2}-\hat{L}^{2}) ] \notag \\
&= \frac{1}{2}\mu_{0} [ (g_{L}+g_{S})\hat{J}^{2}+(g_{L}-g_{S})(\hat{L}^{2}-\hat{S}^{2}) ] \notag \\
\end{align}

$\therefore$
\begin{equation}\nonumber 		% \nonumber使该公式不编号
\begin{aligned}
\left< jm \right| \mu_{0} \left| jm \right> &= m\hbar \frac{\left< jm \right| \hat{J}\cdot\hat{\mu} \left| jm \right>}{j(j+1)\hbar^{2}} \\
&= \frac{\mu_{0}m\hbar}{2j(j+1)}\{ g_{L}[j(j+1)+l(l+1)-s(s+1)]+g_{s}[j(j+1)+s(s+1)-l(l+1)] \} \\
&= \frac{\mu_{0}m\hbar}{2}\left[ (g_{L}+g_{S})+(g_{L}-g_{S})\frac{l(l+1)-s(s+1)}{j(j+1)} \right]
\end{aligned}
\end{equation}
~\\
~\\
\textbf{5 \quad 两个角动量$\bm j_{1},\bm j_{2}$耦合成总角动量$\bm J$,试推导出约化矩阵元$\displaystyle\left< j'_{1}j'_{2}J \left\| \hat{T}_{L}(1) \right\| j_{1}j_{2}J \right>$的表达式,式中$\displaystyle \left| j_{1}j_{2}J\right>=\sum\limits_{m_{1}m_{2}}C^{JM}_{j_{1}m_{1}j_{2}m_{2}}\left| j_{1}m_{1}\right>\left| j_{2}m_{2}\right>$,$\hat{T}_{L}(1)$为作用于第一个角动量的不可约张量算符。}\\
解:由$Wignet-Eckart$定理
\begin{equation}
\left< j'_{1}j'_{2}J \left\| \hat{T}_{L}(1) \right\| j_{1}j_{2}J \right> = \frac{\left< j'_{1}j'_{2}J \left| \hat{T}_{L}(1) \right| j_{1}j_{2}J \right>}{C^{JM}_{JMj_{1}m_{1}}}
\end{equation}
对$C^{JM}_{JMj_{1}m_{1}}$有$J=J+j_{1}$和$M=M+m_{1}$,所以$j_{1}=m_{1}=0$,$C^{JM}_{JMj_{1}m_{1}}=C^{JM}_{JM00}=1$。\\
对$\bm J=\bm j_{1}+\bm j_{2}$,当$j_{1}=0$时,$j_{2}=J$。\\
$\therefore$
\begin{equation}
\left| j_{1}j_{2}J \right> = \left| 0JJ \right> = C^{JM}_{00JM}\left|00\right> \left|JM\right> = \left|00\right> \left|JM\right>
\end{equation}
$\therefore$
\begin{align} 		% 公式自动断页
\left< j'_{1}j'_{2}J \left\| \hat{T}_{L}(1) \right\| j_{1}j_{2}J \right> &= \left< j'_{1}j'_{2}J \left| \hat{T}_{L}(1) \right| j_{1}j_{2}J \right> = \left< j'_{1}j'_{2}J \left| \hat{T}_{L}(1) \right| 0JJ \right> \notag \\		% \notag删除align对每一行的自动编号
&= \sum_{m'_{1}m'_{2}}C^{JM}_{j'_{1}m'_{1}j'_{2}m'_{2}}\left<j'_{1}m'_{1}\left| \hat{T}_{L}(1) \right|00\right>\left<j'_{2}m'_{2}|JM\right> \notag \\
&= \sum_{m'_{1}m'_{2}}C^{JM}_{j'_{1}m'_{1}j'_{2}m'_{2}}\left<j'_{1}m'_{1}\left| \hat{T}_{L}(1) \right|00\right>\delta_{j'_{2}J}\delta_{m'_{2}M} \notag \\
&= \sum_{m'_{1}}C^{JM}_{j'_{1}m'_{1}JM}\left<j'_{1}m'_{1}\left| \hat{T}_{L}(1) \right|00\right> \notag
\end{align}
类似,在$C^{JM}_{j'_{1}m'_{1}JM}$中,$j'_{1}=m'_{1}=0$,$C^{JM}_{j'_{1}m'_{1}JM}=1$。\\
~\\
$\therefore \left< j'_{1}j'_{2}J \left\| \hat{T}_{L}(1) \right\| j_{1}j_{2}J \right>=\left<00\left| \hat{T}_{L}(1) \right|00\right>$
~\\
~\\
\textbf{6 \quad 设两个独立的谐振子组成一个体系,以$n_{1},n_{2}$分别表示二者的量子数,以$\hat{a}^{\dagger}_{1},\hat{a}_{1},\hat{a}^{\dagger}_{2},\hat{a}_{2}$分别表示二者的产生消灭算符,粒子数表象中的归一化本征态记为$\left| n_{1}n_{2} \right>$。令$a=\begin{pmatrix} \hat{a}_{1} \\ \hat{a}_{2} \end{pmatrix}$,定义算符$\displaystyle \hat{\bm{J}}=\frac{1}{2}\hat{a}^{\dagger}\sigma\hat{a}$,其中$\sigma$为泡利矩阵。(1)写出$\bm J$的各个分量的表达式。(2)证明如此定义的$\hat{\bm{J}}$满足角动量算符的全部代数性质。(3)求出$\hat{\bm{J}^{2}},\hat{\bm{J}_{z}}$的本征值。}\\
解:\\
(1) 
\begin{equation}
\begin{aligned}
\hat{J}_{x} &= \frac{1}{2}\left(\hat{a}^{\dagger}_{1},\hat{a}^{\dagger}_{2}\right) \begin{bmatrix} 0 & 1 \\ 1 & 0 \end{bmatrix}\begin{pmatrix} \hat{a}_{1} \\ \hat{a}_{2} \end{pmatrix} = \frac{1}{2}(\hat{a}^{\dagger}_{2}\hat{a}_{1} + \hat{a}^{\dagger}_{1}\hat{a}_{2}) \\
\hat{J}_{y} &= \frac{1}{2}\left(\hat{a}^{\dagger}_{1},\hat{a}^{\dagger}_{2}\right) \begin{bmatrix} 0 & -i \\ i & 0 \end{bmatrix}\begin{pmatrix} \hat{a}_{1} \\ \hat{a}_{2} \end{pmatrix} = \frac{i}{2}(\hat{a}^{\dagger}_{2}\hat{a}_{1} - \hat{a}^{\dagger}_{1}\hat{a}_{2}) \\
\hat{J}_{z} &= \frac{1}{2}\left(\hat{a}^{\dagger}_{1},\hat{a}^{\dagger}_{2}\right) \begin{bmatrix} 1 & 0 \\ 0 & -1 \end{bmatrix}\begin{pmatrix} \hat{a}_{1} \\ \hat{a}_{2} \end{pmatrix} = \frac{1}{2}(\hat{a}^{\dagger}_{1}\hat{a}_{1} - \hat{a}^{\dagger}_{2}\hat{a}_{2})
\end{aligned}
\end{equation}
(2)
\begin{equation}
\begin{aligned}
\left[ \hat{J}_{x},\hat{J}_{y} \right] &= \hat{J}_{x}\hat{J}_{y}-\hat{J}_{y}\hat{J}_{x} = \frac{i}{4} \left[(\hat{a}^{\dagger}_{2}\hat{a}_{1} + \hat{a}^{\dagger}_{1}\hat{a}_{2}) (\hat{a}^{\dagger}_{2}\hat{a}_{1} - \hat{a}^{\dagger}_{1}\hat{a}_{2}) - (\hat{a}^{\dagger}_{2}\hat{a}_{1} - \hat{a}^{\dagger}_{1}\hat{a}_{2}) (\hat{a}^{\dagger}_{2}\hat{a}_{1} + \hat{a}^{\dagger}_{1}\hat{a}_{2})\right]\\
&= \frac{i}{4} \left( { \hat{a}^{\dagger}_{2}\hat{a}_{1}\hat{a}^{\dagger}_{2}\hat{a}_{1} + \hat{a}^{\dagger}_{1}\hat{a}_{2}\hat{a}^{\dagger}_{2}\hat{a}_{1} - \hat{a}^{\dagger}_{2}\hat{a}_{1}\hat{a}^{\dagger}_{1}\hat{a}_{2} - \hat{a}^{\dagger}_{1}\hat{a}_{2}\hat{a}^{\dagger}_{1}\hat{a}_{2} - \hat{a}^{\dagger}_{2}\hat{a}_{1}\hat{a}^{\dagger}_{2}\hat{a}_{1} - \hat{a}^{\dagger}_{2}\hat{a}_{1}\hat{a}^{\dagger}_{1}\hat{a}_{2} } \right. \\
&\left. {+ \hat{a}^{\dagger}_{1}\hat{a}_{2}\hat{a}^{\dagger}_{2}\hat{a}_{1} + \hat{a}^{\dagger}_{1}\hat{a}_{2}\hat{a}^{\dagger}_{1}\hat{a}_{2} } \right)\\
\end{aligned}
\end{equation}
% 对于长公式换行,如果涉及到从长公式括号中的某一处截断,请参考https://blog.csdn.net/i10630226/article/details/44700361的方法
又$\because$\\
\begin{equation}
\begin{aligned}
\hat{a}^{\dagger}_{1}\hat{a}_{2}\hat{a}^{\dagger}_{2}\hat{a}_{1} - \hat{a}^{\dagger}_{2}\hat{a}_{1}\hat{a}^{\dagger}_{1}\hat{a}_{2} &= \left[ \hat{a}^{\dagger}_{1}\hat{a}_{2},\hat{a}^{\dagger}_{2}\hat{a}_{1} \right] = \hat{a}^{\dagger}_{1}\left[ \hat{a}_{2},\hat{a}^{\dagger}_{2}\hat{a}_{1} \right] + \left[ \hat{a}^{\dagger}_{1},\hat{a}^{\dagger}_{2}\hat{a}_{1} \right]\hat{a}_{2} \\
&= \hat{a}^{\dagger}_{1}\left[ \hat{a}_{2},\hat{a}^{\dagger}_{2} \right]\hat{a}_{1} + \hat{a}^{\dagger}_{2}\left[ \hat{a}^{\dagger}_{1},\hat{a}_{1} \right]\hat{a}_{2} \\
&= \hat{a}^{\dagger}_{1}\hat{a}_{1} - \hat{a}^{\dagger}_{2}\hat{a}_{2}
\end{aligned}
\end{equation}
$\therefore$\\
\begin{equation}
\begin{aligned}
\left[ \hat{J}_{x},\hat{J}_{y} \right] &= \frac{i}{4} \cdot 2 \left( \hat{a}^{\dagger}_{1}\hat{a}_{2}\hat{a}^{\dagger}_{2}\hat{a}_{1} - \hat{a}^{\dagger}_{2}\hat{a}_{1}\hat{a}^{\dagger}_{1}\hat{a}_{2} \right) =\frac{i}{2} \left(\hat{a}^{\dagger}_{1}\hat{a}_{1} - \hat{a}^{\dagger}_{2}\hat{a}_{2}\right)\\
&= i\hat{J}_{z}
\end{aligned}
\end{equation}
$\therefore \left[ \hat{J}_{x},\hat{J}_{y} \right] = i\hat{J}_{z}$,类似有$\left[ \hat{J}_{y},\hat{J}_{z} \right] = i\hat{J}_{x}$和$\left[ \hat{J}_{z},\hat{J}_{x} \right] = i\hat{J}_{y}$。所以
\begin{equation}\nonumber 		% \nonumber使该公式不编号
\left[ \hat{J}_{i},\hat{J}_{j} \right] = i\epsilon_{ijk}\hat{J}_{k} \quad i,j,k=x,y,z
\end{equation}
$\therefore \hat{\bm{J}}$满足角动量算符的全部代数性质。\\
(3)
\begin{equation}
\begin{aligned}
\left<n_{1}n_{2}\left| \hat{J}_{z} \right|n_{1}n_{2}\right> &= \left<n_{1}n_{2}\left| \frac{1}{2} \left(\hat{a}^{\dagger}_{1}\hat{a}_{1} - \hat{a}^{\dagger}_{2}\hat{a}_{2}\right) \right|n_{1}n_{2}\right> \\
&= \frac{1}{2} \left<n_{1}n_{2}\left| \left(\hat{n}_{1}-\hat{n}_{2}\right) \right|n_{1}n_{2}\right> \\
&= \frac{1}{2} \left(n_{1}-n_{2}\right)
\end{aligned}
\end{equation}

\begin{equation}
\begin{aligned}
\left<n_{1}n_{2}\left| \hat{J}^{2} \right|n_{1}n_{2}\right> &= \left<n_{1}n_{2}\left|\left( \frac{1}{2}(\hat{J}_{+}\hat{J}_{-}+\hat{J}_{-}\hat{J}_{+})+\hat{J}_{z}^{2} \right)\right|n_{1}n_{2}\right>\\
&= \left<n_{1}n_{2}\left| \frac{1}{2} \left( \hat{a}^{\dagger}_{1}\hat{a}_{2}\hat{a}^{\dagger}_{2}\hat{a}_{1} + \hat{a}^{\dagger}_{2}\hat{a}_{1}\hat{a}^{\dagger}_{1}\hat{a}_{2} \right) \right|n_{1}n_{2}\right> + \left<n_{1}n_{2}\left| \hat{J}_{z}^{2} \right|n_{1}n_{2}\right>
\end{aligned}
\end{equation}
$\because \hat{a}^{\dagger}_{1}\hat{a}_{2}\hat{a}^{\dagger}_{2}\hat{a}_{1} = \hat{a}^{\dagger}_{1}(1+\hat{a}^{\dagger}_{2}\hat{a}_{2})\hat{a}_{1}$\\
$\therefore$
\begin{equation}
\left<n_{1}n_{2}\left| \frac{1}{2} \left( \hat{a}^{\dagger}_{1}\hat{a}_{2}\hat{a}^{\dagger}_{2}\hat{a}_{1} + \hat{a}^{\dagger}_{2}\hat{a}_{1}\hat{a}^{\dagger}_{1}\hat{a}_{2} \right) \right|n_{1}n_{2}\right> = \frac{1}{2}(n_{1}+n_{2}+2n_{1}n_{2})
\end{equation}
$\therefore$
\begin{equation}
\left<n_{1}n_{2}\left| \hat{J}^{2} \right|n_{1}n_{2}\right> = \frac{1}{2}(n_{1}+n_{2}) + \frac{1}{4}(n_{1}+n_{2})^{2}
\end{equation}
~\\
~\\
\textbf{7 \quad 写出单电子相对论性$Dirac$方程中算符$\alpha$和$\beta$所满足的代数关系。$\alpha$和$\beta$的矩阵表示不是唯一的,在$Weyl$表象中,取$\beta=\begin{bmatrix} 0 & I \\ I & 0 \end{bmatrix}$,其中$I$为二阶单位矩阵,试推导出在这一表象中$\alpha$的矩阵表示。}\\
解:
\begin{equation}\nonumber 		% \nonumber使该公式不编号
\beta^{2}=1 \quad \alpha_{i}^{2}=1 \quad \{\alpha_{i},\alpha_{j}\}=\delta_{ij} \quad \{\alpha_{i},\beta\}=0 \; \text{其中}i,j=x,y,z
\end{equation}
设$\alpha_{i}=\begin{bmatrix} A_{i} & B_{i} \\ C_{i} & D_{i} \end{bmatrix}$,代入$\{\alpha_{i},\beta\}=0$,得
\begin{equation}
\begin{bmatrix} A_{i} & B_{i} \\ C_{i} & D_{i} \end{bmatrix}\begin{bmatrix} 0 & I \\ I & 0 \end{bmatrix} + \begin{bmatrix} 0 & I \\ I & 0 \end{bmatrix}\begin{bmatrix} A_{i} & B_{i} \\ C_{i} & D_{i} \end{bmatrix} = \begin{bmatrix} B_{i} & A_{i} \\ D_{i} & C_{i} \end{bmatrix}+\begin{bmatrix} C_{i} & D_{i} \\ A_{i} & B_{i} \end{bmatrix}=0
\end{equation}
$\therefore \alpha_{i}=\begin{bmatrix} A_{i} & B_{i} \\ -B_{i} & -A_{i} \end{bmatrix}$,代入$\{\alpha_{i},\alpha_{j}\}=\delta_{ij}$,得\\
\begin{equation}
\begin{vmatrix} A_{i}^{2}+B_{i}^{2} & \quad \\ \quad & A_{i}^{2}-B_{i}^{2} \end{vmatrix} = 1
\end{equation}
$\therefore B_{i}^{2}=0,A_{i}^{2}=1$,选取$A_{i}=\sigma_{i}$,$\therefore \alpha_{i}=\begin{bmatrix} \sigma_{i} & \quad \\ \quad & -\sigma_{i} \end{bmatrix}\quad i=x,y,z$。\\
~\\
~\\
\textbf{8 \quad 求$Dirac$粒子在深为$V_{0}$,宽为$a$的一维方势阱中的能级。}\\
解:已知$V=\left\{ 
\begin{aligned} 	% 大括号公式的写法
-&V_{0} \quad 0<x<a \\ 
&0 \quad x>a\,\text{或}\,x<0
\end{aligned} \right. $,设该粒子波函数为$\bm\Psi$,由能量本征方程 \\
\begin{equation}
\hat{H}\bm\Psi=E_{n}\bm\Psi=(c \bm\alpha \cdot \bm{p} + mc^{2}\bm\beta-V_{0})\bm\Psi
\end{equation}		
$\therefore$有
\begin{equation}
\begin{vmatrix}
mc^{2} & c\bm\sigma\cdot\bm{p} \\ c\bm\sigma\cdot\bm{p} & -mc^{2}
\end{vmatrix} \bm\Psi = (E_{n}+V_{0})\bm\Psi
\end{equation}
若存在非零解,则有
\begin{equation}
\begin{vmatrix}
mc^{2}-(E_{n}+V_{0}) & c\bm\sigma\cdot\bm{p} \\ c\bm\sigma\cdot\bm{p} & -mc^{2}-(E_{n}+V_{0})
\end{vmatrix} \bm\Psi = 0
\end{equation}
$\therefore$
\begin{equation}
E_{n}=\left\{ 
\begin{aligned} 	% 大括号公式的写法
&-V_{0}+\sqrt{c^{2}p^{2}+m^{2}c^{4}} \quad 0<x<a \\ 
&-V_{0}-\sqrt{c^{2}p^{2}+m^{2}c^{4}} \quad x>a\,\text{或}\,x<0
\end{aligned} \right. 
\end{equation}
~\\
~\\
\textbf{9 \quad 试在$Schr\ddot{o}dinger$图像下计算一维自由粒子的传播子$K(x''t'',x't')$。}\\
解:由传播子满足的方程得
\begin{equation}
(i\hbar\frac{\partial}{\partial t}+\frac{\hbar^{2}}{2m}\frac{\dif^{2}}{\dif x^{2}})K(xt,x't')=i\hbar\delta(x-x')\delta(t-t')
\end{equation}
由$Schr\ddot{o}dinger$方程出发,设$\displaystyle \frac{\partial H}{\partial t}=0$,$Schr\ddot{o}dinger$方程的形式解为
\begin{equation}\nonumber 		% \nonumber使该公式不编号
\left|\psi(t)\right> = e^{-\frac{i}{\hbar}Ht}\left|\psi(0)\right>
\end{equation}
时间演化算符$U(t)=e^{-\frac{i}{\hbar}Ht}$为幺正算符,建立$\left|\psi(t'')\right>$和$\left|\psi(t')\right>$的演化关系:
\begin{equation}\nonumber 		% \nonumber使该公式不编号
\left|\psi(t'')\right> = e^{-\frac{i}{\hbar}H(t''-t')}\left|\psi(t')\right>
\end{equation}
上式左乘$\displaystyle \left<x''\right|$,并插入封闭关系$\displaystyle \int\dif\tau\left|x'\right>\left<x'\right|$,得
\begin{equation}
\left<x''|\psi(t'')\right> = \int\dif x'^{3}\left<x''\right| e^{-\frac{i}{\hbar}H(t''-t')} \left|x'\right>\left<x'|\psi(t'')\right>
\end{equation}
$\therefore$
\begin{equation}
\begin{aligned}
\psi(x'',t'') &= \int\dif x'^{3}\left<x''\right| e^{-\frac{i}{\hbar}H(t''-t')} \left|x'\right> \psi(x',t') \\
\psi(x'',t'') &= \int\dif x'^{3}K(x''t'',x't')\psi(x',t')
\end{aligned}
\end{equation}
其中$\displaystyle  K(x''t'',x't') =\left<x''\left|e^{-\frac{i}{\hbar}H(t''-t')}\right|x'\right> $称为传播子。



%%%%%%%%%%%%%%%%%%%%%%%%%%%%%%%%%%%%%%%%%%%%%%%%%%
%%%%%%%%%%%%%%%%%%%% 2005年期末题 %%%%%%%%%%%%%%%%%%%%%%%
%%%%%%%%%%%%%%%%%%%%%%%%%%%%%%%%%%%%%%%%%%%%%%%%%%
\newpage
\subsection{2005年习题}
~\\
~\\
\textbf{1 \quad 试证复数共轭算符$\hat{K}$为反幺正算符,并求解其本征问题。}\\
解:先证明$\hat{K}$具有反线性:对$\forall\psi \in R$\\
\begin{equation}
\hat{K}\psi = \psi^{*} \qquad \hat{K}\sum_{i}c_{i}\psi_{i} = \sum_{i}c^{*}_{i}\hat{K}\psi_{i}
\end{equation}
$\because$
\begin{equation}
\hat{K}^{2}\psi = \psi \Rightarrow \hat{K}^{2}=1 \Rightarrow \hat{K}^{-1}=\hat{K}
\end{equation}
又$\because$
\begin{equation}
\begin{aligned}
\int\psi^{*}\hat{K}\phi\dif\tau &= \int\psi^{*}\phi^{*}\dif\tau = \int(\hat{K}\psi)\phi^{*}\dif\tau \\
&= \int\phi^{*}\psi^{*}\dif\tau = \int(\hat{K}\phi^{*})\psi\dif\tau
\end{aligned}
\end{equation}
$\therefore \hat{K}^{\dagger}=\hat{K} \Rightarrow \hat{K}^{-1}=\hat{K}^{\dagger}$,$\hat{K}$为反幺正算符。设$\hat{K}$本征矢为$\left|k\right>$,对应本征值为$k$
\begin{equation}
\hat{K}\left|k\right> = k\left|k\right>
\end{equation}
$\therefore$
\begin{equation}
\hat{K}^{2}\left|k\right> = \left|k\right> = \hat{K}k\left|k\right> = k^{*}\hat{K}\left|k\right> = k^{*}k\left|k\right>
\end{equation}
$\therefore \left|k\right> = k^{*}k\left|k\right>$,则有$k^{*}k=1$,
$\therefore k=e^{\pm im\alpha}$。\\
~\\
~\\
\textbf{2 \quad 设$\hat{T}_{l_{1}m_{1}}(\tau_{1})$和$\hat{T}_{l_{2}m_{2}}(\tau_{2})$分别为阶和阶不可约张量算符,求证由下式定义的算符
为阶不可约张量算符:
\begin{equation}\nonumber 		% \nonumber使该公式不编号
\hat{T}_{LM}(\tau_{1}\tau_{2}) = \sum_{m_{1}m_{2}}C_{l_{1}m_{1}l_{2}m_{2}}^{LM}\hat{T}_{l_{1}m_{1}}(\tau_{1})\hat{T}_{l_{2}m_{2}}(\tau_{2})
\end{equation}}\\
~\\
解:在无穷小转动变换$U(\bm{n},\dif\theta)$下\\
\begin{align*}
U\hat{T}_{LM}(\tau_{1}\tau_{2})U^{-1} &= \sum_{m_{1}m_{2}}C_{l_{1}m_{1}l_{2}m_{2}}^{LM} U\hat{T}_{l_{1}m_{1}}(\tau_{1})U^{-1}U\hat{T}_{l_{2}m_{2}}(\tau_{2})U^{-1} \\
&= \sum_{m_{1}m_{2}}C_{l_{1}m_{1}l_{2}m_{2}}^{LM} \sum_{\mu_{1}}D_{\mu_{1}m_{1}}^{l_{1}} \hat{T}_{l_{1}m_{1}}(\tau_{1}) \sum_{\mu_{2}}D_{\mu_{2}m_{2}}^{l_{2}} \hat{T}_{l_{2}m_{2}}(\tau_{2}) \\
&= \sum_{\mu_{1}\mu_{2}m_{1}m_{2}}C_{l_{1}m_{1}l_{2}m_{2}}^{LM} \sum_{L'\mu_{1}M'}C_{l_{1}\mu_{1}l_{2}\mu_{2}}^{L'\mu}C_{l_{1}m_{1}l_{2}m_{2}}^{L'M'} D_{\mu M'}^{L'} \hat{T}_{l_{1}m_{1}}(\tau_{1})\hat{T}_{l_{2}m_{2}}(\tau_{2}) \\
&= \sum_{\mu_{1}\mu_{2}} \sum_{L'\mu_{1}M'}C_{l_{1}\mu_{1}l_{2}\mu_{2}}^{L'\mu} \delta_{LL'} \delta_{MM'} D_{\mu M'}^{L'} \hat{T}_{l_{1}m_{1}}(\tau_{1})\hat{T}_{l_{2}m_{2}}(\tau_{2}) \\
&= \sum_{\mu} \sum_{\mu_{1}\mu_{2}} C_{l_{1}\mu_{1}l_{2}\mu_{2}}^{L\mu} D_{\mu M}^{L} \hat{T}_{l_{1}m_{1}}(\tau_{1})\hat{T}_{l_{2}m_{2}}(\tau_{2}) \\
&= \sum_{\mu} D_{\mu M}^{L} \hat{T}_{LM}(\tau_{1}\tau_{2})
\end{align*}
$\therefore \hat{T}_{LM}(\tau_{1}\tau_{2})$是L阶不可约张量。 \\
~\\
~\\


\textbf{3 \quad 对一个由两个自旋$\displaystyle \frac{1}{2}$粒子组成的体系,定义算符
\begin{equation}\nonumber 		% \nonumber使该公式不编号
S_{12}=\frac{(\bm\sigma_{1}\cdot\bm{r})(\bm\sigma_{2}\cdot\bm{r})}{r^{2}}-\frac{1}{3}\bm\sigma_{1}\cdot\bm\sigma_{2} \quad (\bm{r}=\bm{r}_{1}-\bm{r}_{2})
\end{equation}
证明:\\
(1) $S_{12}$与算符$\bm{S}^{2}$对易,这里$\displaystyle \bm{S}=\frac{\hbar}{2}\left( \bm\sigma_{1}+\bm\sigma_{2} \right)$。\\
(2) 对$S=0$,有$S_{12}^{2}=2S_{12}$,而$S=1$时,$\displaystyle S_{12}^{2}=\frac{8}{9}-\frac{2}{3}S_{12}$。\\
(3) $S_{12}\chi_{00}=0$,这里$\displaystyle \chi_{00} = \frac{1}{\sqrt{2}}\left( \left|+\right>_{1}\left|-\right>_{2} - \left|-\right>_{1}\left|+\right>_{2} \right)$,其中$\left|\pm\right>$为$\sigma_{z}$分别对应于本征值$\pm1$的本征态。} \\
~\\
解:\\
(1) $\because \displaystyle \bm{S}^{2} = \frac{\hbar^{2}}{4}\left(\bm\sigma_{1}+\bm\sigma_{2}\right)\left(\bm\sigma_{1}+\bm\sigma_{2}\right) = \frac{\hbar^{2}}{4} \left(\bm\sigma_{1}^{2}+\bm\sigma_{2}^{2}+2\right)\bm\sigma_{1}\cdot\bm\sigma_{2} = \frac{\hbar^{2}}{2}\left(3I+\bm\sigma_{1}\cdot\bm\sigma_{2}\right)$\\
$\therefore$ 
\begin{equation}
\left[ S_{12}, \bm{S}^{2}\right] = \left[ S_{12}, \frac{\hbar^{2}}{2}\left(3I+\bm\sigma_{1}\cdot\bm\sigma_{2}\right) \right] = \frac{\hbar^{2}}{2}\left[ S_{12}, \bm\sigma_{1}\cdot\bm\sigma_{2}\right]
\end{equation}
其中
\begin{equation}
\left[ S_{12}, \bm\sigma_{1}\cdot\bm\sigma_{2}\right] = \frac{1}{r^{2}}2I\left(r^{2}I^{(1)} - r^{2}I^{(2)}\right) = \frac{1}{r^{2}}2I\left(r^{2}- r^{2}\right)=0
\end{equation}
$\therefore$ $S_{12}$与算符$\bm{S}^{2}$对易。\\
(2) \begin{align*}
S_{12}^{2} &=\frac{(\bm\sigma_{1}\cdot\bm{r})(\bm\sigma_{2}\cdot\bm{r})(\bm\sigma_{1}\cdot\bm{r})(\bm\sigma_{2}\cdot\bm{r})}{r^{4}} + \frac{1}{9}\left(\bm\sigma_{1}\cdot\bm\sigma_{2}\right)^{2} - \frac{2}{3}\frac{(\bm\sigma_{1}\cdot\bm{r})(\bm\sigma_{2}\cdot\bm{r})}{r^{2}}\left(\bm\sigma_{1}\cdot\bm\sigma_{2}\right) \\
&= I + \frac{1}{9}\left(\bm\sigma_{1}\cdot\bm\sigma_{2}\right)^{2} -\frac{2}{3}\frac{(\bm\sigma_{1}\cdot\bm{r})(\bm\sigma_{2}\cdot\bm{r})}{r^{2}}
\end{align*}
由$\displaystyle \bm{S}^{2} = \frac{\hbar^{2}}{2}\left(3I+\bm\sigma_{1}\cdot\bm\sigma_{2}\right)$得 \\
\begin{equation}
S=0,\; \bm{S}^{2} = \frac{\hbar^{2}}{2}\left(3I+\bm\sigma_{1}\cdot\bm\sigma_{2}\right) = 0 \Rightarrow \bm\sigma_{1}\cdot\bm\sigma_{2}=-3I
\end{equation}
$\therefore$ 
\begin{equation}\nonumber 		% \nonumber使该公式不编号
S_{12}^{2} = 2I + 2\frac{(\bm\sigma_{1}\cdot\bm{r})(\bm\sigma_{2}\cdot\bm{r})}{r^{2}} = 2\frac{(\bm\sigma_{1}\cdot\bm{r})(\bm\sigma_{2}\cdot\bm{r})}{r^{2}}-\frac{2}{3}\bm\sigma_{1}\cdot\bm\sigma_{2} = 2S_{12}
\end{equation}
以及
\begin{equation}
S=1,\; \bm{S}^{2} = \frac{\hbar^{2}}{2}\left(3I+\bm\sigma_{1}\cdot\bm\sigma_{2}\right) = 2\hbar^{2}I \Rightarrow \bm\sigma_{1}\cdot\bm\sigma_{2}=I
\end{equation}
$\therefore$ 
\begin{equation}\nonumber 		% \nonumber使该公式不编号
S_{12}^{2} = \frac{10}{9}I - \frac{2}{3}\frac{(\bm\sigma_{1}\cdot\bm{r})(\bm\sigma_{2}\cdot\bm{r})}{r^{2}} = \frac{8}{9}I + \frac{2}{3}\cdot\frac{1}{3}\bm\sigma_{1}\cdot\bm\sigma_{2} - \frac{2}{3}\frac{(\bm\sigma_{1}\cdot\bm{r})(\bm\sigma_{2}\cdot\bm{r})}{r^{2}} = \frac{8}{9}-\frac{2}{3}S_{12}
\end{equation}
(3) $S=0 \Rightarrow \bm\sigma_{1} = -\bm\sigma_{2}$,此时$S_{12}=0$,$\therefore S_{12}\chi_{00}=0$。\\
~\\
~\\
\textbf{4 \quad 令$\hat{n}=\hat{a}_{\alpha}^{\dag}\hat{a}_{\alpha}$,其中$\alpha$为量子态标记,证明无论对玻色子还是费米子,均有
\begin{equation}\nonumber 		% \nonumber使该公式不编号
\left[\hat{n},\hat{a}_{\alpha}^{\dag}\right]=\hat{a}_{\alpha}^{\dag} \quad \left[\hat{n},\hat{a}_{\alpha}\right]=-\hat{a}_{\alpha}
\end{equation}}
解:\\
$\bullet$ 对玻色子有$\left[\hat{a}_{\alpha},\hat{a}_{\alpha}^{\dag}\right]=1 \Rightarrow \hat{a}_{\alpha}\hat{a}_{\alpha}^{\dag}-\hat{a}_{\alpha}^{\dag}\hat{a}_{\alpha}=1$\\
$\therefore$
\begin{align*}
\left[\hat{n},\hat{a}_{\alpha}^{\dag}\right] &= \hat{n}\hat{a}_{\alpha}^{\dag} - \hat{a}_{\alpha}^{\dag}\hat{n} \\
&= \hat{a}_{\alpha}^{\dag}\hat{a}_{\alpha}\hat{a}_{\alpha}^{\dag} - \hat{a}_{\alpha}^{\dag}\hat{a}_{\alpha}^{\dag}\hat{a}_{\alpha} = \hat{a}_{\alpha}^{\dag}\left(1+\hat{a}_{\alpha}^{\dag}\hat{a}_{\alpha}\right) - \hat{a}_{\alpha}^{\dag}\hat{a}_{\alpha}^{\dag}\hat{a}_{\alpha}\\
&= \hat{a}_{\alpha}^{\dag}
\end{align*}
以及
\begin{align*}
\left[\hat{n},\hat{a}_{\alpha}\right] &= \hat{n}\hat{a}_{\alpha} - \hat{a}_{\alpha}\hat{n} \\
&= \hat{a}_{\alpha}^{\dag}\hat{a}_{\alpha}\hat{a}_{\alpha} - \hat{a}_{\alpha}\hat{a}_{\alpha}^{\dag}\hat{a}_{\alpha} = \left(1+\hat{a}_{\alpha}^{\dag}\hat{a}_{\alpha}\right)\hat{a}_{\alpha} - \hat{a}_{\alpha}\hat{a}_{\alpha}^{\dag}\hat{a}_{\alpha}\\
&= -\hat{a}_{\alpha}
\end{align*}

$\bullet$ 对费米子有$\left\{\hat{a}_{\alpha},\hat{a}_{\alpha}^{\dag}\right\}=1 \Rightarrow \hat{a}_{\alpha}\hat{a}_{\alpha}^{\dag}+\hat{a}_{\alpha}^{\dag}\hat{a}_{\alpha}=1$,以及$\left\{\hat{a}_{\alpha},\hat{a}_{\alpha}\right\}=0,\left\{\hat{a}_{\alpha}^{\dag},\hat{a}_{\alpha}^{\dag}\right\}=0$\\
$\therefore \hat{a}_{\alpha}\hat{a}_{\alpha} = \hat{a}_{\alpha}^{\dag}\hat{a}_{\alpha}^{\dag} = 0$,进而有
\begin{align*}
\left[\hat{n},\hat{a}_{\alpha}^{\dag}\right] &= \hat{n}\hat{a}_{\alpha}^{\dag} - \hat{a}_{\alpha}^{\dag}\hat{n}\\
&= \hat{a}_{\alpha}^{\dag}\hat{a}_{\alpha}\hat{a}_{\alpha}^{\dag} - \hat{a}_{\alpha}^{\dag}\hat{a}_{\alpha}^{\dag}\hat{a}_{\alpha} = \hat{a}_{\alpha}^{\dag}\left(1+\hat{a}_{\alpha}^{\dag}\hat{a}_{\alpha}\right) - \hat{a}_{\alpha}^{\dag}\hat{a}_{\alpha}^{\dag}\hat{a}_{\alpha} \\
&= \hat{a}_{\alpha}^{\dag} - 2\hat{a}_{\alpha}^{\dag}\hat{a}_{\alpha}^{\dag}\hat{a}_{\alpha} = \hat{a}_{\alpha}^{\dag}
\end{align*}
同理有\\
\begin{align*}
\left[\hat{n},\hat{a}_{\alpha}\right] &= \hat{n}\hat{a}_{\alpha} - \hat{a}_{\alpha}\hat{n}\\
&= \hat{a}_{\alpha}^{\dag}\hat{a}_{\alpha}\hat{a}_{\alpha} - \hat{a}_{\alpha}\hat{a}_{\alpha}^{\dag}\hat{a}_{\alpha} = \hat{a}_{\alpha}^{\dag}\hat{a}_{\alpha}\hat{a}_{\alpha} - \left(1-\hat{a}_{\alpha}^{\dag}\hat{a}_{\alpha}\right)\hat{a}_{\alpha}  \\
&= 2\hat{a}_{\alpha}^{\dag}\hat{a}_{\alpha}\hat{a}_{\alpha} - \hat{a}_{\alpha} = -\hat{a}_{\alpha}
\end{align*}
~\\
~\\


\textbf{5 \quad 中微子是自旋为$\displaystyle \frac{1}{2}$,静止质量极小的基本粒子。将中微子静止质量取为$0$,试建立其相对论性波动方程,并讨论其守恒量。}\\
~\\
解:中微子不带电,静止质量$m \rightarrow 0$,$\displaystyle s=\frac{1}{2}$,由$E^{2}=c^{2}p^{2}$,对中微子而言
\begin{equation}
i\hbar\frac{\partial\bm\Psi}{\partial t} = c\bm{\alpha\cdot p \Psi}
\end{equation}
由$H\bm\Psi=E\bm\Psi$得$H^{2}\bm\Psi=E^{2}\bm\Psi=c^{2}p^{2}\bm\Psi$,进而有$(\bm{\alpha\cdot p})^{2}=p^{2}$,由$\bm\alpha$的代数关系可将$\bm\alpha$取为$\bm\sigma$\\
$\therefore$ 中微子的相对论性波动方程为
\begin{equation}
i\hbar\frac{\partial\bm\Psi}{\partial t} = c\bm{\sigma\cdot p \Psi}
\end{equation}
$\bullet$ 讨论中微子的守恒量:显然能量守恒,$H=c\bm{\sigma\cdot p}$是一个守恒量。\\
$\because$
\begin{equation}
\left[H,\bm{p}\right]=0 \quad \left[H,\frac{\bm{\sigma\cdot p}}{|\bm{p}|}\right]=0
\end{equation}
$\therefore$ 中微子的动量算符$\bm{p}$和螺旋量算符$\displaystyle \frac{\bm{\sigma\cdot p}}{|\bm{p}|}$都是守恒量。\\
又$\because$
\begin{equation}
\left[H,\bm{L}\right] \ne 0 \quad \left[H,P\right] \ne 0
\end{equation}
$\therefore$ 中微子的轨道角动量算符$\bm{L}$和宇称算符$P$都不是守恒量。\\
对于总角动量$\bm{J}$而言,虽然具有$\left[H,\bm{J}\right]=0$,但$\left[\bm{J},\bm{p}\right] \ne 0$,不能纳入力学量完全集,所以总角动量$\bm{J}$不是守恒量。\\
综上所述,中微子的能量算符$H$,动量算符$\bm{p}$和螺旋量算符$\displaystyle \frac{\bm{\sigma\cdot p}}{|\bm{p}|}$是一组守恒量。\\
~\\
~\\
\textbf{6 \quad 试采用$Feynman$的多边折线道方案计算一维自由粒子的传播子$K(x''t'',x't')$,计算中,取$\displaystyle C_{N} = \left(\frac{2\pi\hbar i\epsilon}{m}\right)^{-N/2}$。}\\
解:由多边折线道方法$\displaystyle K_{N}(x''t'',x't') = C_{N}\int \cdot\cdot\cdot \int exp\left\{\frac{im}{2\hbar\epsilon}\sum_{j}^{N}\left(x_{j}-x_{j-1}\right)^{2}\right\} \dif x_{1} \cdot\cdot\cdot \dif x_{N}$ \\
$\therefore$
\begin{align*}
K_{N}(x''t'',x't') &= C_{N}\int \cdot\cdot\cdot \int exp\left\{\frac{im}{2\hbar\epsilon}\sum_{j}^{N}\left(x_{j}-x_{j-1}\right)^{2}\right\} \dif x_{1} \cdot\cdot\cdot \dif x_{N} \\
&= \left(\frac{2\pi\hbar i\epsilon}{m}\right)^{-N/2}\int \cdot\cdot\cdot \int exp\left\{\frac{im}{2\hbar\epsilon}\sum_{j}^{N}\left(x_{j}-x_{j-1}\right)^{2}\right\} \dif x_{1} \cdot\cdot\cdot \dif x_{N} \\
&= \left(\frac{m}{2\pi\hbar i(t''-t')}\right)^{\frac{1}{2}} e^{\frac{im}{2(t''-t')\hbar}(x''-x')^{2}}
\end{align*}


%%%%%%%%%%%%%%%%%%%%%%%%%%%%%%%%%%%%%%%%%%%%%%%%%%
%%%%%%%%%%%%%%%%%%%% 2006年期末题 %%%%%%%%%%%%%%%%%%%%%%%
%%%%%%%%%%%%%%%%%%%%%%%%%%%%%%%%%%%%%%%%%%%%%%%%%%
\newpage
\subsection{2006年习题}
~\\
~\\
\textbf{1 \quad 称按规律
\begin{equation}\nonumber 		% \nonumber使该公式不编号
U(\bm{n},\dif\theta) \hat{T}_{lm}(\tau) U^{-1}(\bm{n},\dif\theta) = \hat{T}_{lm}(\tau') = \sum_{m'}D^{l}_{m'm}(\alpha\beta\gamma)\hat{T}_{lm'}(\tau)
\end{equation}
变换的$2l+1$个算符$\hat{T}_{lm}(\tau) \; (m=l,l-1,\cdot\cdot\cdot,-l)$为$l$阶不可约张量算符,其中为$U(\bm{n},\dif\theta)$转动算符,$D^{l}_{m'm}(\alpha\beta\gamma)$为$D$函数。证明$\hat{T}_{lm}(\tau)$满足$\displaystyle \left[ \hat{L}_{z},\hat{T}_{lm} \right]=m\hbar\hat{T}_{lm}$,这里$\hat{L}_{z}$为轨道角动量算符。
}\\
~\\
证明:对无穷小转动$U(\bm{n},\dif\theta)$有$\displaystyle U(\bm{n},\dif\theta) = 1-\frac{i}{\hbar}\dif\theta\hat{L}_{z}$,$\displaystyle U^{-1}(\bm{n},\dif\theta) = 1+\frac{i}{\hbar}\dif\theta\hat{L}_{z}$\\
$\therefore$
\begin{equation}
\left(1-\frac{i}{\hbar}\dif\theta\hat{L}_{z}\right) \hat{T}_{lm} \left(1+\frac{i}{\hbar}\dif\theta\hat{L}_{z}\right) = \sum_{m'}\left<lm'\left|1-\frac{i}{\hbar}\dif\theta\hat{L}_{z}\right|lm\right>\hat{T}_{lm'}
\end{equation}
略去含有$\dif\theta^{2}$得高阶项,得
\begin{equation}
\hat{T}_{lm} -\frac{i}{\hbar}\dif\theta\hat{L}_{z}\hat{T}_{lm} +\frac{i}{\hbar}\dif\theta\hat{T}_{lm}\hat{L}_{z}  = \sum_{m'}\left[ \delta_{m'm} -im\dif\theta \delta_{m'm} \right]\hat{T}_{lm'}
\end{equation}
$\therefore$
\begin{equation}
\hat{T}_{lm} - \frac{i}{\hbar}\left[ \hat{L}_{z},\hat{T}_{lm} \right] = \hat{T}_{lm} - im\dif\theta\hat{T}_{lm} 
\end{equation}
$\therefore \displaystyle \left[ \hat{L}_{z},\hat{T}_{lm} \right]=m\hbar\hat{T}_{lm}$,得证。\\
~\\
~\\


\textbf{2 \quad 设费米子体系的每个单粒子能级都是二重简并的,属于单粒子能级$\epsilon_{\mu}$的两个简并态用$\mu,\bar\mu$标记,相应的产生、消灭算符记为$a_{\mu}^{\dag},a_{\mu},a_{\bar\mu}^{\dag},a_{\mu}$。定义
\begin{equation}\nonumber 		% \nonumber使该公式不编号
S_{\mu}^{\dag}=a_{\mu}^{\dag}a_{\bar\mu}^{\dag} \quad S_{\mu}=\left(S_{\mu}^{\dag}\right)^{\dag}=a_{\bar\mu}a_{\mu} \quad \hat{n}_{\mu}=a_{\mu}^{\dag}a_{\mu}+a_{\bar\mu}^{\dag}a_{\bar\mu}
\end{equation}
说明算符$S_{\mu}^{\dag}$,$S_{\mu}$和$\hat{n}_{\mu}$的意义。证明:$\displaystyle \left[S_{\mu},S_{\nu}^{\dag}\right]=\left(1-\hat{n}_{\mu}\right)\delta_{\mu\nu}$,$\displaystyle \left[\hat{n}_{\mu},S_{\nu}^{\dag}\right]=2S_{\mu}^{\dag}\delta_{\mu\nu}$。}\\
~\\
证明:$S_{\mu}^{\dag}(S_{\mu})$算符可在能级$\epsilon_{\mu}$上产生(消灭)一对粒子,$\hat{n}_{\mu}$是能级$\epsilon_{\mu}$上的粒子数算符。对费米子体系有如下关系:\\
\begin{equation}
\left\{a_{\mu}^{\dag},a_{\nu}^{\dag}\right\} = \left\{a_{\mu},a_{\nu}\right\} = 0 \quad \left\{a_{\mu},a_{\nu}^{\dag}\right\} = \delta_{\mu\nu}
\end{equation}
$\therefore$
\begin{align*}
\left[S_{\mu},S_{\nu}^{\dag}\right] &= \left[a_{\bar\mu}a_{\mu} , a_{\nu}^{\dag}a_{\bar\nu}^{\dag}\right] \\
&= a_{\bar\mu}\left[a_{\mu} , a_{\nu}^{\dag}\right]a_{\bar\nu}^{\dag} + \left[a_{\bar\mu} , a_{\nu}^{\dag}\right]a_{\bar\nu}^{\dag}a_{\mu} + a_{\bar\mu}a_{\nu}^{\dag}\left[a_{\mu} , a_{\bar\nu}^{\dag}\right] + a_{\nu}^{\dag}\left[a_{\bar\mu} , a_{\bar\nu}^{\dag}\right]a_{\mu}\\
&= a_{\bar\mu}a_{\mu}a_{\nu}^{\dag}a_{\bar\nu}^{\dag} - a_{\nu}^{\dag}a_{\bar\nu}^{\dag}a_{\bar\mu}a_{\mu} \\
&= a_{\bar\mu}a_{\bar\nu}^{\dag}\delta_{\mu\nu} - a_{\nu}^{\dag}a_{\mu}\delta_{\mu\nu}
\end{align*}
当$\mu \ne \nu$时,显然有$\displaystyle \left[S_{\mu},S_{\nu}^{\dag}\right] = 0$;\\
当$\mu = \nu$时,
\begin{align*}
\left[S_{\mu},S_{\nu}^{\dag}\right] &= \left(a_{\bar\mu}a_{\bar\mu}^{\dag} - a_{\mu}^{\dag}a_{\mu}\right)\delta_{\mu\nu} = \left[1-\left(a_{\bar\mu}a_{\bar\mu}^{\dag} + a_{\mu}^{\dag}a_{\mu}\right)\right]\delta_{\mu\nu} \\
&= \left(1 - \hat{n}_{\mu}\right)\delta_{\mu\nu}
\end{align*}
同理有
\begin{align*}
\left[\hat{n}_{\mu},S_{\nu}^{\dag}\right] &= \left[a_{\mu}^{\dag}a_{\mu} , a_{\nu}^{\dag}a_{\bar\nu}^{\dag}\right] + \left[a_{\bar\mu}^{\dag}a_{\bar\mu} , a_{\nu}^{\dag}a_{\bar\nu}^{\dag}\right] \\
&= a_{\mu}^{\dag}\left[a_{\mu} , a_{\nu}^{\dag}a_{\bar\nu}^{\dag}\right] + \left[a_{\mu}^{\dag} , a_{\nu}^{\dag}a_{\bar\nu}^{\dag}\right]a_{\mu} + a_{\bar\mu}^{\dag}\left[a_{\bar\mu} , a_{\nu}^{\dag}a_{\bar\nu}^{\dag}\right] + \left[a_{\bar\mu}^{\dag} , a_{\nu}^{\dag}a_{\bar\nu}^{\dag}\right]a_{\bar\mu} \\
&= \left(a_{\mu}^{\dag}a_{\bar\mu}^{\dag} + a_{\mu}^{\dag}a_{\bar\mu}^{\dag}\right)\delta_{\mu\nu}
\end{align*}
当$\mu \ne \nu$时,显然有$\displaystyle \left[\hat{n}_{\mu},S_{\nu}^{\dag}\right] = 0$;\\
当$\mu = \nu$时,$\displaystyle \left[\hat{n}_{\mu},S_{\nu}^{\dag}\right] = 2S_{\mu}^{\dag}\delta_{\mu\nu}$。\\
综上所述,$\displaystyle \left[S_{\mu},S_{\nu}^{\dag}\right]=\left(1-\hat{n}_{\mu}\right)\delta_{\mu\nu}$,$\displaystyle \left[\hat{n}_{\mu},S_{\nu}^{\dag}\right]=2S_{\mu}^{\dag}\delta_{\mu\nu}$。\\
~\\
~\\


\textbf{3 \quad 直接写出$Klein-Gordon$方程,由此导出相应的几率守恒方程,说明该方程存在的负几率困难,以及$Pauli$对此给出的合理解释,并回答$K-G$方程描述什么样的粒子。}\\
~\\
解:自由粒子的$Klein-Gordon$方程为
\begin{equation}
-\hbar^{2}\frac{\partial^{2}}{\partial t^{2}} \Psi(\bm{r},t) = \left[ -\hbar^{2}c^{2}\nabla^{2}+m^{2}c^{4} \right] \Psi(\bm{r},t)
\end{equation}
可写为协变形式
\begin{equation}
\left(\Box-\kappa^{2}\right)\Psi(x) = 0
\end{equation}
其中$\displaystyle \kappa=\frac{mc}{\hbar}$,$\displaystyle \Box=\frac{\partial^{2}}{\partial x_{\mu}\partial x_{\mu}},\;\mu=1,2,3,4$,$x_{\mu} \equiv \left( \bm{r},ict \right)$。
由自由粒子的$Klein-Gordon$方程可得
\begin{equation}
-\hbar^{2}\frac{\partial^{2}}{\partial t^{2}} \Psi^{*}(\bm{r},t) = \left[ -\hbar^{2}c^{2}\nabla^{2}+m^{2}c^{4} \right] \Psi(\bm{r},t)^{*}
\end{equation}
$\therefore$ 
\begin{equation}
-\hbar^{2}\left(\Psi^{*}\frac{\partial^{2}\Psi}{\partial t^{2}} - \Psi\frac{\partial^{2}\Psi^{*}}{\partial t^{2}} \right) = -\hbar^{2}c^{2}\left( \Psi^{*}\nabla\Psi - \Psi\nabla\Psi^{*} \right)
\end{equation}
即
\begin{equation}
\frac{1}{c^{2}}\frac{\partial}{\partial t}\left(\Psi^{*}\frac{\partial\Psi}{\partial t} - \Psi\frac{\partial\Psi^{*}}{\partial t} \right) = \nabla\cdot\left( \Psi^{*}\nabla^{2}\Psi - \Psi\nabla^{2}\Psi^{*} \right)
\end{equation}
$\therefore \displaystyle \frac{\partial\rho}{\partial t}+\nabla\cdot\bm{J} = 0$,其中$\displaystyle \rho = \frac{i\hbar}{2mc^{2}}\left(\Psi^{*}\frac{\partial^{2}\Psi}{\partial t^{2}} - \Psi\frac{\partial^{2}\Psi^{*}}{\partial t^{2}} \right)$。 但$\rho$不一定是正定的,不能解释为几率密度,这便是负几率困难的由来。\\
$Pauli$和$Weisskopf$认为应该把$Klein-Gordon$方程视为一个场方程,并把$q\rho$和$q\bm{J}$解释为电荷密度和电流密度,其中$q$为粒子电荷,可正可负。当今普遍认为$Klein-Gordon$方程视为一个标量场方程,场量子自旋为$0$,可以用来描述自旋为$0$的粒子,比如$\pi$介子。
~\\
~\\



\textbf{4 \quad 设哈密顿量不显含时间,由含时薛定谔方程出发,证明可将体系状态波函数表为
\begin{equation}\nonumber 		% \nonumber使该公式不编号
\Psi(\bm{r},t) = \int\dif\tau'K(\bm{r}t,\bm{r'}t')\Psi(\bm{r'},t')
\end{equation}
这里$\displaystyle K(\bm{r}t,\bm{r'}t') = \left<\bm{r}\left|exp\left[-\frac{i}{\hbar}H(t-t')\right]\right|\bm{r'}\right>$,称为传播子。设能量本征方程为$H\left|n\right>=E_{n}\left|n\right>$
,试在能量表象下写出$K(\bm{r}t,\bm{r'}t')$的表达式。}\\
~\\
解:
\begin{align*}
K(\bm{r''}t,\bm{r'}t') &= \sum_{n}\left<\bm{r''}\left|exp\left[-\frac{i}{\hbar}H(t''-t')\right]\right|n\right>\left<n\bigg|\bm{r'}\right> \\
&= \sum_{n} exp\left[-\frac{i}{\hbar}E_{n}(t''-t')\right]\phi_{n}(r'')\phi^{*}_{n}(r')\\
&= \sum_{n} \phi_{n}(r'',t'')\phi^{*}_{n}(r',t')\\
\end{align*}
~\\
~\\

%%%%%%%%%%%%%%%%%%%%%%%%%%%%%%%%%%%%%%%%%%%%%%%%%%
%%%%%%%%%%%%%%%%%%%% 2007年期末题 %%%%%%%%%%%%%%%%%%%%%%%
%%%%%%%%%%%%%%%%%%%%%%%%%%%%%%%%%%%%%%%%%%%%%%%%%%
\newpage
\subsection{2007年习题}
~\\
~\\
\textbf{1 \quad 自旋不为零的粒子的时间反演算符可表为$T=UK$,这里$U=e^{-\frac{i}{\hbar}\pi S_{y}}$,其中$S_{y}$为粒子自旋算符的$y$轴分量,$K$为复共轭算符。\\
(1)算符$U$表示一个什么样的操作?证明$T$为反幺正算符。(2)证明对易子$[U,K]=0$,进而求出$T^{2}$的本征值。(3)对自旋$\displaystyle \frac{1}{2}$粒子,证明$U=-i\sigma_{y}$,进而讨论$S_{z}$的本征态的时间反演态。}\\
~\\
解:\\
(1) \quad 
$U$代表绕$y$轴旋转$\pi$,是一个幺正算符。\\
$\because$
\begin{equation}\nonumber 		% \nonumber使该公式不编号
(UK)(UK)^{\dagger} = UK \cdot KU^{\dagger} = UK^{2}U^{\dagger} = UU^{\dagger} = 1,\text{其中}K^{2}=1
\end{equation}
$\therefore$ 可得$(UK)^{-1} = KU^{\dagger}$。\\
又$\because$
\begin{equation}\nonumber 		% \nonumber使该公式不编号
\begin{aligned}
\int \psi^{*}(UK)\phi\dif\tau &= \int \psi^{*}U\phi^{*}\dif\tau = \int (U^{\dagger}\psi)^{*}\phi^{*}\dif\tau \\
&= \int \psi^{*}K(U^{\dagger}\phi)\dif\tau
\end{aligned}
\end{equation}
$\therefore$ 可得$(UK)^{\dagger} = KU^{\dagger}$,即有$T^{\dagger} = T^{-1}$,$T$为反幺正算符。\\
(2) \quad $T$为反幺正算符,则有\\
\begin{equation}
\left[T,T^{\dagger}\right] = T^{\dagger}T - T^{\dagger}T = I - I = 0
\end{equation}
$\therefore$
\begin{equation}
\begin{aligned}
\left[T,T^{\dagger}\right] &= \left[UK,KU^{\dagger}\right] = U\left[K,KU^{\dagger}\right] + \left[U,KU^{\dagger}\right]K\\
&= U\left[K,K\right]U^{\dagger} + UK\left[K,U^{\dagger}\right] + K\left[U,U^{\dagger}\right]K + \left[U,K\right]U^{\dagger}K \\
&= UK\left[K,U^{\dagger}\right] + \left[U,K\right]U^{\dagger}K \\
&= UK\left[U,K\right]^{\dagger} + \left[U,K\right]U^{\dagger}K = 0
\end{aligned}
\end{equation}
$\therefore \left[U,K\right] = \left[U,K\right]^{\dagger} = 0$。\\
$\because$
\begin{equation}
T^{2} = e^{-\frac{i}{\hbar}\pi S_{y}}K e^{-\frac{i}{\hbar}\pi S_{y}}K =e^{-\frac{i}{\hbar}2\pi S_{y}}
\end{equation}
对$N$个全同粒子体系,有
\begin{equation}
T^{2} = \left\{
\begin{aligned}
&1 \quad \text{玻色子或偶数个费米子}\\
-&1  \quad \text{奇数个费米子}
\end{aligned}
\right.	% 注意这个点不要拉下\right.	
\end{equation}
$\therefore T^{2}$对应的本征值为$\pm1$。\\
(3) \quad 对自旋$\displaystyle \frac{1}{2}$粒子,$\displaystyle U=e^{-\frac{i\pi}{2} \sigma_{y}}$,由谱分解定理得\\
\begin{equation}
\sigma_{y} = 1\cdot\hat{P}_{1} + (-1)\cdot\hat{P}_{-1} = \left|+\left>_{yy}\right<+\right| - \left|-\left>_{yy}\right<-\right|
\end{equation}

\begin{equation}
\begin{aligned}
e^{-\frac{i\pi}{2} \sigma_{y}} \left|+\right>_{y} &= e^{-\frac{i\pi}{2}} \left|+\right>_{y} = -i \left|+\right>_{y}\\
e^{-\frac{i\pi}{2} \sigma_{y}} \left|-\right>_{y} &= e^{\frac{i\pi}{2}} \left|-\right>_{y} = i \left|-\right>_{y}
\end{aligned}
\end{equation}
$\therefore$
\begin{equation}
\begin{aligned}
e^{-\frac{i\pi}{2} \sigma_{y}} &= e^{-\frac{i\pi}{2}}\left|+\left>_{yy}\right<+\right| + e^{\frac{i\pi}{2}}\left|-\left>_{yy}\right<-\right| = -i\left|+\left>_{yy}\right<+\right| + i\left|-\left>_{yy}\right<-\right|\\
&= -i(\left|+\left>_{yy}\right<+\right| - \left|-\left>_{yy}\right<-\right|) = -i\sigma_{y}
\end{aligned}
\end{equation}
$\therefore U=-i\sigma_{y}$。\\
在$S_{z}$表象下,$S_{z}$得本征态可表示为$\xi_{m_{s}}(s_{z})$,$\displaystyle m_{s} = \pm\frac{1}{2}$时分别对应$\begin{pmatrix} 1 \\ 0 \end{pmatrix}$和$\begin{pmatrix} 0 \\ 1 \end{pmatrix}$
\begin{equation}
T\xi_{m_{s}}(s_{z}) = U\xi_{m_{s}}(s_{z}) = -i\sigma_{y}\xi_{m_{s}}(s_{z})
\end{equation}
$\therefore$
\begin{equation}
\begin{aligned}
-i\sigma_{y}\xi_{\frac{1}{2}}(s_{z})\begin{pmatrix} 1 \\ 0  \end{pmatrix} &= \begin{bmatrix} 0 & -1 \\ 1 & 0 \end{bmatrix} \begin{pmatrix} 1 \\ 0 \end{pmatrix} = \begin{pmatrix} 0 \\ 1 \end{pmatrix} \\
-i\sigma_{y}\xi_{-\frac{1}{2}}(s_{z})\begin{pmatrix} 0 \\ 1  \end{pmatrix} &= \begin{bmatrix} 0 & -1 \\ 1 & 0 \end{bmatrix} \begin{pmatrix} 0 \\ 1 \end{pmatrix} = -\begin{pmatrix} 1 \\ 0 \end{pmatrix} \\
\end{aligned}
\end{equation}
$\therefore$对自旋$\displaystyle \frac{1}{2}$粒子
\begin{equation}
T\xi_{m_{s}}(s_{z}) = (-1)^{\frac{1}{2}-m_{s}}\xi_{m_{s}}(s_{z})
\end{equation}
~\\
~\\
\textbf{2 \quad 某原子核的态矢$\left|JM\right>$由外壳层上三个中子的单粒子态$\left|j_{i}m_{i}\right> \; (i=1,2,3)$按角动量耦合规则耦合而成,在二次量子化表象下可表为
\begin{equation}\nonumber 		% \nonumber使该公式不编号
\left|j_{1}j_{2}(j_{12})j_{3};JM\right> = A\sum_{m_{1}m_{2}m_{3}m_{12}}C^{j_{12}m_{12}}_{j_{1}m_{1}j_{2}m_{2}}C^{JM}_{j_{12}m_{12}j_{3}m_{3}}a^{\dag}_{j_{1}m_{1}}a^{\dag}_{j_{2}m_{2}}a^{\dag}_{j_{3}m_{3}}\left|0\right>
\end{equation}
式中$A$为归一化因子。试按以下三种情况分别计算的取值:(1)$j_{1}\ne j_{2}\ne j_{3}$;(2)$j_{1} = j_{2}\ne j_{3}$;(3)$j_{1} = j_{2} = j_{3} = j$
}\\
~\\
解:
\begin{align*}
\left|j_{1}j_{2}(j_{12})j_{3};JM\right> &= A\sum_{m_{1}m_{2}m_{3}m_{12}}C^{j_{12}m_{12}}_{j_{1}m_{1}j_{2}m_{2}}C^{JM}_{j_{12}m_{12}j_{3}m_{3}}a^{\dag}_{j_{1}m_{1}}a^{\dag}_{j_{2}m_{2}}a^{\dag}_{j_{3}m_{3}}\left|0\right> = A\sum_{m_{1}m_{2}m_{3}m_{12}} \\ &C^{j_{12}m_{12}}_{j_{1}m_{1}j_{2}m_{2}}C^{JM}_{j_{12}m_{12}j_{3}m_{3}}\left|j_{1}j_{2}j_{3}\right>
\end{align*}
$\therefore$
\begin{equation}
\begin{aligned}
\left<j_{1}j_{2}(j_{12})j_{3};JM|j_{1}j_{2}(j_{12})j_{3};JM\right> &= \left|A\right|^{2} \sum_{m_{1}m_{23}} \sum_{m_{2}m_{3}m_{23}} \left<  j_{1}j_{2}j_{3} \right| C^{j_{12}m_{12}}_{j_{1}m_{1}j_{2}m_{2}}C^{j_{23}m_{23}}_{j_{2}m_{2}j_{3}m_{3}} \\ &C^{JM}_{j_{12}m_{12}j_{3}m_{3}}C^{JM}_{j_{1}m_{1}j_{23}m_{23}} \left| j_{1}j_{2}j_{3} \right> \\
&= \left|A\right|^{2} \sum_{m_{1}m_{23}} \sum_{m_{2}m_{3}m_{23}} C^{j_{12}m_{12}}_{j_{1}m_{1}j_{2}m_{2}}C^{j_{23}m_{23}}_{j_{2}m_{2}j_{3}m_{3}} C^{JM}_{j_{12}m_{12}j_{3}m_{3}}C^{JM}_{j_{1}m_{1}j_{23}m_{23}} \\
&= \left|A\right|^{2} \sum_{m_{1}m_{23}} U(j_{1}j_{2}Jj_{3};j_{12}j_{23}) C^{JM}_{j_{1}m_{1}j_{23}m_{23}} C^{JM}_{j_{1}m_{1}j_{23}m_{23}}\\
&= \left|A\right|^{2} U(j_{1}j_{2}Jj_{3};j_{12}j_{23}) = 1
\end{aligned}
\end{equation}
(1) 当$j_{1}\ne j_{2}\ne j_{3}$时,$\displaystyle A=\pm\frac{1}{\sqrt{U(j_{1}j_{2}Jj_{3};j_{12}j_{23})}}$,习惯选取$\displaystyle A=\frac{1}{\sqrt{U(j_{1}j_{2}Jj_{3};j_{12}j_{23})}}$;\\
(2) 引入$6-j$符号
\begin{equation}
\begin{Bmatrix} j_{1} & j_{2} & j_{12} \\ j_{3} & J & j_{23} \end{Bmatrix} = (-1)^{j_{1}+j_{2}+j_{3}+J}\frac{U(j_{1}j_{2}Jj_{3};j_{12}j_{23})}{\sqrt{(2j_{12}+1)(2j_{23}+1)}}
\end{equation}
由$6-j$符号的幺正性
\begin{align}
\sum_{j_{23}}(2j_{12}+1)(2j_{23}+1)\begin{Bmatrix} j_{1} & j_{2} & j_{12} \\ j_{3} & J & j_{23} \end{Bmatrix}\begin{Bmatrix} j_{1} & j_{2} & j_{12} \\ j_{3} & J & j_{23} \end{Bmatrix} = 1
\end{align}
$\therefore$
\begin{equation}
\sum_{j_{23}}(2j_{12}+1)(2j_{23}+1) \frac{U^{2}(j_{1}j_{2}Jj_{3};j_{12}j_{23})}{(2j_{12}+1)(2j_{23}+1)}=\sum_{j_{23}}U^{2}(j_{1}j_{2}Jj_{3};j_{12}j_{23})=1
\end{equation}
即有$\displaystyle U^{2}(j_{1}j_{2}Jj_{3};j_{12}j_{23}) = \frac{1}{2j_{min}+1}$,其中$\displaystyle j_{min}=min(j_{2},j_{3})$。\\
$\therefore \displaystyle A = \pm\frac{1}{\sqrt[4]{2j_{min}+1}}$ \\
(3) 同(2),由$6-j$符号的幺正性以及$\displaystyle j_{1} = j_{2} = j_{3} = j$可得
\begin{align}
\sum_{j_{12}}(2j_{12}+1)(2j_{23}+1)\begin{Bmatrix} j & j & j_{12} \\ j & J & j_{23} \end{Bmatrix}\begin{Bmatrix} j & j & j_{12} \\ j & J & j_{23} \end{Bmatrix} = 1
\end{align}
其中$j_{12}=j_{23}$,取值范围是$-2j,\cdot\cdot\cdot,2j$。\\
$\therefore$
\begin{equation}
\sum_{j_{23}}(2j_{12}+1)(2j_{23}+1) \frac{U^{2}(j_{1}j_{2}Jj_{3};j_{12}j_{23})}{(2j_{12}+1)(2j_{23}+1)}=\sum_{j_{23}}U^{2}(j_{1}j_{2}Jj_{3};j_{12}j_{23})=1
\end{equation}
即有$\displaystyle U(j_{1}j_{2}Jj_{3};j_{12}j_{23})^{2} = \frac{1}{2j+1}$。\\
$\therefore \displaystyle A = \pm\frac{1}{\sqrt[4]{2j+1}}$ \\
~\\
~\\

\textbf{3 \quad 对自由电子,\\
(1) 写出$Dirac$方程;\\
(2) 将电子的态函数写为$\displaystyle \psi=\begin{pmatrix} \phi \\ \chi \end{pmatrix}e^{-\frac{i}{\hbar}mc^{2}t}$,在非相对论极限下,导出所满足的方程;\\
(3) 引入$\gamma_{i}=-i\beta\alpha_{i} \quad i=1,2,3$,$\gamma_{4}=\beta$,将$Dirac$方程改写为相对论协变形式。}\\
~\\
解:\\
(1) 由$\displaystyle i\hbar\frac{\partial}{\partial t}\psi = \hat{H}\psi$,和能量本征方程$\hat{H}\psi = E\psi$,其中$\hat{H}=c\bm{\alpha \cdot p} +mc^{2}\beta$,$E^{2}=c^{2}\bm{p}^{2}+m^{2}c^{4}$
\begin{equation}
i\hbar\frac{\partial}{\partial t}\psi = \left(c\bm{\alpha \cdot p} +mc^{2}\beta\right)\psi = \left(-i\hbar c\bm{\alpha \cdot \nabla} +mc^{2}\beta\right)\psi
\end{equation}\\
其中
\begin{equation}
\beta = 
\begin{bmatrix}
I & \quad\\
\quad & -I
\end{bmatrix}
\qquad
\alpha_{i} =
\begin{bmatrix}
\quad & \sigma_{i} \\
\sigma_{i} & \quad
\end{bmatrix} \quad i=1,2,3
\end{equation}
(2) 将电子的态函数代入$Dirac$方程\\
$\therefore$
\begin{equation}
\left(c\bm{\alpha \cdot p} +mc^{2}\beta\right) \begin{pmatrix} \phi \\ \chi \end{pmatrix} = E\begin{pmatrix} \phi \\ \chi \end{pmatrix}
\end{equation}
矩阵形式为
\begin{equation}
\begin{bmatrix}
mc^{2} & c\bm{\sigma \cdot p} \\
c\bm{\sigma \cdot p} & mc^{2}
\end{bmatrix} \begin{pmatrix} \phi \\ \chi \end{pmatrix} = E\begin{pmatrix} \phi \\ \chi \end{pmatrix}
\end{equation}
其中
\begin{equation}
\begin{vmatrix}
mc^{2}-E & c\bm{\sigma \cdot p} \\
c\bm{\sigma \cdot p} & mc^{2}-E
\end{vmatrix} = 0
\end{equation}
(3) 
\begin{equation}
-c\hbar\frac{\partial\psi}{\partial x_{4}} = \left( -ic\hbar\sum_{i=1}^{3}x_{i}\frac{\partial}{\partial x_{i}} + mc^{2}\beta \right) \psi \quad (x_{4}=ict)
\end{equation}
$\therefore$
\begin{equation}
\left( -i\sum_{i=1}^{3}\alpha_{i}\frac{\partial}{\partial x_{i}} + \frac{\partial}{\partial x_{4}} + \kappa\beta \right) \psi = 0
\end{equation}
对上式左乘$\beta$,令$\gamma_{i}=-i\beta\alpha_{i},\gamma_{4}=\beta$\\
$\therefore$
\begin{equation}
\left( \sum_{i=1}^{3}\gamma_{i}\frac{\partial}{\partial x_{i}} + \gamma_{4}\frac{\partial}{\partial x_{4}} + \kappa \right) \psi = 0
\end{equation}
$\therefore  Dirac$方程相对论协变形式为
\begin{equation}
\left( \gamma_{\mu}\frac{\partial}{\partial x_{\mu}} + \kappa \right) \psi = 0 \quad \mu=1,2,3,4
\end{equation}
~\\
~\\
\textbf{4 \quad 证明质量为$m$的一维自由粒子的传播子可表示为
\begin{equation}\nonumber 		% \nonumber使该公式不编号
K(x't',xt) = \left(\frac{m}{2\pi\hbar i(t'-t)}\right)^{\frac{1}{2}} e^{\frac{i}{\hbar}S(x't',xt)}
\end{equation}
式中$S(x't',xt)$为一个经典自由粒子从点运动到点的作用量。}\\
\\
解:
\begin{equation}
S(x''t'',x't') = \int^{t''}_{t'} L(x,\dot{x},t) \dif t
\end{equation}
式中
\begin{equation}
L(x,\dot{x},t) = T - V = T = \frac{1}{2}mv^{2} = \frac{1}{2}m\left(\frac{x''-x'}{t''-t'}\right)^{2}
\end{equation}
$\therefore$由多边折线道方法$\displaystyle K_{N}(x't',xt) = C_{N}\int \cdot\cdot\cdot \int exp\left\{\frac{im}{2\hbar\epsilon}\sum_{j}^{N}\left(x_{j}-x_{j-1}\right)^{2}\right\} \dif x_{1} \cdot\cdot\cdot \dif x_{N}$
\begin{equation}\nonumber 		% \nonumber使该公式不编号
K(x't',xt) = \left(\frac{m}{2\pi\hbar i(t'-t)}\right)^{\frac{1}{2}} e^{\frac{i}{\hbar}S(x't',xt)}
\end{equation}

%%%%%%%%%%%%%%%%%%%%%%%%%%%%%%%%%%%%%%%%%%%%%%%%%%
%%%%%%%%%%%%%%%%%%%% 2008年期末题 %%%%%%%%%%%%%%%%%%%%%%%
%%%%%%%%%%%%%%%%%%%%%%%%%%%%%%%%%%%%%%%%%%%%%%%%%%
\newpage
\subsection{2008年习题}
~\\
~\\
\textbf{1 \quad 设有角动量$\bm J$,其平方$\bm J^{2}$和$z$分量$J_{z}$的共同本征态矢记为$\left| jm \right>$,有
\begin{equation}
\bm{J}^{2}\left|jm\right>=\eta_{j}\hbar^{2}\left|jm\right> \quad \hat{J_{z}}\left|jm\right>=m\hbar\left|jm\right>
\end{equation}
(1) 试证$\eta_{j}$,若一定,则磁量子数$m$有最大值$\overline{m}$和最小值$\underline{m}$,讨论二者之间的关系;\\
(2) 引入算符$\displaystyle \hat{J_{\pm}}=\hat{J_{x}} \pm i\hat{J_{y}}$,证明,$\hat{J_{\pm}}$对的作用是使升(降)$1$;\\
(3) 由(2)可知,将$\hat{J_{-}}$对$\left| jm \right>$逐次作用,能得到$m$的全部可取值,直至$\underline{m}$。但有人认为,如此作用到最后,得到一个$\underline{m}' \in (\underline{m},\underline{m}+1)$也是合理的。这种说法对吗?为什么?
}\\
~\\
解:\\
(1) 
\begin{equation}
\left<jm\left|\bm{J}^{2} - \hat{J_{z}}^{2}\right|jm\right> = \left<jm\left|\hat{J_{x}}^{2}+\hat{J_{y}}^{2}\right|jm\right>
\end{equation}
即有
\begin{equation}
\left( \eta_{j}-m^{2} \right) = \int\left|\hat{J_{x}}\psi_{jm}\right|^{2}\dif\tau + \int\left|\hat{J_{y}}\psi_{jm}\right|^{2}\dif\tau \ge 0
\end{equation}
$\therefore \displaystyle \sqrt{\eta_{j}} \ne m \ne \sqrt{\eta_{j}}$,$m$有最大值$\overline{m}$和最小值$\underline{m}$,二者关系见(2)。\\
(2) 对$\hat{J_{\pm}}$有$\displaystyle \left[\bm{J}^{2},\hat{J_{\pm}}\right]=0$,$\displaystyle \left[\hat{J_{z}},\hat{J_{\pm}}\right]=\pm\hbar\hat{J_{\pm}}$。\\
$\therefore$
\begin{equation}
\hat{J_{z}}\hat{J_{\pm}}\left|jm\right> = \left(\hat{J_{\pm}}\hat{J_{z}}\pm\hbar\hat{J_{\pm}}\right)\left|jm\right> = \left(m\pm1\right)\hat{J_{\pm}}\left|jm\right>
\end{equation}
其中$\displaystyle \hat{J_{\pm}}\left|jm\right> = \Gamma_{\pm}(m)\left|j,m\pm1\right>$。\\
利用$\hat{J_{-}}\hat{J_{+}}\left|j\overline{m}\right>=0$和$\hat{J_{+}}\hat{J_{-}}\left|j\underline{m}\right>=0$ \\
\begin{align*}
\hat{J_{-}}\hat{J_{+}}\left|j\overline{m}\right> &= \left( \bm{J}^{2}-\hat{J_{z}}^{2}-\hbar\hat{J_{z}} \right)\left|j\overline{m}\right> \\
&\Rightarrow \eta_{j}-\overline{m}^{2}-\overline{m}=0
\end{align*}
类似有$\eta_{j}-\underline{m}^{2}+\underline{m}=0$。\\
$\therefore$ 由上两式可得
\begin{equation}
\underline{m}^{2}-\underline{m}=\overline{m}^{2}+\overline{m}
\end{equation}
$\therefore \overline{m} = -\underline{m} = j$,进而可得
\begin{equation}
\left\{		% 大括号公式的写法
\begin{aligned}
\underline{m} &= \overline{m}-n\\
\overline{m} &= -\underline{m} = j\\
\end{aligned}
\right.	% 注意这个点不要拉下\right.	
\end{equation}
其中$-j=j-n,n=2j; m=j,j-1,\cdot\cdot\cdot,-j; \eta_{j}=j(j+1)$。\\
对于$\Gamma_{+}(m)$有
\begin{equation}
\left<jm\left|\hat{J_{-}}\hat{J_{+}}\right|jm\right> = |\Gamma_{+}(m)|^{2} \left<jm|jm\right>
\end{equation}
又$\because \hat{J_{-}}\hat{J_{+}}=\bm{J}^{2}-\hat{J_{z}}^{2}-\hbar\hat{J_{z}}$ \\
$\therefore$
\begin{equation}
|\Gamma_{+}(m)|^{2} = \left[ j(j+1)-m(m+1) \right]; \quad \text{同理有} \; |\Gamma_{-}(m)|^{2} = \left[ j(j+1)-m(m-1) \right]
\end{equation}\\
$\therefore$
\begin{equation}
\hat{J_{\pm}}\left|jm\right> = \sqrt{j(j+1)-m(m\pm1)} \left|j,m\pm1\right>
\end{equation}
(3) 不合理。由(2)中的结果可知$m=j,j-1,\cdot\cdot\cdot,-j$,共有$2j+1$个,不能得到$\underline{m}' \in (-j,-j+1)$。
~\\
~\\


\textbf{2 \quad 设$\bm{T}_{LM}(\tau) \; (M=L,L-1,\cdot\cdot\cdot,-L)$为一组$L$阶不可约张量算符,$\psi_{jm}(\tau)$为角动量本征函数,定义
\begin{equation}\nonumber 		% \nonumber使该公式不编号
\Psi_{JM_{J}}(\tau) = \sum_{mM}C_{jmLM}^{JM_{J}}\bm{T}_{LM}(\tau)\psi_{jm}(\tau)
\end{equation}
式中$C_{jmLM}^{JM_{J}}$为$C-G$系数。试证明如此定义的$\Psi_{JM}(\tau)$也是角动量的本征函数。}\\
~\\
证明:$J_{z}\psi_{jm}(\tau) = m\hbar\psi_{jm}(\tau)$,对一组$L$阶不可约张量算符有$\left[ J_{z},\bm{T}_{LM}(\tau) \right] = M\hbar\bm{T}_{LM}(\tau)$ \\
$\therefore$
\begin{align*}
J_{z}\Psi_{JM_{J}}(\tau) &= \sum_{mM}C_{jmLM}^{JM_{J}} \left[ M\hbar\bm{T}_{LM}(\tau) + \bm{T}_{LM}(\tau)J_{z} \right] \psi_{jm}(\tau) \\
&= \sum_{mM}C_{jmLM}^{JM_{J}} \left[ M\hbar\bm{T}_{LM}(\tau)\psi_{jm}(\tau) + \bm{T}_{LM}(\tau)m\hbar\psi_{jm}(\tau) \right] \\
&= \left(M+m\right)\hbar \sum_{mM}C_{jmLM}^{JM_{J}}\bm{T}_{LM}(\tau)\psi_{jm}(\tau)
\end{align*}
$\therefore \displaystyle J_{z}\Psi_{JM_{J}}(\tau) = \left(M+m\right)\hbar\Psi_{JM_{J}}(\tau)$,$\Psi_{JM}(\tau)$也是角动量的本征函数。
~\\
~\\



\textbf{3 \quad 试利用$Wick$定理,计算单体算符$\hat{T}$在$N$个全同粒子态函数上的平均值。}\\
~\\
解:$N$个全同粒子占据$m$个粒子态,有$\displaystyle N=\sum_{i=1}^{m}n_{i}$
\begin{equation}
\begin{aligned}
\left<\hat{T}\right> &= \left<\Psi_{N}\left|\hat{T}\right|\Psi_{N}\right> = \sum_{\alpha\beta}\left<\alpha\left|t\right|\beta\right> \left<n_{1}n_{2}\cdot\cdot\cdot n_{i}\cdot\cdot\cdot n_{m}\left| a_{\alpha}^{\dagger} a_{\beta} \right|n_{1}n_{2}\cdot\cdot\cdot n_{i}\cdot\cdot\cdot n_{m}\right> \\
&= \sum_{\alpha\beta}\left<\alpha\left|t\right|\beta\right> \sum_{\alpha_{i}}n_{i}\delta_{\alpha_{i}\alpha}\delta_{\beta\alpha_{i}} = \sum_{\alpha_{i}} \left<\alpha_{i}\left|t\right|\alpha_{i}\right> n_{i}
\end{aligned}
\end{equation}
~\\
~\\
\textbf{4 \quad $Dirac$在建立单电子相对论性运动方程时,有哪些物理上的考虑?这些物理考虑又是如何体现在$Dirac$方程里的?给出必要的论证。以及$Dirac$方程如何克服负能量困难?}\\
~\\
解:\\
$\bullet$ 考虑了几率密度正定即$\rho(\bm{r},t) \ge 0$;几率守恒$\displaystyle \frac{\dif}{\dif t}\int\dif^{3}\bm{x}\rho(\bm{r},t) = 0$;以及相对论协变。同时考虑了电子的自旋,提出电子的波函数应记为多分量的形式。\\
具体表示为:自由电子$Dirac$方程为
\begin{equation}
i\hbar\frac{\partial}{\partial t}\bm\Psi = \left(c\bm{\alpha \cdot p} +mc^{2}\beta\right)\bm\Psi = \left(-i\hbar c\bm{\alpha \cdot \nabla} +mc^{2}\beta\right)\bm\Psi
\end{equation}
其中$\hat{H} = c\bm{\alpha \cdot p} +mc^{2}\beta$,上式可化为
\begin{equation}
-i\hbar\frac{\partial}{\partial t}\bm\Psi^{\dagger} = i\hbar c\left(\bm{\nabla\Psi^{\dagger} \cdot \alpha} \right) +mc^{2}\bm\Psi^{\dagger}\beta
\end{equation}
由上两式可得
\begin{equation}
i\hbar\left( \bm\Psi^{\dagger}\frac{\partial\bm\Psi}{\partial t} + \frac{\partial\bm\Psi^{\dagger}}{\partial t}\bm\Psi \right) = \bm\Psi^{\dagger}\left(-i\hbar c\bm{\alpha \cdot \nabla} +mc^{2}\beta\right)\bm\Psi - \left( i\hbar c\bm{\nabla\Psi^{\dagger} \cdot \alpha}\bm\Psi  +mc^{2}\bm\Psi^{\dagger}\beta\bm\Psi \right)
\end{equation}
$\therefore$
\begin{equation}
\frac{\partial}{\partial t}\left( \bm\Psi^{\dagger}\bm\Psi \right) = -c \left( \bm\Psi^{\dagger}\bm{\alpha \cdot \nabla}\bm\Psi + \bm{\nabla\Psi^{\dagger}\cdot\alpha}\bm\Psi \right) = -c\bm\nabla\left( \bm\Psi^{\dagger}\bm\alpha\bm\Psi \right)
\end{equation}
进而得到几率流守恒方程$\displaystyle \frac{\partial \rho}{\partial t} + \bm{\nabla \cdot J}=0$,其中
\begin{equation}
\begin{aligned}
\rho &= \bm\Psi^{\dagger}\bm\Psi \\
\bm{J} &= \bm\Psi^{\dagger}c\bm\alpha\bm\Psi
\end{aligned}
\end{equation}
显然有$\rho(\bm{r},t) \ge 0$和$\displaystyle \frac{\dif}{\dif t}\int\dif^{3}\bm{x}\rho(\bm{r},t) = 0$成立,而对$Dirac$方程有
\begin{equation}
-c\hbar\frac{\partial\bm\Psi}{\partial x_{4}} = \left( -ic\hbar\sum_{i=1}^{3}x_{i}\frac{\partial}{\partial x_{i}} + mc^{2}\beta \right) \bm\Psi \quad (x_{4}=ict)
\end{equation}
$\therefore$
\begin{equation}
\left( -i\sum_{i=1}^{3}\alpha_{i}\frac{\partial}{\partial x_{i}} + \frac{\partial}{\partial x_{4}} + \kappa\beta \right) \bm\Psi = 0
\end{equation}
对上式左乘$\beta$,令$\gamma_{i}=-i\beta\alpha_{i},\gamma_{4}=\beta$\\
$\therefore$
\begin{equation}
\left( \sum_{i=1}^{3}\gamma_{i}\frac{\partial}{\partial x_{i}} + \gamma_{4}\frac{\partial}{\partial x_{4}} + \kappa \right) \bm\Psi = 0
\end{equation}
$\therefore  Dirac$方程相对论协变形式为
\begin{equation}
\left( \gamma_{\mu}\frac{\partial}{\partial x_{\mu}} + \kappa \right) \bm\Psi = 0
\end{equation}
$\bullet$ $Dirac$方程为克服负能量困难引入了$Dirac$海的概念,$Dirac$海总能量$E$和总电量$Q$均不可测,使所有负能级被电子填满,同时受泡利不相容原理限制正能级上的电子不能向下跃迁。负能海中每少一个电子,则有$\triangle E = E'-E=\epsilon, \triangle Q = Q'-Q=q$,由此预测了正电子的存在。
~\\
~\\
\textbf{5 \quad \\
(1)简述传播子$K(xt,x't')$的物理意义。\\
(2)设一维自由粒子从$t'$时刻运动到$t$时刻,试采用多边折线道方案,分别将时间$(t',t)$间隔二等分和三等分,计算相应的传播子。计算中,可取$\displaystyle C_{N}= \left(\frac{2\pi\hbar i\epsilon}{m}\right)^{-N/2}$。\\
(3)对你在2中得到的结果进行讨论。}\\
~\\
解:$\displaystyle \epsilon = \frac{t-t'}{N} = \frac{T}{N} \rightarrow T=N\epsilon$\\
(1) \quad 设粒子在初始时刻$t'$时刻处于空间上$x'$处,$K(xt,x't')$表示在之后的$t(t>t')$时刻粒子处于$x$处的概率波幅。\\
(2) \quad $\bullet$二等分:$N=2,T=2\epsilon$
\begin{equation}
\begin{aligned}
K_{2}(xt,x't') &= C_{2} \int e^{\frac{im}{2\hbar \epsilon}\left[(x_{1}-x')^{2} + (x-x_{1})^{2}\right]}\dif x_{1} \\
&= \left(\frac{m}{2\pi\hbar i\epsilon}\right) \sqrt{\frac{i\pi\hbar\epsilon}{m}} e^{\frac{im}{4\hbar \epsilon}(x-x')^{2}} \\
&= \frac{1}{\sqrt{2}} \left(\frac{m}{2\pi\hbar i\epsilon}\right)^{\frac{1}{2}} e^{\frac{im}{2T\hbar}(x-x')^{2}} \\
&= \left(\frac{m}{2\pi\hbar iT}\right)^{\frac{1}{2}} e^{\frac{im}{2T\hbar}(x-x')^{2}}
\end{aligned}
\end{equation}
$\therefore \displaystyle K_{2}(xt,x't') = \left(\frac{m}{2\pi\hbar iT}\right)^{\frac{1}{2}} e^{\frac{im}{2T\hbar}(x-x')^{2}}$\\

$\bullet$三等分:$N=3,T=3\epsilon$
\begin{equation}
\begin{aligned}
K_{3}(xt,x't') &= C_{3} \int e^{\frac{im}{2\hbar \epsilon}\left[(x_{1}-x')^{2} + (x_{2}-x_{1})^{2} + (x-x_{2})^{2} \right]} \dif x_{1}\dif x_{2}\\
&= \left(\frac{m}{2\pi\hbar i\epsilon}\right)^{\frac{3}{2}} \int e^{\frac{im}{2\hbar \epsilon}(x_{1}-x')^{2}} \dif x_{1} \int e^{\frac{im}{2\hbar \epsilon}\left[ (x_{2}-x_{1})^{2} + (x-x_{2})^{2} \right]} \dif x_{2} \\
&= \frac{1}{\sqrt{2}} \left(\frac{m}{2\pi\hbar i\epsilon}\right)^{\frac{3}{2}} \left(\frac{m}{2\pi\hbar i\epsilon}\right)^{-\frac{1}{2}} \int e^{\frac{im}{2\hbar \epsilon}(x_{1}-x')^{2}} e^{\frac{im}{4\hbar \epsilon}(x-x_{1})^{2}} \dif x_{1} \\
&= \frac{1}{\sqrt{2}} \left(\frac{m}{2\pi\hbar i\epsilon}\right) \sqrt{\frac{2}{3}} \left(\frac{m}{2\pi\hbar i\epsilon}\right)^{-\frac{1}{2}} e^{\frac{im}{2T\hbar}(x-x')^{2}} \\
&= \left(\frac{m}{2\pi\hbar iT}\right)^{\frac{1}{2}} e^{\frac{im}{2T\hbar}(x-x')^{2}}
\end{aligned}
\end{equation}
$\therefore \displaystyle K_{3}(xt,x't') = \left(\frac{m}{2\pi\hbar iT}\right)^{\frac{1}{2}} e^{\frac{im}{2T\hbar}(x-x')^{2}}$\\
(3) \quad 由(1)、(2)结果可得$\displaystyle K_{2}(xt,x't') = K_{3}(xt,x't') = \left(\frac{m}{2\pi\hbar iT}\right)^{\frac{1}{2}} e^{\frac{im}{2T\hbar}(x-x')^{2}}$,对于$N$等分有$\displaystyle K_{N}(xt,x't') = \left(\frac{m}{2\pi\hbar iT}\right)^{\frac{1}{2}} e^{\frac{im}{2T\hbar}(x-x')^{2}}$。\\
~\\
~\\
\textbf{6 \quad 设体系哈密顿量$\hat{H}$显含时间,但瞬时本征方程成立,且瞬时本征函数$\psi_{n}(t)$仍构成正交归一完备基。若将$t$时刻态函数展开为
\begin{equation}\nonumber 		% \nonumber使该公式不编号
\Psi(t) = \sum_{n}c_{n}(t)\psi_{n}(t)e^{i\theta_{n}(t)}
\end{equation}
这里$\displaystyle \theta_{n}(t) \equiv \int^{t}_{0}E_{n}(t')\dif t'$,试导出展开系数$c_{n}(t)$所满足的运动方程。进而在绝热近似下,证明有绝热定理成立。}\\
~\\
解:由能量本征方程$\hat{H}(t)\psi_{n}(t) = E_{n}(t)\psi_{n}(t)$和含时薛定谔方程$\displaystyle i\hbar\frac{\partial}{\partial t}\Psi(t) = \hat{H}(t)\Psi(t)$\\
\begin{equation}
i\hbar\sum_{n}\left( \dot{c}_{n}(t)\psi_{n}(t) + c_{n}(t)\dot{\psi}_{n}(t) + i\dot{\theta}_{n}(t)c_{n}(t)\psi_{n}(t) \right)e^{i\theta_{n}(t)} = \hat{H}(t)\sum_{n}c_{n}(t)\psi_{n}(t)e^{i\theta_{n}(t)}
\end{equation}
$\therefore$
\begin{equation}
\begin{aligned}
&\dot{c}_{m}(t)e^{i\theta_{m}(t)} = -\sum_{n}c_{n}(t) \left< \psi_{m}(t) \Big| \dot{\psi}_{n}(t) \right> e^{i\theta_{n}(t)} \\
&\dot{c}_{m}(t) = -c_{m}(t) \left< \psi_{m}(t) \Big| \dot{\psi}_{m}(t) \right> - {\sum\limits_{n}}'\left< \psi_{m}(t) \Big| \dot{\psi}_{n}(t) \right> e^{i(\theta_{n}(t)-\theta_{m}(t))}
\end{aligned}
\end{equation}
对$\hat{H}(t)\psi_{n}(t) = E_{n}(t)\psi_{n}(t)$两侧对时间求导
\begin{equation}
\dot{\hat{H}}(t)\psi_{n}(t) + \hat{H}(t)\dot{\psi}_{n}(t) = \dot{E}_{n}(t)\psi_{n}(t) + E_{n}(t)\dot{\psi}_{n}(t)
\end{equation}
以左矢$\left< \psi_{m}(t) \right|$作用
\begin{equation}
\left<\psi_{m}(t)\right|\dot{\hat{H}}(t)\left|\psi_{n}(t)\right> + \left<\psi_{m}(t)\left|\hat{H}(t)\right|\dot{\psi}_{n}(t)\right> = \left<\psi_{m}(t)\right|\dot{E}_{n}(t)\left|\psi_{n}(t)\right> + \left<\psi_{m}(t)\Big|E_{n}(t)\Big|\dot{\psi}_{n}(t)\right>
\end{equation}
$\therefore$
\begin{equation}
\left<\psi_{m}(t)\right|\dot{\hat{H}}(t)\left|\psi_{n}(t)\right> = (E_{n}-E_{m})\left<\psi_{m}(t) \Big| \dot{\psi}_{n}(t)\right>\delta_{mn}
\end{equation}
$\therefore$
\begin{equation}
\dot{c}_{m}(t) = -c_{m}(t) \left< \psi_{m}(t) \Big| \dot{\psi}_{m}(t) \right> - {\sum\limits_{n}}'c_{n}\frac{\left<\psi_{m}(t)\right|\dot{\hat{H}}(t)\left|\psi_{n}(t)\right>}{E_{n}-E_{m}}e^{i(\theta_{n}(t)-\theta_{m}(t))}
\end{equation}
$\hat H$趋于$0$,第二项可忽略。
$\therefore$
\begin{equation}
\dot{c}_{m}(t) = -c_{m}(t) \left< \psi_{m}(t) \Big| \dot{\psi}_{m}(t) \right> \Longrightarrow c_{m}(t) = c_{m}(0)e^{i\nu_{m}(t)}
\end{equation}
$\therefore$
\begin{equation}
c_{n}(t) = c_{n}(0)e^{i\nu_{n}(t)} \text{,其中} \nu_{n}(t)=i\int^{t}_{0} \left< \psi_{n}(t') \bigg| \frac{\partial}{\partial t'}\psi_{n}(t') \right> \dif t'
\end{equation}
特别地,如果粒子开始时处于第$n$本征态,即$c_{n}(0)=1$,$c_{m}(0)=0$,$m \ne n$,则$\Psi_{n}(t)$变为
\begin{equation}
\Psi_{n}(t) = \Psi_{n}(t)e^{i\theta_{n}(t)}e^{i\nu_{n}(t)}
\end{equation}
仍处于$\hat{H}(t)$得第$n$本征态,仅仅增加了一对相因子,证毕。
~\\
~\\


%%%%%%%%%%%%%%%%%%%%%%%%%%%%%%%%%%%%%%%%%%%%%%%%%%
%%%%%%%%%%%%%%%%%%%% 2009年期末题 %%%%%%%%%%%%%%%%%%%%%%%
%%%%%%%%%%%%%%%%%%%%%%%%%%%%%%%%%%%%%%%%%%%%%%%%%%
\newpage
\subsection{2009年习题}
~\\
~\\
\textbf{1 \quad 考虑标量波函数所描述的体系。\\
(1) 写出转动算符$U(\bm{e}_{z},\dif\theta)$的表达式。若体系在此转动变换下不变,守恒量是什么?\\
(2) 证明绕两个任意轴的无穷小转动变换$U(\bm{n}_{1},\dif\theta_{1})$和$U(\bm{n}_{2},\dif\theta_{2})$是对易的。\\
(3) 试论证,两个相继的转动变换$U(\bm{n}_{2},\theta_{2})U(\bm{n}_{1},\theta_{1})$与$R_{n_{1}}(\theta_{1})R_{n_{2}}(\theta_{2})$相对应,这里$R_{n}(\theta)$为与$U(\bm{n},\theta)$相联系的几何转动算符。
}\\
~\\
解:(1) 
\begin{equation}
U(\bm{e}_{z},\dif\theta) = \hat{I} - \dif\theta\frac{i}{\hbar}\hat{J}_{z}
\end{equation}
若体系在此转动变换下不变,守恒量是角动量。\\
(2) 绕任意轴的无穷小转动变换有$\displaystyle U(\bm{n},\dif\theta) = e^{-\frac{i}{\hbar}\dif\theta\bm{n}\cdot\bm{J}} = \hat{I} - \dif\theta\frac{i}{\hbar}\hat{J}_{n}$\\
$\therefore$
\begin{equation}
\begin{aligned}
U(\bm{n}_{1},\dif\theta_{1})U(\bm{n}_{2},\dif\theta_{2}) &= \left(\hat{I} - \dif\theta_{1}\frac{i}{\hbar}\hat{J}_{n_{1}}\right) \left(\hat{I} - \dif\theta_{2}\frac{i}{\hbar}\hat{J}_{n_{2}}\right) \\
&= \hat{I} - \frac{i}{\hbar}\left( \dif\theta_{1}\hat{J}_{n_{1}} + \dif\theta_{2}\hat{J}_{n_{2}} \right) + O(\dif\theta^{2})
\end{aligned}
\end{equation}
略去高阶无穷小得
\begin{equation}\nonumber 		% \nonumber使该公式不编号
U(\bm{n}_{1},\dif\theta_{1})U(\bm{n}_{2},\dif\theta_{2}) = \hat{I} - \frac{i}{\hbar}\left( \dif\theta_{1}\hat{J}_{n_{1}} + \dif\theta_{2}\hat{J}_{n_{2}} \right)
\end{equation}
同理可得
\begin{equation}\nonumber 		% \nonumber使该公式不编号
U(\bm{n}_{2},\dif\theta_{2})U(\bm{n}_{1},\dif\theta_{1}) = \hat{I} - \frac{i}{\hbar}\left( \dif\theta_{1}\hat{J}_{n_{1}} + \dif\theta_{2}\hat{J}_{n_{2}} \right)
\end{equation}
$\therefore U(\bm{n}_{2},\dif\theta_{2})U(\bm{n}_{1},\dif\theta_{1}) = U(\bm{n}_{1},\dif\theta_{1})U(\bm{n}_{2},\dif\theta_{2})$,即有
\begin{equation}
\left[U(\bm{n}_{1},\dif\theta_{1}),U(\bm{n}_{2},\dif\theta_{2})\right] = 0
\end{equation}
$\therefore$ 绕两个任意轴的无穷小转动变换$U(\bm{n}_{1},\dif\theta_{1})$和$U(\bm{n}_{2},\dif\theta_{2})$是对易的。\\
(3) 在转动变换$\displaystyle U(\bm{n},\dif\theta) = e^{-\frac{i}{\hbar}\dif\theta\bm{n}\cdot\bm{J}}$下,考虑$\bm{J}=\bm{L}$,$\bm{r}'=R_{n}(\theta)\bm{r}$,体系波函数有
\begin{equation}\nonumber 		% \nonumber使该公式不编号
\Psi(\bm{r}') = e^{-\frac{i}{\hbar}\dif\theta\bm{n}\cdot\bm{J}}\Psi(\bm{r}) = \Psi\left[R_{n}(\theta)\bm{r}\right] =\widetilde\Psi(\bm{r})
\end{equation}		% \widetilde在字母上加波浪线
$\therefore$ 
\begin{equation}
\begin{aligned}
U(\bm{n}_{2},\theta_{2})U(\bm{n}_{1},\theta_{1})\Psi(\bm{r}) &= U(\bm{n}_{2},\theta_{2})\widetilde\Psi(\bm{r_{1}}) = \widetilde\Psi\left[R_{n_{2}}(\theta_{2})\bm{r}\right] \\
&= \Psi\left[R_{n_{1}}(\theta_{1})R_{n_{2}}(\theta_{2})\bm{r}\right]
\end{aligned}
\end{equation}
$\therefore$ 我们认为 $U(\bm{n}_{2},\theta_{2})U(\bm{n}_{1},\theta_{1}) \rightleftharpoons R_{n_{1}}(\theta_{1})R_{n_{2}}(\theta_{2})$。\\
~\\
~\\
\textbf{2 \quad 试利用一阶张量投影定理计算在态$\displaystyle ^{2}P_{\frac{3}{2}}$上电子的自旋算符$S_{z}$的约化矩阵元。}\\
~\\
解:态$\displaystyle ^{2}P_{\frac{3}{2}}$上电子轨道量子数为$l=1$,自旋量子数为$\displaystyle s=\frac{1}{2}$,总量子数为$\displaystyle j=\frac{3}{2}$\\
$\therefore$
\begin{align*}
\left< jm' \left| \hat{S}_{z} \right| jm \right> &= \frac{\left<jm'\left|\hat{J}_{z}\right|jm\right> \left<jm\left|(\hat{J}\cdot\hat{S})\right|jm\right>}{j(j+1)\hbar^{2}} \\
&= \frac{m\hbar\delta_{m'm} \left<jm\left|(\hat{L}\cdot\hat{S}+\hat{S}^{2})\right|jm\right>}{j(j+1)\hbar^{2}} \\
&= \frac{m\hbar\delta_{m'm} \displaystyle\frac{1}{2} \left<jm\left|(\hat{J}^{2}- \hat{L}^{2}+\hat{S}^{2})\right|jm\right>}{j(j+1)\hbar^{2}} \\
&= \frac{1}{2}\frac{m\hbar\delta_{m'm} [j(j+1)\hbar^{2}-l(l+1)\hbar^{2}+s(s+1)\hbar^{2}] }{j(j+1)\hbar^{2}} \\
&= \frac{2}{5}m\hbar\delta_{m'm}
\end{align*}
$\therefore$ $S_{z}$的约化矩阵元为
\begin{align*}
\left<j\left\|\hat{S}_{z}\right\|j\right> &= \frac{\left<jm'\left|\hat{S}_{z}\right|jm\right>}{C_{jmLM}^{j'm'}} \\
&= \frac{\displaystyle \frac{2}{5}m\hbar\delta_{m'm}}{C_{jmLM}^{jm'}} \\
&= \frac{2}{5}m\hbar \cdot \frac{1}{C_{jmLM}^{jm}} = \frac{2}{5}m\hbar \cdot \frac{1}{C_{jm10}^{jm}} = \frac{\sqrt{15}}{5}\hbar
\end{align*}
~\\
~\\


\textbf{3 \quad 三个全同玻色子构成的体系中,两个单粒子态$\alpha_{1}$和$\alpha_{2}$的占据数分别为$2$和$1$,试分别在坐标表象和粒子数表象下写出体系态函数的表达式,并求出可分离型二体算符$V$在此态上的平均值,这里
\begin{equation}\nonumber 		% \nonumber使该公式不编号
V = \sum_{i<j}V(i,j) \equiv \sum_{i<j}v(i)v(j)
\end{equation}
}
~\\
解:坐标表象下
\begin{align*}
\Psi(1,2,3) &= \sqrt{\frac{2!\cdot1!}{3!}}\sum_{P_{i}}\hat{P_{i}}\phi_{\alpha_{1}}(1)\phi_{\alpha_{1}}(2)\phi_{\alpha_{2}}(3) \\
&= \frac{1}{\sqrt{3}}\left[ \phi_{\alpha_{1}}(1)\phi_{\alpha_{1}}(2)\phi_{\alpha_{2}}(3) + \phi_{\alpha_{1}}(3)\phi_{\alpha_{1}}(1)\phi_{\alpha_{2}}(2) + \phi_{\alpha_{1}}(2)\phi_{\alpha_{1}}(3)\phi_{\alpha_{2}}(1) \right]
\end{align*}
粒子数表象下
\begin{align*}
\left| \Psi(1,2,3) \right> &= \frac{1}{\sqrt{2!1!}}a^{\dag}_{\alpha_{1}}a^{\dag}_{\alpha_{1}}a^{\dag}_{\alpha_{2}} \left|0\right> \\
&= \frac{1}{\sqrt{2}}a^{\dag}_{\alpha_{1}}a^{\dag}_{\alpha_{1}}a^{\dag}_{\alpha_{2}} \left|0\right>
\end{align*}
$\therefore$ 
\begin{align*}
\left<V\right> &= \left< \Psi(1,2,3) \right| V \left| \Psi(1,2,3) \right> \\
&= \frac{1}{2} \left<0\right| a_{\alpha_{2}}a_{\alpha_{1}}a_{\alpha_{1}} \sum_{i<j}v(i)v(j) a^{\dag}_{\alpha_{1}}a^{\dag}_{\alpha_{1}}a^{\dag}_{\alpha_{2}} \left|0\right> \\
&= \frac{1}{2} \sum_{\alpha\beta} \left<\alpha\left|v_{i}\right|\beta\right> \sum_{\gamma\delta} \left<\gamma\left|v_{j}\right|\delta\right> \left<0\right| a_{\alpha_{2}}a_{\alpha_{1}}a_{\alpha_{1}} a^{\dag}_{\alpha_{1}}a^{\dag}_{\alpha_{1}}a^{\dag}_{\alpha_{2}} \left|0\right>
\end{align*}
由Wick定理
\begin{equation}
\begin{aligned}
\left<V\right> &= \frac{1}{2} \sum_{\alpha_{i}<\alpha_{j}} \left( {\left<\alpha_{i}\left|v_{i}\right|\alpha_{i}\right>\left<\alpha_{j}\left|v_{j}\right|\alpha_{j}\right> - \left<\alpha_{i}\left|v_{i}\right|\alpha_{j}\right>\left<\alpha_{j}\left|v_{j}\right|\alpha_{i}\right>} \right. \\
 & \left. {- \left<\alpha_{j}\left|v_{i}\right|\alpha_{i}\right>\left<\alpha_{i}\left|v_{j}\right|\alpha_{j}\right> + \left<\alpha_{j}\left|v_{i}\right|\alpha_{j}\right>\left<\alpha_{i}\left|v_{j}\right|\alpha_{i}\right>} \right)
\end{aligned}
\end{equation}
~\\
~\\
\textbf{4 \quad 写出在电磁场$\left( \bm{A},\phi \right)$中运动的电子的相对论性运动方程,推导其非相对论极限,并对结果进行讨论。}\\
~\\
解:有电磁场$\left(\bm{A},\phi\right)$情况下的Dirac方程为
\begin{equation}
\left(i\hbar\frac{\partial}{\partial t}-q\phi\right) \bm\Psi\left(\bm{r},t\right) = \left[c\bm\alpha\cdot\left(\bm{p}-\frac{q}{c}\bm{A}\right)+mc^{2}\beta\right] \bm\Psi\left(\bm{r},t\right)
\end{equation}
$\therefore$ 电子在电磁场情况下的Dirac方程为,
\begin{equation}\nonumber 		% \nonumber使该公式不编号
\left(i\hbar\frac{\partial}{\partial t}+e\bm\phi\right) \bm\Psi\left(\bm{r},t\right) = \left[c\bm\alpha\cdot\left(\bm{p}+\frac{e}{c}\bm{A}\right)+mc^{2}\beta\right] \bm\Psi\left(\bm{r},t\right)
\end{equation}
讨论其非相对论极限,令$\displaystyle \bm\Psi\left(\bm{r},t\right) = \begin{pmatrix} \phi \\ \chi \end{pmatrix}e^{-\frac{i}{\hbar}mc^{2}t}$,上式可改写为
\begin{equation}
i\hbar\frac{\partial}{\partial t} \left[ \begin{pmatrix} \phi \\ \chi \end{pmatrix}e^{-\frac{i}{\hbar}mc^{2}t} \right] = \left[ c\bm\alpha\cdot\left(\bm{p} +\frac{e}{c}\bm{A}\right) - e\bm\phi +mc^{2}\beta \right]\begin{pmatrix} \phi \\ \chi \end{pmatrix}e^{-\frac{i}{\hbar}mc^{2}t}
\end{equation}
$\therefore$
\begin{equation}
i\hbar\frac{\partial}{\partial t} \begin{pmatrix} \phi \\ \chi \end{pmatrix} = \left[ c\bm\alpha\cdot\left(\bm{p} +\frac{e}{c}\bm{A}\right) - e\bm\phi +mc^{2}\beta -mc^{2} \right]\begin{pmatrix} \phi \\ \chi \end{pmatrix}
\end{equation}
$\therefore$ 代入$\alpha=\begin{bmatrix} 0 & \bm\sigma \\ \bm\sigma & 0\end{bmatrix}, \beta=\begin{bmatrix} I & 0 \\ 0 & I \end{bmatrix}$
\begin{equation}
i\hbar\frac{\partial}{\partial t} \begin{pmatrix} \phi \\ \chi \end{pmatrix} = \begin{bmatrix}
-e\bm\phi & \displaystyle c\bm\sigma\cdot\left(\bm{p} +\frac{e}{c}\bm{A}\right) \\
\displaystyle c\bm\sigma\cdot\left(\bm{p} +\frac{e}{c}\bm{A}\right) & -e\bm\phi-2mc^{2}
\end{bmatrix} \begin{pmatrix} \phi \\ \chi \end{pmatrix} 
\end{equation}
解得
\begin{align*}
i\hbar\frac{\partial}{\partial t}\phi &= c\bm\sigma\cdot\left(\bm{p} +\frac{e}{c}\bm{A}\right)\chi - e\bm\phi\phi \\
i\hbar\frac{\partial}{\partial t}\chi &= c\bm\sigma\cdot\left(\bm{p} +\frac{e}{c}\bm{A}\right)\phi - \left(e\bm\phi + 2mc^{2}\right)\chi
\end{align*}
非相对论极限下
\begin{equation}
\chi = \frac{1}{2mc}\bm\sigma\cdot\left(\bm{p} +\frac{e}{c}\bm{A}\right)\phi
\end{equation}
$\therefore$
\begin{equation}
i\hbar\frac{\partial}{\partial t}\phi = \frac{1}{2m}\left[\bm\sigma\cdot\left(\bm{p} +\frac{e}{c}\bm{A}\right)\right]^{2}\phi - e\bm\phi\phi 
	\tag{$\bigstar$}
\end{equation}
由$(\bm\sigma\cdot\bm{A})(\bm\sigma\cdot\bm{B})=\bm{A}\cdot\bm{B}+i\bm\sigma(\bm{A}\times\bm{B})$可得
\begin{equation}
\left[\bm\sigma\cdot\left(\bm{p} +\frac{e}{c}\bm{A}\right)\right]^{2} = \left(\bm{p} +\frac{e}{c}\bm{A}\right)^{2} + \bm\sigma\cdot\frac{e\hbar}{m}\bm{B}
\end{equation}
将上式代入$\bigstar$式,得
\begin{equation}
i\hbar\frac{\partial}{\partial t}\phi = \left[\frac{1}{2m}\left(\bm{p} +\frac{e}{c}\bm{A}\right)^{2} + \frac{e\hbar}{2mc}\bm\sigma\cdot\bm{B} -e\bm\phi \right]\phi
\end{equation}

有电磁场$\left(\bm{A},\bm\phi\right)$情况下的Klein-Gordon方程为
\begin{equation}
\left(i\hbar\frac{\partial}{\partial t}-q\bm\phi\right)^{2} \bm\Psi\left(\bm{r},t\right) = \left[c^{2}\left(\bm{p}-\frac{q}{c}\bm{A}\right)^{2}+m^{2}c^{4}\right] \bm\Psi\left(\bm{r},t\right)
\end{equation}
在非相对论极限下,令$\displaystyle \bm\Psi\left(\bm{r},t\right) = \psi\left(\bm{r},t\right)e^{-\frac{i}{\hbar}mc^{2}t}$,代入上式得
\begin{equation}
i\hbar\frac{\partial}{\partial t}\psi\left(\bm{r},t\right) = \left[\frac{1}{2m}\left( \bm{p}-\frac{q}{c}\bm{A} \right)^{2}+q\bm\phi\right]\psi\left(\bm{r},t\right)
\end{equation}
以上为非相对论极限下的Klein-Gordon方程,与之相比,内禀自旋不再作为一个假定,它已经包含在方程之中,引入了玻耳兹子$\displaystyle \mu_{B}=\frac{e\hbar}{2mc}$。\\
~\\
~\\
\textbf{5 \quad 设质量为$m$的粒子在势场$V(x,t)$中运动,$t$时刻粒子的状态为$\psi(x',t)$。按路径积分理论,在$t+\epsilon\;(\epsilon \rightarrow 0^{+})$时刻,粒子的状态可表为
\begin{equation}\nonumber 		% \nonumber使该公式不编号
\psi(x,t+\epsilon) = C\int\dif x' exp\left[\frac{i}{\hbar}s(x'\rightarrow x,t\rightarrow t+\epsilon)\right] \psi(x,t')
\end{equation}
试确定式中$C$的表达式}\\
~\\
解:\begin{equation}
C=\left(\frac{2\pi\hbar i\epsilon}{m}\right)^{-\frac{1}{2}}
\end{equation}



%%%%%%%%%%%%%%%%%%%%%%%%%%%%%%%%%%%%%%%%%%%%%%%%%%
%%%%%%%%%%%%%%%%%%%% 2010年期末题 %%%%%%%%%%%%%%%%%%%%%%%
%%%%%%%%%%%%%%%%%%%%%%%%%%%%%%%%%%%%%%%%%%%%%%%%%%
\newpage
\subsection{2010年习题}
~\\
~\\
\textbf{1 \quad 设体系角动量为$\bm J$,考虑绕$x$轴旋转角$\beta$的转动,写出转动算符$U(\bm{e_{x}},\beta)$的表达式,并用小$d$函数表出其矩阵元$\displaystyle \left<jm'\left|U(\bm{e_{x}},\beta)\right|jm\right>$。}\\
~\\
解:$U(\bm{e_{x}},\beta)$可写为\\
\begin{equation}
U(\bm{e_{x}},\beta) = e^{-\frac{i}{\hbar}\beta\bm{e_{x}}\cdot\bm{J}} = e^{-\frac{i}{\hbar}\beta J_{x}}
\end{equation}
以及
\begin{equation}
d^{j}_{m'm}(\beta) = \left<jm'\left|e^{-\frac{i}{\hbar}\beta J_{y}}\right|jm\right>
\end{equation}
以欧拉角表示D函数
\begin{equation}
D_{m'm}^{j}(-\frac{\pi}{2},\beta,\frac{\pi}{2}) = \left<jm'\left|U(\bm{e_{x}},\beta)\right|jm\right>
\end{equation}
又$\because$
\begin{equation}
D_{m'm}^{j}(-\frac{\pi}{2},\beta,\frac{\pi}{2}) = e^{im'\frac{\pi}{2}}d_{m'm}^{j}(\beta)e^{-im\frac{\pi}{2}}
\end{equation}
$\therefore \left<jm'\left|U(\bm{e_{x}},\beta)\right|jm\right> = e^{im'\frac{\pi}{2}}d_{m'm}^{j}(\beta)e^{-im\frac{\pi}{2}}$。\\
~\\
~\\

\textbf{2 \quad }
~\\
~\\
\textbf{3 \quad 写出$Dirac$单电子相对论性运动方程,由此导出几率流守恒方程,并说明这里的几率密度满足什么要求。再引入$\gamma$矩阵,$(\gamma_{\mu},\mu=1,2,3,4)$,将$Dirac$方程化为相对论协变形式。}
~\\
解:自由电子$Dirac$方程为
\begin{equation}
i\hbar\frac{\partial}{\partial t}\bm\Psi = \left(c\bm{\alpha \cdot p} +mc^{2}\beta\right)\bm\Psi = \left(-i\hbar c\bm{\alpha \cdot \nabla} +mc^{2}\beta\right)\bm\Psi
\end{equation}
其中$\hat{H} = c\bm{\alpha \cdot p} +mc^{2}\beta$,上式可化为
\begin{equation}
-i\hbar\frac{\partial}{\partial t}\bm\Psi^{\dagger} = i\hbar c\left(\bm{\nabla\Psi^{\dagger} \cdot \alpha} \right) +mc^{2}\bm\Psi^{\dagger}\beta
\end{equation}
由上两式可得
\begin{equation}
i\hbar\left( \bm\Psi^{\dagger}\frac{\partial\bm\Psi}{\partial t} + \frac{\partial\bm\Psi^{\dagger}}{\partial t}\bm\Psi \right) = \bm\Psi^{\dagger}\left(-i\hbar c\bm{\alpha \cdot \nabla} +mc^{2}\beta\right)\bm\Psi - \left( i\hbar c\bm{\nabla\Psi^{\dagger} \cdot \alpha}\bm\Psi  +mc^{2}\bm\Psi^{\dagger}\beta\bm\Psi \right)
\end{equation}
$\therefore$
\begin{equation}
\frac{\partial}{\partial t}\left( \bm\Psi^{\dagger}\bm\Psi \right) = -c \left( \bm\Psi^{\dagger}\bm{\alpha \cdot \nabla}\bm\Psi + \bm{\nabla\Psi^{\dagger}\cdot\alpha}\bm\Psi \right) = -c\bm\nabla\left( \bm\Psi^{\dagger}\bm\alpha\bm\Psi \right)
\end{equation}
进而得到几率流守恒方程$\displaystyle \frac{\partial \rho}{\partial t} + \bm{\nabla \cdot J}=0$,其中
\begin{equation}
\begin{aligned}
\rho &= \bm\Psi^{\dagger}\bm\Psi \\
\bm{J} &= \bm\Psi^{\dagger}c\bm\alpha\bm\Psi
\end{aligned}
\end{equation}
显然有$\rho(\bm{r},t) \ge 0$和$\displaystyle \frac{\dif}{\dif t}\int\dif^{3}\bm{x}\rho(\bm{r},t) = 0$成立,而对$Dirac$方程有
\begin{equation}
-c\hbar\frac{\partial\bm\Psi}{\partial x_{4}} = \left( -ic\hbar\sum_{i=1}^{3}x_{i}\frac{\partial}{\partial x_{i}} + mc^{2}\beta \right) \bm\Psi \quad (x_{4}=ict)
\end{equation}
$\therefore$
\begin{equation}
\left( -i\sum_{i=1}^{3}\alpha_{i}\frac{\partial}{\partial x_{i}} + \frac{\partial}{\partial x_{4}} + \kappa\beta \right) \bm\Psi = 0
\end{equation}
对上式左乘$\beta$,令$\gamma_{i}=-i\beta\alpha_{i},\gamma_{4}=\beta$\\
$\therefore$
\begin{equation}
\left( \sum_{i=1}^{3}\gamma_{i}\frac{\partial}{\partial x_{i}} + \gamma_{4}\frac{\partial}{\partial x_{4}} + \kappa \right) \bm\Psi = 0
\end{equation}
$\therefore  Dirac$方程相对论协变形式为
\begin{equation}
\left( \gamma_{\mu}\frac{\partial}{\partial x_{\mu}} + \kappa \right) \bm\Psi = 0 \quad \mu=1,2,3,4
\end{equation}
~\\
~\\
\textbf{4 \quad 设一个量子体系的哈密顿量为$\hat{H}$,初始时刻体系处于状态$\displaystyle \left|\psi(0)\right> = \left|\psi(t=0)\right>$。\\
(1) 给出该体系在任意时刻,态矢量的形式解。\\
(2) 由此推出,在坐标表象下,时刻的波函数可写为
\begin{equation}\nonumber 		% \nonumber使该公式不编号
\psi(\bm{r},t) = \int\dif^{3}x'\left<\bm{r}\left|e^{-\frac{i}{\hbar}\hat{H}(t-t')}\right|\bm{r}'\right> \psi(r',t')
\end{equation}
其中,$\displaystyle K(\bm{r}t,\bm{r}'t')$为传播子,试阐述其物理意义。\\
(3) 按费曼假定,$\displaystyle K(\bm{r}t,\bm{r}'t')$与体系的作用量$S[\bm{r}(t)]$之间存在什么样的关系?}
~\\
解:(1) 由含时薛定谔方程$\displaystyle i\hbar\frac{\partial }{\partial t} \left|\psi(t)\right> = \hat{H} \left|\psi(t)\right>$
\begin{equation}\nonumber 		% \nonumber使该公式不编号
\left|\psi(t)\right> = e^{-\frac{i}{\hbar}\hat{H}t}\left|\psi(0)\right>
\end{equation}
(2) 由(1)得
\begin{align*}
&\left<\bm{r}|\psi(t)\right> = \int\dif^{3}x'\left<\bm{r}\left|e^{-\frac{i}{\hbar}\hat{H}(t-t')}\right|\bm{r}'\right> \left<\bm{r}'\Big|\psi(t)\right> \\
&\Rightarrow \psi(\bm{r},t) = \int\dif^{3}x'\left<\bm{r}\left|e^{-\frac{i}{\hbar}\hat{H}(t-t')}\right|\bm{r}'\right> \psi(r',t')
\end{align*}
其中,$\displaystyle K(\bm{r}t,\bm{r}'t')=\left<\bm{r}\left|e^{-\frac{i}{\hbar}\hat{H}(t-t')}\right|\bm{r}'\right>$为传播子,物理意义为:$t'$时刻位于$\bm{r}'$的粒子,$t$时刻位于$\bm{r}$的概率波幅为$\psi(\bm{r},t) = K(\bm{r}t,\bm{r}'t')$。\\
(3) 按费曼假定$\displaystyle K(\bm{r}t,\bm{r}'t') = \int e^{\frac{i}{\hbar}S[\bm{r}(t)]}D[\bm{r}(t)]$。\\
~\\
~\\


\newpage
\section{2011年习题~2017年习题}
%%%%%%%%%%%%%%%%%%%%%%%%%%%%%%%%%%%%%%%%%%%%%%%%%%
%%%%%%%%%%%%%%%%%%%% 2011年期末题 %%%%%%%%%%%%%%%%%%%%%%%
%%%%%%%%%%%%%%%%%%%%%%%%%%%%%%%%%%%%%%%%%%%%%%%%%%
\subsection{2011年习题}
~\\
~\\
\textbf{1 \quad }\\
~\\
~\\
\textbf{2 \quad }\\
~\\
~\\
\textbf{3 \quad }\\
~\\
~\\
\textbf{4 \quad 已知一维线性谐振子的传播子为
\begin{equation}\nonumber 		% \nonumber使该公式不编号
K(xt,x't') = \left[ \frac{m\omega}{2\pi i\hbar\sin(\omega T)} \right]^{1/2} \exp\left\{ \frac{i}{\hbar}\frac{m\omega}{2\sin(\omega T)} \left[ \left(x^{2}+x'^{2}\right)\cos(\omega T)-2xx' \right] \right\}
\end{equation}
式中和分别为谐振子的质量和振动频率,。\\
(1) 指出的物理意义,写出其所满足的运动方程。\\
(2) 证明,对$t'<t_{1}<t$,有$\displaystyle K(xt,x't') = \int\dif x_{1}K(xt,x_{1}t_{1}) K(x_{1}t_{1},x't') $ \\
(3) 由所给谐振子传播子导出一维自由粒子的传播子。}\\
~\\
解:\\
(1) 物理意义为$t'$时刻在$x'$位置的粒子,在$t$时刻在$x$位置被观测到的概率。满足的方程为
\begin{equation}
(i\hbar\frac{\partial}{\partial t}+\frac{\hbar^{2}}{2m}\frac{\dif^{2}}{\dif x^{2}})K(xt,x't')=i\hbar\delta(x-x')\delta(t-t')
\end{equation}
(2) 一维线性谐振子作用量
\begin{equation}
S(xt,x't') = \exp\left\{ \frac{i}{\hbar}\frac{m\omega}{2\sin(\omega T)} \left[ \left(x^{2}+x'^{2}\right)\cos(\omega T)-2xx' \right] \right\}
\end{equation}
(3) 由多边折线道方法$\displaystyle K_{N}(x''t'',x't') = C_{N}\int \cdot\cdot\cdot \int exp\left\{\frac{im}{2\hbar\epsilon}\sum_{j}^{N}\left(x_{j}-x_{j-1}\right)^{2}\right\} \dif x_{1} \cdot\cdot\cdot \dif x_{N}$ \\
$\therefore$
\begin{align*}
K_{N}(x''t'',x't') &= C_{N}\int \cdot\cdot\cdot \int exp\left\{\frac{im}{2\hbar\epsilon}\sum_{j}^{N}\left(x_{j}-x_{j-1}\right)^{2}\right\} \dif x_{1} \cdot\cdot\cdot \dif x_{N} \\
&= \left(\frac{2\pi\hbar i\epsilon}{m}\right)^{-N/2}\int \cdot\cdot\cdot \int exp\left\{\frac{im}{2\hbar\epsilon}\sum_{j}^{N}\left(x_{j}-x_{j-1}\right)^{2}\right\} \dif x_{1} \cdot\cdot\cdot \dif x_{N} \\
&= \left(\frac{m}{2\pi\hbar i(t''-t')}\right)^{\frac{1}{2}} e^{\frac{im}{2(t''-t')\hbar}(x''-x')^{2}}
\end{align*}
~\\
~\\



%%%%%%%%%%%%%%%%%%%%%%%%%%%%%%%%%%%%%%%%%%%%%%%%%%
%%%%%%%%%%%%%%%%%%%% 2012年期末题 %%%%%%%%%%%%%%%%%%%%%%%
%%%%%%%%%%%%%%%%%%%%%%%%%%%%%%%%%%%%%%%%%%%%%%%%%%
\newpage
\subsection{2012年习题}
~\\
~\\
\textbf{1 \quad }\\
~\\
~\\
\textbf{2 \quad }\\
~\\
~\\
\textbf{3 \quad 写出有电磁场$\left(\bm{A},\bm\phi\right)$情况下的$Klein-Gordon$方程,推导其非相对论极限,并与电磁场中$Dirac$方程的非相对论极限比较异同。}\\
~\\
解:有电磁场$\left(\bm{A},\bm\phi\right)$情况下的Klein-Gordon方程为
\begin{equation}
\left(i\hbar\frac{\partial}{\partial t}-q\bm\phi\right)^{2} \bm\Psi\left(\bm{r},t\right) = \left[c^{2}\left(\bm{p}-\frac{q}{c}\bm{A}\right)^{2}+m^{2}c^{4}\right] \bm\Psi\left(\bm{r},t\right)
\end{equation}
在非相对论极限下,令$\displaystyle \bm\Psi\left(\bm{r},t\right) = \psi\left(\bm{r},t\right)e^{-\frac{i}{\hbar}mc^{2}t}$,代入上式得
\begin{equation}
i\hbar\frac{\partial}{\partial t}\psi\left(\bm{r},t\right) = \left[\frac{1}{2m}\left( \bm{p}-\frac{q}{c}\bm{A} \right)^{2}+q\bm\phi\right]\psi\left(\bm{r},t\right)
\end{equation}
以上为非相对论极限下的Klein-Gordon方程。
~\\
有电磁场$\left(\bm{A},\phi\right)$情况下的Dirac方程为
\begin{equation}
\left(i\hbar\frac{\partial}{\partial t}-q\phi\right) \bm\Psi\left(\bm{r},t\right) = \left[c\bm\alpha\cdot\left(\bm{p}-\frac{q}{c}\bm{A}\right)+mc^{2}\beta\right] \bm\Psi\left(\bm{r},t\right)
\end{equation}
以电子为例讨论其非相对论极限,电子在电磁场情况下的Dirac方程为
\begin{equation}\nonumber 		% \nonumber使该公式不编号
\left(i\hbar\frac{\partial}{\partial t}+e\bm\phi\right) \bm\Psi\left(\bm{r},t\right) = \left[c\bm\alpha\cdot\left(\bm{p}+\frac{e}{c}\bm{A}\right)+mc^{2}\beta\right] \bm\Psi\left(\bm{r},t\right)
\end{equation}
令$\displaystyle \bm\Psi\left(\bm{r},t\right) = \begin{pmatrix} \phi \\ \chi \end{pmatrix}e^{-\frac{i}{\hbar}mc^{2}t}$,上式可改写为
\begin{equation}
i\hbar\frac{\partial}{\partial t} \left[ \begin{pmatrix} \phi \\ \chi \end{pmatrix}e^{-\frac{i}{\hbar}mc^{2}t} \right] = \left[ c\bm\alpha\cdot\left(\bm{p} +\frac{e}{c}\bm{A}\right) - e\bm\phi +mc^{2}\beta \right]\begin{pmatrix} \phi \\ \chi \end{pmatrix}e^{-\frac{i}{\hbar}mc^{2}t}
\end{equation}
$\therefore$
\begin{equation}
i\hbar\frac{\partial}{\partial t} \begin{pmatrix} \phi \\ \chi \end{pmatrix} = \left[ c\bm\alpha\cdot\left(\bm{p} +\frac{e}{c}\bm{A}\right) - e\bm\phi +mc^{2}\beta -mc^{2} \right]\begin{pmatrix} \phi \\ \chi \end{pmatrix}
\end{equation}
$\therefore$ 代入$\alpha=\begin{bmatrix} 0 & \bm\sigma \\ \bm\sigma & 0\end{bmatrix}, \beta=\begin{bmatrix} I & 0 \\ 0 & I \end{bmatrix}$
\begin{equation}
i\hbar\frac{\partial}{\partial t} \begin{pmatrix} \phi \\ \chi \end{pmatrix} = \begin{bmatrix}
-e\bm\phi & \displaystyle c\bm\sigma\cdot\left(\bm{p} +\frac{e}{c}\bm{A}\right) \\
\displaystyle c\bm\sigma\cdot\left(\bm{p} +\frac{e}{c}\bm{A}\right) & -e\bm\phi-2mc^{2}
\end{bmatrix} \begin{pmatrix} \phi \\ \chi \end{pmatrix} 
\end{equation}
解得
\begin{align*}
i\hbar\frac{\partial}{\partial t}\phi &= c\bm\sigma\cdot\left(\bm{p} +\frac{e}{c}\bm{A}\right)\chi - e\bm\phi\phi \\
i\hbar\frac{\partial}{\partial t}\chi &= c\bm\sigma\cdot\left(\bm{p} +\frac{e}{c}\bm{A}\right)\phi - \left(e\bm\phi + 2mc^{2}\right)\chi
\end{align*}
非相对论极限下
\begin{equation}
\chi = \frac{1}{2mc}\bm\sigma\cdot\left(\bm{p} +\frac{e}{c}\bm{A}\right)\phi
\end{equation}
$\therefore$
\begin{equation}
i\hbar\frac{\partial}{\partial t}\phi = \frac{1}{2m}\left[\bm\sigma\cdot\left(\bm{p} +\frac{e}{c}\bm{A}\right)\right]^{2}\phi - e\bm\phi\phi 
	\tag{$\bigstar$}
\end{equation}
由$(\bm\sigma\cdot\bm{A})(\bm\sigma\cdot\bm{B})=\bm{A}\cdot\bm{B}+i\bm\sigma(\bm{A}\times\bm{B})$可得
\begin{equation}
\left[\bm\sigma\cdot\left(\bm{p} +\frac{e}{c}\bm{A}\right)\right]^{2} = \left(\bm{p} +\frac{e}{c}\bm{A}\right)^{2} + \bm\sigma\cdot\frac{e\hbar}{m}\bm{B}
\end{equation}
将上式代入$\bigstar$式,得
\begin{equation}
i\hbar\frac{\partial}{\partial t}\phi = \left[\frac{1}{2m}\left(\bm{p} +\frac{e}{c}\bm{A}\right)^{2} + \frac{e\hbar}{2mc}\bm\sigma\cdot\bm{B} -e\bm\phi \right]\phi
\end{equation}
与非相对论极限下的Klein-Gordon方程相比,内禀自旋不再作为一个假定,它已经包含在方程之中,引入了玻耳兹子$\displaystyle \mu_{B}=\frac{e\hbar}{2mc}$。
~\\
~\\
\textbf{4 \quad }\\
~\\
~\\
\textbf{5 \quad }\\
~\\
~\\
\textbf{6 \quad }\\


%%%%%%%%%%%%%%%%%%%%%%%%%%%%%%%%%%%%%%%%%%%%%%%%%%
%%%%%%%%%%%%%%%%%%%% 2013年期末题 %%%%%%%%%%%%%%%%%%%%%%%
%%%%%%%%%%%%%%%%%%%%%%%%%%%%%%%%%%%%%%%%%%%%%%%%%%
\newpage
\subsection{2013年习题}
~\\
~\\
\textbf{1 \quad }\\
~\\
~\\
\textbf{2 \quad }\\
~\\
~\\
\textbf{3 \quad }\\
~\\
~\\
\textbf{4 \quad }\\
~\\
~\\
\textbf{5 \quad }\\
~\\
~\\
\textbf{6 \quad }\\

%%%%%%%%%%%%%%%%%%%%%%%%%%%%%%%%%%%%%%%%%%%%%%%%%%
%%%%%%%%%%%%%%%%%%%% 2014年期末题 %%%%%%%%%%%%%%%%%%%%%%%
%%%%%%%%%%%%%%%%%%%%%%%%%%%%%%%%%%%%%%%%%%%%%%%%%%
\newpage
\subsection{2014年习题}
~\\
~\\
\textbf{1 \quad }\\
~\\
~\\
\textbf{2 \quad }\\
~\\
~\\
\textbf{3 \quad }\\
~\\
~\\
\textbf{4 \quad }\\
~\\
~\\
\textbf{5 \quad }\\
~\\
~\\
\textbf{6 \quad 考虑量子比特$1, 2$构成的复合系统,计算基矢取为$\left|00\right>,\left|01\right>,\left|10\right>,\left|11\right>$。设该系统的密度算符为
\begin{equation}\nonumber 		% \nonumber使该公式不编号
\rho = a\left|\phi^{+}\right>\left<\phi^{+}\right| + (1-a)\left|11\right>\left<11\right| \quad (0 \le a \le 1)
\end{equation}
其中$\displaystyle \left|\phi^{+}\right>=\frac{1}{\sqrt{2}}\left(\left|00\right>+\left|11\right>\right)$。\\
(1) 给出$\rho$、反转算符$\tilde\rho$和$\rho\tilde\rho$的矩阵表示;\\
(2) 计算量子比特$1$的约化密度矩阵$\rho_{1}$;\\
(3) 计算$\rho$的并发度$\mathbb{C}$,并对结果进行讨论。		% \mathbb{C}输出结果为空心的C
}\\
~\\
解:\\
(1) $\tilde{\rho} = (\sigma_{y_{1}}\otimes\sigma_{y_{2}})\rho^{*}(\sigma_{y_{1}}\otimes\sigma_{y_{2}})$
\begin{equation}\nonumber 		% \nonumber使该公式不编号
\rho = \begin{bmatrix} \displaystyle \frac{a}{2} & 0 & 0 & \displaystyle \frac{a}{2} \\ 0 & 0 & 0 & 0 \\ 0 & 0 & 0 & 0 \\ \displaystyle \frac{a}{2} & 0 & 0 & \displaystyle 1-\frac{a}{2} \end{bmatrix}
\end{equation}
\begin{align*}
\tilde{\rho} &= \begin{bmatrix} \quad & \quad & \quad & -1 \\ \quad & \quad & 1 &\quad \\ \quad & 1 & \quad & \quad \\ -1 & \quad & \quad & \quad \end{bmatrix} \begin{bmatrix} \displaystyle \frac{a}{2} & 0 & 0 & \displaystyle \frac{a}{2} \\ 0 & 0 & 0 & 0 \\ 0 & 0 & 0 & 0 \\ \displaystyle \frac{a}{2} & 0 & 0 & \displaystyle 1-\frac{a}{2} \end{bmatrix} \begin{bmatrix} \quad & \quad & \quad & -1 \\ \quad & \quad & 1 &\quad \\ \quad & 1 & \quad & \quad \\ -1 & \quad & \quad & \quad \end{bmatrix} \\
\therefore \tilde{\rho} &= \begin{bmatrix} \displaystyle 1-\frac{a}{2} & 0 & 0 & \displaystyle \frac{a}{2} \\ 0 & 0 & 0 & 0 \\ 0 & 0 & 0 & 0 \\ \displaystyle \frac{a}{2} & 0 & 0 & \displaystyle \frac{a}{2} \end{bmatrix}
\end{align*}
\begin{equation}\nonumber 		% \nonumber使该公式不编号
\therefore \rho\tilde\rho = \begin{bmatrix} \displaystyle \frac{a}{2} & 0 & 0 & \displaystyle \frac{a}{2} \\ 0 & 0 & 0 & 0 \\ 0 & 0 & 0 & 0 \\ \displaystyle \frac{a}{2} & 0 & 0 & \displaystyle 1-\frac{a}{2} \end{bmatrix}  \begin{bmatrix} \displaystyle 1-\frac{a}{2} & 0 & 0 & \displaystyle \frac{a}{2} \\ 0 & 0 & 0 & 0 \\ 0 & 0 & 0 & 0 \\ \displaystyle \frac{a}{2} & 0 & 0 & \displaystyle \frac{a}{2} \end{bmatrix} = \begin{bmatrix} \displaystyle 1 & 0 & 0 & \displaystyle \frac{a^{2}}{2} \\ 0 & 0 & 0 & 0 \\ 0 & 0 & 0 & 0 \\ \displaystyle a-\frac{a^{2}}{2} & 0 & 0 & \displaystyle 1 \end{bmatrix}
\end{equation}
(2) 
\begin{align*}
\rho_{1} &= Tr_{2}\rho = \sum_{\lambda}\left<\lambda\left|\rho\right|\lambda\right> \\
&= \sum_{\lambda} \left[\frac{a}{2}\left<\lambda|\phi^{*}\right>\left<\phi^{*}|\lambda\right> + (1-a)\delta_{1\lambda}\delta_{\lambda1}\right] \\
&= \begin{bmatrix} \displaystyle \frac{a}{2} & \displaystyle \frac{a}{2} \\ \displaystyle \frac{a}{2} & \displaystyle 1-\frac{a}{2} \end{bmatrix}
\end{align*}
(3) $\rho\tilde\rho$ 的本征值为$\displaystyle \lambda_{1}=1,\lambda_{2}=1,\lambda_{3}=0,\lambda_{4}=0$ \\
$\therefore$
\begin{equation}
\mathbb{C} = \max\left\{0,0\right\} = 0
\end{equation}
$\mathbb{C} = 0$,说明体系无纠缠。

%%%%%%%%%%%%%%%%%%%%%%%%%%%%%%%%%%%%%%%%%%%%%%%%%%
%%%%%%%%%%%%%%%%%%%% 2015年期末题 %%%%%%%%%%%%%%%%%%%%%%%
%%%%%%%%%%%%%%%%%%%%%%%%%%%%%%%%%%%%%%%%%%%%%%%%%%
\newpage
\subsection{2015年习题}
~\\
~\\
\textbf{1 \\
(1) 对于标量波函数$\Psi(\bm{r},t)$,写出其在绕空间任意轴$\bm{n}$的无穷小转动变换下的转动算符$U(\bm{e_{n}},\dif\theta)$的表达式;\\
(2) 若粒子需用矢量波函数$\Psi(\bm{r},t)$描述,推导出此时无穷小转动算符的表达式,并进行讨论。}\\
~\\
解:\\
(1) 
\begin{equation}
U(\bm{e_{n}},\dif\theta) = e^{-\frac{i}{\hbar}\dif\theta\bm{e_{n}}\cdot\bm{L}} = \hat{I}-\frac{i}{\hbar}\dif\theta\bm{e_{n}}\cdot\bm{L}
\end{equation}
(2) 设绕三维空间$\bm{e_{n}}$转动$\dif\theta$得变换矩阵为$R(\bm{e_{n}},\dif\theta)$ \\
对矢量波函数有
\begin{equation}
\psi_{k}(x'_{j},t) = R_{ki}\psi_{i}(x_{j},t) = \psi_{k}(x_{j},t) + \dif\theta n_{i}\psi_{l}(x_{j},t)\epsilon_{ilk}
\end{equation}
$\therefore$
\begin{equation}
U(\bm{e_{n}},\dif\theta) = \hat{I}_{3\times3}-\frac{i}{\hbar}\dif\theta\left(\hat{I}_{3\times3}\hat{L}_{z}+\begin{bmatrix} 0 & -i\hbar & 0 \\ i\hbar & 0 & 0 \\ 0 & 0 & 0 \end{bmatrix}\right)
\end{equation}
~\\
~\\
\textbf{2 \quad 试利用一阶张量投影定理计算电子自旋算符$\bm{S}$在总角动量本征态
\begin{equation}\nonumber 		% \nonumber使该公式不编号
\left|jm\right> = \sum_{m_{l}m_{s}}C^{jm}_{lm_{l}\frac{1}{2}m_{s}}\left|lm_{l}\right>\left|\frac{1}{2}m_{s}\right>
\end{equation}
上的平均值。}\\
~\\
解:将$\bm{S}$在球基矢上进行投影,得$\displaystyle \bm{S}=\sum^{1}_{\mu=-1}\left<jm\left|\hat{S}_{\mu}\right|jm\right>\bm\xi^{*}_{\mu}$,由Wignet-Eckart定理
\begin{equation}
\left<jm\left|\hat{S}_{\mu}\right|jm\right> = C^{jm}_{jm1\mu}\left<j\left\|\hat{S}_{1}\right\|j\right>
\end{equation}
只能取$\mu=0$,即有
\begin{equation}
\left<jm\left|\bm{S}\right|jm\right> = \left<jm\left|\hat{S}_{0}\right|jm\right>\bm\xi^{*}_{0} = \left<jm\left|\hat{S}_{0}\right|jm\right>\bm{e}_{z}
\end{equation}
$\therefore$
\begin{equation}
\left<jm\left|\hat{S}_{0}\right|jm\right> = C^{jm}_{jm10}\left<j\left\|\hat{S}_{1}\right\|j\right>
\end{equation}
球基矢下$\hat{S}_{0}=\hat{S}_{z}$,进而有
\begin{align*}
\left<jm\left|\hat{S}_{0}\right|jm\right> &= \left<jm\left|\hat{S}_{z}\right|jm\right> = \frac{\left<jm\left|\hat{J}_{z}\right|jm\right>\left<jm\left|\hat{J}\cdot\hat{S}\right|jm\right>}{j(j+1)\hbar^{2}} \\
&= \frac{m\left<jm\left|\hat{J}\cdot\hat{S}\right|jm\right>}{j(j+1)\hbar^{2}} \\
&= \frac{m\left<jm\left| \displaystyle \frac{1}{2}\left(\hat{J}^{2}-\hat{L}^{2}+\hat{S}^{2}\right) \right|jm\right>}{j(j+1)\hbar^{2}} \\
&= \frac{m\left[j(j+1)-l(l+1)+\displaystyle \frac{3}{4}\right]\hbar}{2j(j+1)}
\end{align*}
$\therefore$
\begin{equation}
\left<jm\left|\bm{S}\right|jm\right> = \frac{m\left[j(j+1)-l(l+1)+\displaystyle \frac{3}{4}\right]\hbar}{2j(j+1)}\bm{e}_{z}
\end{equation}
~\\
~\\
\textbf{3 \quad }\\
~\\
~\\
\textbf{4 \quad 试从量子力学的算符假定和相对论的能量-动量关系出发,导出$Klein-Gordon$方程,写出其相对论协变形式。进而写出有电磁势情况下的$K-G$方程,并推导其非相对论极限。
}\\
~\\
解:从含时薛定谔方程出发,将$\displaystyle E=\frac{p^{2}}{2m}$算符化,有
\begin{equation}\nonumber 		% \nonumber使该公式不编号
E \rightarrow i\hbar\frac{\partial}{\partial t} \quad \bm{p} \rightarrow -i\hbar\nabla
\end{equation}
将上述关系代入相对论情形
\begin{equation}
E^{2}=c^{2}p^{2}+m^{2}c^{4} \Rightarrow -\hbar^{2}\frac{\partial^{2}}{\partial^{2} t}\bm\Psi\left(\bm{r},t\right) = \left(-\hbar^{2}c^{2}\nabla^{2}+m^{2}c^{4}\right)\bm\Psi\left(\bm{r},t\right)
\end{equation}\nonumber 		% \nonumber使该公式不编号
$\therefore$ Klein-Gordon方程为
\begin{equation}
-\hbar^{2}\frac{\partial^{2}}{\partial^{2} t}\bm\Psi\left(\bm{r},t\right) = \left(-\hbar^{2}c^{2}\nabla^{2}+m^{2}c^{4}\right)\bm\Psi\left(\bm{r},t\right)
\end{equation}
相应协变形式为
\begin{equation}
(\Box - \kappa^{2})\bm\Psi\left(\bm{r},t\right) = 0 \qquad \Box=\frac{\partial^{2}}{\partial x_{\mu}\partial x_{\mu}},\kappa=\frac{mc}{\hbar}
\end{equation}
$\therefore$ 有电磁场$\left(\bm{A},\bm\phi\right)$情况下的Klein-Gordon方程为
\begin{equation}
\left(i\hbar\frac{\partial}{\partial t}-q\bm\phi\right)^{2} \bm\Psi\left(\bm{r},t\right) = \left[c^{2}\left(\bm{p}-\frac{q}{c}\bm{A}\right)^{2}+m^{2}c^{4}\right] \bm\Psi\left(\bm{r},t\right)
\end{equation}
在非相对论极限下,令$\displaystyle \bm\Psi\left(\bm{r},t\right) = \psi\left(\bm{r},t\right)e^{-\frac{i}{\hbar}mc^{2}t}$,代入上式得
\begin{equation}
i\hbar\frac{\partial}{\partial t}\psi\left(\bm{r},t\right) = \left[\frac{1}{2m}\left( \bm{p}-\frac{q}{c}\bm{A} \right)^{2}+q\bm\phi\right]\psi\left(\bm{r},t\right)
\end{equation}
以上为非相对论极限下的Klein-Gordon方程。
~\\
~\\
\textbf{5 \quad 已知作一维简谐振动的粒子从时刻运动到时刻的作用量为
\begin{equation}\nonumber 		% \nonumber使该公式不编号
S(xt,x't') = \exp\left\{ \frac{i}{\hbar}\frac{m\omega}{2\sin(\omega T)} \left[ \left(x^{2}+x'^{2}\right)\cos(\omega T)-2xx' \right] \right\}
\end{equation}
式中$m$和$\omega$分别为粒子的质量和振动频率,$T=t-t'\ge0$。若时刻该粒子处于坐标算符的本征态上,求时刻在处发现该粒子的概率。}\\
~\\
解:一维简谐子传播子为
\begin{equation}\nonumber 		% \nonumber使该公式不编号
K(xt,x't') = \left[ \frac{m\omega}{2\pi i\hbar\sin(\omega T)} \right]^{1/2} \exp\left\{ \frac{i}{\hbar}\frac{m\omega}{2\sin(\omega T)} \left[ \left(x^{2}+x'^{2}\right)\cos(\omega T)-2xx' \right] \right\}
\end{equation}
$\therefore$
\begin{align*}
P(x,t) &= \psi^{*}(x,t)\psi(x,t) = K^{*}(xt,x't')K(xt,x't') \\
&= \frac{m\omega}{2\pi\hbar\sin(\omega T)}
\end{align*}
~\\
~\\
\textbf{6 \quad 两量子比特A, B构成复合体系,其计算基矢为$\left|00\right>,\left|01\right>,\left|10\right>,\left|11\right>$。设有密度矩阵
\begin{equation}\nonumber 		% \nonumber使该公式不编号
\rho = \begin{bmatrix} \rho_{11} & \rho_{12} & \rho_{13} & \rho_{14} \\ \rho_{21} & \rho_{22} & \rho_{23} & \rho_{24} \\ \rho_{31} & \rho_{32} & \rho_{33} & \rho_{34} \\ \rho_{41} & \rho_{42} & \rho_{43} & \rho_{44} \end{bmatrix}
\end{equation}
(1) 讨论$\rho_{ij}$应满足的条件;\\
(2) 按定义,求的反转矩阵;\\
(3) 对X型密度矩阵$\rho_{X}$,导出其并发度$\mathbb{C}$的表达式;\\
(4) 求约化密度矩阵$\rho_{A}=Tr_{B}\rho_{X}$。}\\
~\\
解:\\
(1) 
\begin{align*}
\rho_{ij} &= \left<i\left|\rho\right|j\right> = \left<i|\Psi_{n}(t)\right>\left<\Psi_{n}(t)|j\right> \\
&= C_{i}(t)C^{*}_{j}(t)
\end{align*}
其中$\displaystyle \left|\Psi_{n}(t)\right> = \sum_{n}C_{n}(t)\left|n\right>$,若$\displaystyle \left|\Psi_{n}(t)\right> = \left|k\right>$,则有
\begin{equation}
\rho_{ij} = \left<i|k\right>\left<k|j\right> =\delta_{ik}\delta_{kj}
\end{equation}
(2) 
\begin{align*}
\tilde{\rho} &= \begin{bmatrix} \quad & \quad & \quad & -1 \\ \quad & \quad & 1 &\quad \\ \quad & 1 & \quad & \quad \\ -1 & \quad & \quad & \quad \end{bmatrix}   \begin{bmatrix} \rho_{11} & \rho_{12} & \rho_{13} & \rho_{14} \\ \rho_{21} & \rho_{22} & \rho_{23} & \rho_{24} \\ \rho_{31} & \rho_{32} & \rho_{33} & \rho_{34} \\ \rho_{41} & \rho_{42} & \rho_{43} & \rho_{44} \end{bmatrix}   \begin{bmatrix} \quad & \quad & \quad & -1 \\ \quad & \quad & 1 &\quad \\ \quad & 1 & \quad & \quad \\ -1 & \quad & \quad & \quad \end{bmatrix} \\
&= \begin{bmatrix} \rho_{44} & -\rho_{43} & -\rho_{42} & \rho_{41} \\ -\rho_{34} & \rho_{33} & \rho_{32} & -\rho_{31} \\ -\rho_{24} & \rho_{23} & \rho_{22} & -\rho_{21} \\ \rho_{14} & -\rho_{13} & -\rho_{12} & \rho_{11} \end{bmatrix}
\end{align*}
(3) 与(2)同理
\begin{equation}
\tilde{\rho}_{X} = \begin{bmatrix} \rho_{44} & 0 & 0 & \rho_{41} \\ 0 & \rho_{33} & \rho_{32} & 0 \\ 0 & \rho_{23} & \rho_{22} & 0 \\ \rho_{14} & 0 & 0 & \rho_{11} \end{bmatrix}
\end{equation}
$\therefore$
\begin{equation}
{\rho}_{X}\tilde{\rho}_{X} = \begin{bmatrix} \rho_{11} & 0 & 0 & \rho_{14} \\ 0 & \rho_{22} & \rho_{23} & 0 \\ 0 & \rho_{32} & \rho_{33} & 0 \\ \rho_{41} & 0 & 0 & \rho_{44} \end{bmatrix}  \begin{bmatrix} \rho_{44} & 0 & 0 & \rho_{41} \\ 0 & \rho_{33} & \rho_{32} & 0 \\ 0 & \rho_{23} & \rho_{22} & 0 \\ \rho_{14} & 0 & 0 & \rho_{11} \end{bmatrix}
\end{equation}
${\rho}_{X}\tilde{\rho}_{X}$对应四个顺次降序排列的本征值$\lambda_{1}$~$\lambda_{4}$,可得\\
\begin{equation}
\mathbb{C} = \max\left\{0,\sqrt{\lambda_{1}}-\sqrt{\lambda_{2}}-\sqrt{\lambda_{3}}-\sqrt{\lambda_{4}}\right\}
\end{equation}
代入$\lambda_{1}$~$\lambda_{4}$即可。\\
(4) 略。
%%%%%%%%%%%%%%%%%%%%%%%%%%%%%%%%%%%%%%%%%%%%%%%%%%
%%%%%%%%%%%%%%%%%%%% 2016年期末题 %%%%%%%%%%%%%%%%%%%%%%%
%%%%%%%%%%%%%%%%%%%%%%%%%%%%%%%%%%%%%%%%%%%%%%%%%%
\newpage
\subsection{2016年习题}
~\\
~\\
\textbf{1 \quad \\
(1) 写出时间平移算符$U(\tau)$和时间反演算符$\hat{T}$对波函数$\Psi(\bm{r},t)$的作用关系式;\\
(2) 论证时间反演算符$\hat{T}$一定是反幺正算符。}\\
~\\
解:\\
(1) 
\begin{equation}
U(\tau)\Psi(\bm{r},t) = \Psi(\bm{r},t+\tau) \quad \hat{T}\Psi(\bm{r},t) = -\Psi^{*}(\bm{r},-t)
\end{equation}
(2) 
$\because$
\begin{equation}\nonumber 		% \nonumber使该公式不编号
\hat{T}e^{-\frac{i}{\hbar}\tau\hat{H}}\Psi(\bm{r},t) = \hat{T}\Psi(\bm{r},t+\tau) = \Psi^{*}(\bm{r},-t+\tau)
\end{equation}
又$\because$
\begin{equation}\nonumber 		% \nonumber使该公式不编号
e^{\frac{i}{\hbar}\tau\hat{H}}\hat{T}\Psi(\bm{r},t) = \Psi^{*}(\bm{r},-(t-\tau)) = \Psi^{*}(\bm{r},-t+\tau)
\end{equation}
$\therefore$
\begin{equation}
\hat{T}e^{-\frac{i}{\hbar}\tau\hat{H}}\Psi(\bm{r},t) = e^{\frac{i}{\hbar}\tau\hat{H}}\hat{T}\Psi(\bm{r},t)
\end{equation}
若$\tau \ll 1$,$\therefore$
\begin{equation}
\hat{T}\left(1-\frac{i}{\hbar}\tau\hat{H}\right) = \left(1+\frac{i}{\hbar}\tau\hat{H}\right)\hat{T}
\end{equation}
哈密顿量$\hat{H}$应保持在时间反演变换下不变,则时间反演算符$\hat{T}$一定是反幺正算符,即有
\begin{equation}
\hat{T}\hat{H}\hat{T}^{-1} = \hat{H}
\end{equation}
~\\
~\\
\textbf{2 \quad 引入函数$\displaystyle \Psi_{jLJM_{J}}(\tau) = \sum_{mM} C_{jmLM}^{JM_{J}}\hat{T}_{LM}(\tau)\psi_{jLJM_{J}}(\tau)$,其中为$C-G$系数,$\displaystyle \psi_{jm}(\tau)$为角动量本征函数,$\hat{T}_{LM}(\tau)$为$L$阶不可约张量算符。证明如此定义的$\displaystyle \Psi_{jLJM_{J}}(\tau)$是角动量$z$分量$\hat{J}_{z}$的本征函数,进而计算$\displaystyle \hat{J}_{\pm}\Psi_{jLJM_{J}}(\tau)$。}\\
~\\
解:由不可约张量算符的拉卡定义得\\
\begin{equation}\nonumber 		% \nonumber使该公式不编号
\left[ \hat{J}_{z},\hat{T}_{LM}(\tau) \right] = M\hbar\hat{T}_{LM}(\tau)
\end{equation}
$\therefore$
\begin{align*}
\hat{J}_{z}\Psi_{jLJM_{J}}(\tau) &= \sum_{mM} C_{jmLM}^{JM_{J}}\hat{J}_{z}\hat{T}_{LM}(\tau)\psi_{jm}(\tau) \\
&= \sum_{mM} C_{jmLM}^{JM_{J}} \left( \hat{T}_{LM}(\tau)\hat{J}_{z}+M\hbar\hat{T}_{LM}(\tau) \right)\psi_{jm}(\tau) \\
&= \sum_{mM} C_{jmLM}^{JM_{J}}(m+M)\hbar\hat{T}_{LM}(\tau)\psi_{jm}(\tau) \\
&= M_{J}\hbar\sum_{mM} C_{jmLM}^{JM_{J}}\hat{T}_{LM}(\tau)\psi_{jm}(\tau)
\end{align*}
$\therefore \hat{J}_{z}\Psi_{jLJM_{J}}(\tau) = M_{J}\hbar\Psi_{jLJM_{J}}(\tau)$,$\displaystyle \Psi_{jLJM_{J}}(\tau)$是角动量$z$分量$\hat{J}_{z}$的本征函数。\\
$\because \displaystyle \hat{J}_{\pm}\psi_{jm}(\tau)=\sqrt{j(j+1)-m(m\pm1)}$,以及不可约张量算符的拉卡定义
\begin{equation}\nonumber 		% \nonumber使该公式不编号
\left[ \hat{J}_{\pm},\hat{T}_{LM}(\tau) \right] = \sqrt{L(L+1)-M(M\pm1)}\hbar\hat{T}_{LM}(\tau)
\end{equation}
$\therefore$
\begin{align*}
\hat{J}_{\pm}\Psi_{jLJM_{J}}(\tau) &= \sum_{mM} C_{jmLM}^{JM_{J}}\hat{J}_{\pm}\hat{T}_{LM}(\tau)\psi_{jm}(\tau) \\
&= \sum_{mM} C_{jmLM}^{JM_{J}} \left( \hat{T}_{LM}(\tau)\hat{J}_{\pm}+\sqrt{L(L+1)-M(M\pm1)}\hbar\hat{T}_{LM}(\tau) \right)\psi_{jm}(\tau) \\
&= \left[\sqrt{L(L+1)-M(M\pm1)} + \sqrt{j(j+1)-m(m\pm1)}\right] \sum_{mM} C_{jmLM}^{JM_{J}}\cdot \\ &\hat{T}_{LM}(\tau)\psi_{jm}(\tau) \\
&= \left[\sqrt{L(L+1)-M(M\pm1)} + \sqrt{j(j+1)-m(m\pm1)}\right] \Psi_{jLJM_{J}}(\tau)
\end{align*}
~\\
~\\


\textbf{3 \quad \\
(1) 证明:若$\psi(1,2)$为交换对称波函数,则将其反对称化,结果一定为零;而若$\psi(1,2)$为交换反对称波函数,则将其对称化,结果也一定为零。\\
(2) 设$n$个全同玻色子占据同一个单粒子态$\mu$,在粒子数表象下,写出体系态矢$\left|\Psi(1,2,\cdot\cdot\cdot,n)\right>$的表达式,求出单体算符$\hat{T}$和两体算符$\hat{V}$在这个态矢上的平均值。
}\\
~\\
解:\\
(1) 若交换对称波函数$\psi(1,2)$
\begin{align*}
\psi(1,2) &= \psi(2,1) \\
\Psi(1,2) &= \frac{1}{\sqrt{2}}\left[\psi(1,2) - \psi(2,1)\right] = \frac{1}{\sqrt{2}}\left[\psi(1,2) - \psi(1,2)\right] = 0
\end{align*}
若交换反对称波函数$\psi(1,2)$
\begin{align*}
\psi(1,2) &= -\psi(2,1) \\
\Psi(1,2) &= \frac{1}{\sqrt{2}}\left[\psi(1,2) + \psi(2,1)\right] = \frac{1}{\sqrt{2}}\left[\psi(1,2) - \psi(1,2)\right] = 0
\end{align*}
(2) 
\begin{equation}
\left|\Psi(1,2,\cdot\cdot\cdot,n)\right> = \frac{1}{\sqrt{n!}}(\hat{a}^{\dag})^{n}\left|0\right> = \left|n_{\mu}\right> = \left|n\right>
\end{equation}
$\because \displaystyle \hat{T}=\sum_{\alpha\beta}\left<\alpha\left|t\right|\beta\right>\hat{a}^{\dag}_{\alpha}\hat{a}_{\beta}$,$\displaystyle \hat{V}=\frac{1}{2}\sum_{\alpha\beta\gamma\delta} \left<\alpha\beta\left|t\right|\gamma\delta\right>\hat{a}^{\dag}_{\alpha}\hat{a}_{\beta}^{\dag}\hat{a}_{\gamma}\hat{a}_{\delta}$ \\
$\therefore$
\begin{align*}
\left<\hat{T}\right> &= \sum_{\alpha\beta}\left<\alpha\left|t\right|\beta\right>\left<\Psi(1,2,\cdot\cdot\cdot,n)\right| \hat{a}^{\dag}_{\alpha}\hat{a}_{\beta} \left|\Psi(1,2,\cdot\cdot\cdot,n)\right> \\
&= \sum_{\mu}\left<\mu\left|t\right|\mu\right>\left<\Psi(1,2,\cdot\cdot\cdot,n)\right| \hat{a}^{\dag}_{\mu}\hat{a}_{\mu} \left|\Psi(1,2,\cdot\cdot\cdot,n)\right> \\
&= \left<\mu\left|t\right|\mu\right>\left<n\right| \hat{a}^{\dag}\hat{a} \left|n\right> = \sum_{\mu}\left<\mu\left|t\right|\mu\right>\left<n\right| \hat{a}^{\dag} \sqrt{n}\left|n\right> \\
&= \left<\mu\left|t\right|\mu\right>\left<n\right| \sqrt{n}\sqrt{n-1+1} \left|n\right> \\
&= n\left<\mu\left|t\right|\mu\right>
\end{align*}
同理可得
\begin{equation}
\left<\hat{V}\right> = \frac{1}{2}\left<\mu\mu\left|v\right|\mu\mu\right>n(n-1)
\end{equation}
~\\
~\\
\textbf{4 \quad 写出单电子相对论性运动方程中算符$\bm\alpha$和$\beta$,以及自旋算符$\displaystyle \bm{S}=\frac{\hbar}{2}\bm\Sigma$中的在$Pauli-Dirac$表象中的矩阵表示。若选取$Weyl$表象,即取$\displaystyle \beta=\begin{bmatrix} 0 & I \\ I & 0 \end{bmatrix}$,其中$I$为二阶单位矩阵,试推导出在这一表象中$\bm\alpha$和$\bm\Sigma$的矩阵表示,并计算对易关系。}\\
~\\
解:由2004年第7题可得,在Pauli-Dirac表象中
\begin{equation}
\bm\alpha = \begin{bmatrix} 0 & \bm\sigma \\ \bm\sigma & 0 \end{bmatrix} \quad \beta = \begin{bmatrix} I & 0 \\ 0 & -I \end{bmatrix} 
\end{equation}
若选取Weyl表象,由$\displaystyle \left\{\alpha_{i},\alpha_{j}\right\}=2\delta_{ij}$,$\displaystyle \left\{\alpha_{i},\beta\right\}=0$
$\therefore$
\begin{equation}
\bm\alpha = \begin{bmatrix} -\bm\sigma & 0 \\ 0 & \bm\sigma \end{bmatrix}
\end{equation}
$\because \Sigma_{i}^{2}=1$,以及
\begin{equation}
\left[\Sigma_{i},\Sigma_{j}\right]=i\epsilon_{ijk}\Sigma_{k} \quad \left[\Sigma_{i},\beta\right]=0 \quad \left[\Sigma_{i},\alpha_{j}\right]=2i\epsilon_{ijk}\alpha_{k}
\end{equation}
$\therefore$
\begin{equation}
\bm\Sigma = \begin{bmatrix} \bm\sigma & 0 \\ 0 & \bm\sigma \end{bmatrix}
\end{equation}
$\therefore$ 在Weyl表象下
\begin{equation}
\bm\alpha = \begin{bmatrix} -\bm\sigma & 0 \\ 0 & \bm\sigma \end{bmatrix} \quad \bm\Sigma = \begin{bmatrix} \bm\sigma & 0 \\ 0 & \bm\sigma \end{bmatrix}
\end{equation}
~\\
~\\
\textbf{5 \quad 设一维自由粒子从$t'$时刻运动到$t''$时刻,将时间间隔$T=t''-t'$分为三段$(t''<t_{1}<t_{2}<t')$,其中$t_{1},t_{2}$分别位于$T$的$\displaystyle \frac{1}{2}$处和$\displaystyle \frac{1}{4}$处。试采用多边折线道方案,计算传播子$K(x''t'',x't')$。对每段时间间隔可取$\displaystyle C=\left(\frac{2\pi\hbar i\epsilon}{m}\right)^{-1/2}$,其中$\epsilon$为间隔大小。}\\
~\\
解:
\begin{equation}\nonumber 		% \nonumber使该公式不编号
L=T=\frac{1}{2}m\left(\frac{x_{j}-x_{j-1}}{t_{j}-t_{j-1}}\right)^{2}
\end{equation}
$\therefore$
\begin{equation}\nonumber 		% \nonumber使该公式不编号
S=\int^{t''}_{t'}L\dif t \sum_{i=1}^{3}\frac{1}{2}m\frac{(x_{i}-x_{i-1})^{2}}{\epsilon_{i}^{2}}\cdot\epsilon_{i}
\end{equation}
其中$x_{0}=x'$,$x_{3}=x''$\\
$\therefore$
\begin{equation}
S=\frac{1}{2}m\frac{(x_{1}-x')^{2}}{\epsilon_{1}} + \frac{1}{2}m\frac{(x_{2}-x_{1})^{2}}{\epsilon_{2}} + \frac{1}{2}m\frac{(x_{3}-x_{2})^{2}}{\epsilon_{3}}
\end{equation}
\begin{align*}
K(x''t'',x't') &= \left(\frac{2\pi\hbar i\epsilon_{1}}{m}\right)^{-1/2}\left(\frac{2\pi\hbar i\epsilon_{2}}{m}\right)^{-1/2}\left(\frac{2\pi\hbar i\epsilon_{3}}{m}\right)^{-1/2} \int\dif x_{1}\int\dif x_{2} \\ & \exp\left\{\frac{i}{\hbar}\left[ \frac{1}{2}m\frac{(x_{1}-x')^{2}}{\epsilon_{1}} + \frac{1}{2}m\frac{(x_{2}-x_{1})^{2}}{\epsilon_{2}} + \frac{1}{2}m\frac{(x_{3}-x_{2})^{2}}{\epsilon_{3}} \right] \right\} \\
&= \cdot\cdot\cdot \\
&= \left(\frac{m}{2\pi\hbar iT}\right)^{\frac{1}{2}}\exp\left\{ \frac{i}{\hbar} \frac{m}{2} \frac{(x''-x')^{2}}{T} \right\}
\end{align*}
$\therefore \displaystyle K(x''t'',x't') = \left(\frac{m}{2\pi\hbar iT}\right)^{\frac{1}{2}}\exp\left\{ \frac{i}{\hbar} \frac{m}{2} \frac{(x''-x')^{2}}{T} \right\}$。
~\\
~\\

\textbf{6 \quad 考虑量子比特$1, 2$构成的复合系统,计算基矢取为$\left|00\right>,\left|01\right>,\left|10\right>,\left|11\right>$。设该系统的密度算符为
\begin{equation}\nonumber 		% \nonumber使该公式不编号
\rho = a\left|\phi^{+}\right>\left<\phi^{+}\right| + (1-a)\left|11\right>\left<11\right| \quad (0 \le a \le 1)
\end{equation}
其中$\displaystyle \left|\phi^{+}\right>=\frac{1}{\sqrt{2}}\left(\left|00\right>+\left|11\right>\right)$。\\
(1) 给出$\rho$、反转算符$\tilde\rho$和$\rho\tilde\rho$的矩阵表示;\\
(2) 计算量子比特$1$的约化密度矩阵$\rho_{1}$;\\
(3) 计算$\rho$的并发度$\mathbb{C}$,并对结果进行讨论。		% \mathbb{C}输出结果为空心的C
}\\
~\\
解:\\
(1) $\tilde{\rho} = (\sigma_{y_{1}}\otimes\sigma_{y_{2}})\rho^{*}(\sigma_{y_{1}}\otimes\sigma_{y_{2}})$
\begin{equation}\nonumber 		% \nonumber使该公式不编号
\rho = \begin{bmatrix} \displaystyle \frac{a}{2} & 0 & 0 & \displaystyle \frac{a}{2} \\ 0 & 0 & 0 & 0 \\ 0 & 0 & 0 & 0 \\ \displaystyle \frac{a}{2} & 0 & 0 & \displaystyle 1-\frac{a}{2} \end{bmatrix}
\end{equation}
\begin{align*}
\tilde{\rho} &= \begin{bmatrix} \quad & \quad & \quad & -1 \\ \quad & \quad & 1 &\quad \\ \quad & 1 & \quad & \quad \\ -1 & \quad & \quad & \quad \end{bmatrix} \begin{bmatrix} \displaystyle \frac{a}{2} & 0 & 0 & \displaystyle \frac{a}{2} \\ 0 & 0 & 0 & 0 \\ 0 & 0 & 0 & 0 \\ \displaystyle \frac{a}{2} & 0 & 0 & \displaystyle 1-\frac{a}{2} \end{bmatrix} \begin{bmatrix} \quad & \quad & \quad & -1 \\ \quad & \quad & 1 &\quad \\ \quad & 1 & \quad & \quad \\ -1 & \quad & \quad & \quad \end{bmatrix} \\
\therefore \tilde{\rho} &= \begin{bmatrix} \displaystyle 1-\frac{a}{2} & 0 & 0 & \displaystyle \frac{a}{2} \\ 0 & 0 & 0 & 0 \\ 0 & 0 & 0 & 0 \\ \displaystyle \frac{a}{2} & 0 & 0 & \displaystyle \frac{a}{2} \end{bmatrix}
\end{align*}
\begin{equation}\nonumber 		% \nonumber使该公式不编号
\therefore \rho\tilde\rho = \begin{bmatrix} \displaystyle \frac{a}{2} & 0 & 0 & \displaystyle \frac{a}{2} \\ 0 & 0 & 0 & 0 \\ 0 & 0 & 0 & 0 \\ \displaystyle \frac{a}{2} & 0 & 0 & \displaystyle 1-\frac{a}{2} \end{bmatrix}  \begin{bmatrix} \displaystyle 1-\frac{a}{2} & 0 & 0 & \displaystyle \frac{a}{2} \\ 0 & 0 & 0 & 0 \\ 0 & 0 & 0 & 0 \\ \displaystyle \frac{a}{2} & 0 & 0 & \displaystyle \frac{a}{2} \end{bmatrix} = \begin{bmatrix} \displaystyle 1 & 0 & 0 & \displaystyle \frac{a^{2}}{2} \\ 0 & 0 & 0 & 0 \\ 0 & 0 & 0 & 0 \\ \displaystyle a-\frac{a^{2}}{2} & 0 & 0 & \displaystyle 1 \end{bmatrix}
\end{equation}
(2) 
\begin{align*}
\rho_{1} &= Tr_{2}\rho = \sum_{\lambda}\left<\lambda\left|\rho\right|\lambda\right> \\
&= \sum_{\lambda} \left[\frac{a}{2}\left<\lambda|\phi^{*}\right>\left<\phi^{*}|\lambda\right> + (1-a)\delta_{1\lambda}\delta_{\lambda1}\right] \\
&= \begin{bmatrix} \displaystyle \frac{a}{2} & \displaystyle \frac{a}{2} \\ \displaystyle \frac{a}{2} & \displaystyle 1-\frac{a}{2} \end{bmatrix}
\end{align*}
(3) $\rho\tilde\rho$ 的本征值为$\displaystyle \lambda_{1}=1,\lambda_{2}=1,\lambda_{3}=0,\lambda_{4}=0$ \\
$\therefore$
\begin{equation}
\mathbb{C} = \max\left\{0,0\right\} = 0
\end{equation}
$\mathbb{C} = 0$,说明体系无纠缠。
%%%%%%%%%%%%%%%%%%%%%%%%%%%%%%%%%%%%%%%%%%%%%%%%%%
%%%%%%%%%%%%%%%%%%%% 2017年期末题 %%%%%%%%%%%%%%%%%%%%%%%
%%%%%%%%%%%%%%%%%%%%%%%%%%%%%%%%%%%%%%%%%%%%%%%%%%
\newpage
\subsection{2017年习题}
~\\
~\\
\textbf{1 \quad \\
(1)试证眀角动量平方算符可写为$\displaystyle \bm{J}^{2}=\frac{1}{2}(J_{+}J_{-}+J_{-}J_{+})+J_{z}^{2}$,其中$J_{\pm}=J_{x}\pm iJ_{y}$;(2)利用上面的关系式及不可约张量算符的$Racah$定义,推导对易关系$\displaystyle \left[ \bm{L}^{2}, \bm{T}_{LM}\right]$,这里$\bm{T}_{LM}(\tau)$为$L$阶不可约张量算符。}\\
~\\
解:(1) $\because J_{\pm}=J_{x}\pm iJ_{y} \Rightarrow \left\{ \begin{aligned}
J_{x} &= \frac{1}{2}(J_{+}+J_{-}) \\ J_{y} &= \frac{1}{2i}(J_{+}-J_{-})
\end{aligned} \right. $ \\
$\therefore$
\begin{equation}
\begin{aligned}
\bm{J}^{2} &= J_{x}^{2}+J_{y}^{2}+J_{z}^{2} = \frac{1}{4}\left((J_{+}+J_{-})^{2}-(J_{+}-J_{-})^{2}\right)+J_{z}^{2} \\
&=\frac{1}{4}(2J_{+}J_{-}+2J_{-}J_{+})+J_{z}^{2} = \frac{1}{2}(J_{+}J_{-}+J_{-}J_{+})+J_{z}^{2}
\end{aligned}
\end{equation}
(2) 
\begin{align*}
\left[\bm{L}^{2}, \bm{T}_{LM}\right] &= \left[\frac{1}{2}\left(L_{+}L_{-}+L_{-}L_{+}\right)+L_{z}^{2},\bm{T}_{LM}\right]\\
&= \frac{1}{2} \left(\left[L_{+}L_{-},\bm{T}_{LM}\right]+\left[L_{-}L_{+},\bm{T}_{LM}\right]\right) + \left[L_{z}^{2},\bm{T}_{LM}\right] \\
&= \frac{\hbar}{2}\left[ \sqrt{L(L+1)-M(M-1)}\left\{L_{+},\bm{T}_{L,M-1}\right\} + \sqrt{L(L+1)-M(M+1)}\left\{L_{-},\bm{T}_{L,M+1}\right\} \right] \\ &+ M\hbar\left\{L_{z},\bm{T}_{LM}\right\} 
\end{align*}
~\\
~\\


\textbf{2 \quad $D$函数$\displaystyle D_{m'm}^{j}(\alpha\beta\gamma)$作为欧拉角的函数可表示作自由转动的对称陀螺的本征函数:
\begin{equation}\nonumber 		% \nonumber使该公式不编号
\Psi_{LMK}(\alpha\beta\gamma) = \sqrt{\frac{2L+1}{8\pi^{2}}}D_{MK}^{L}(-\alpha,-\beta,-\gamma)
\end{equation}
在惯量主轴坐标系$O-\xi\eta\zeta$下写出体系的哈密顿算符,讨论力学量完全集,给出其中各力学量的本征值表达式,并讨论能量简并度。}\\
~\\
解:在惯量主轴坐标系$O-\xi\eta\zeta$下体系的哈密顿算符为
\begin{equation}
\hat{H} = \frac{1}{2I}\hat{L}^{2} + \frac{1}{2}\left(\frac{1}{I'}-\frac{1}{I}\right)\hat{L}_{\zeta}^{2} 
\end{equation}
其中$I$、$I'$为转动惯量。\\
$\therefore$
\begin{align*}
\hat{L}^{2} &= -\hbar^{2} \left[ \frac{1}{\sin\beta}\frac{\partial}{\partial\beta}\left(\sin\beta\frac{\partial}{\partial\beta}\right) + \frac{1}{\sin^{2}\beta}\left( \frac{\partial^{2}}{\partial\alpha^{2}} - 2\cos\beta\frac{\partial^{2}}{\partial\alpha\partial\gamma} + \frac{\partial^{2}}{\partial\gamma^{2}} \right) \right] \\
~\\
\hat{L}_{\zeta} &= -i\hbar\frac{\partial}{\partial\gamma}
\end{align*}
力学量完全集为$\left\{\hat{H},\hat{L}^{2},\hat{L}_{\zeta},\hat{L}_{z}\right\}$,对应的本征值和本征方程为
\begin{equation}
\left\{		% 大括号公式的写法
\begin{aligned}
&\hat{H}\Psi_{LMK} = \left[ \frac{1}{2I}L(L+1) +\frac{1}{2}\left(\frac{1}{I'}-\frac{1}{I}\right)K^{2}\hbar^{2} \right]\Psi_{LMK} \\
&\hat{L}^{2}\Psi_{LMK} = L(L+1)\hbar^{2}\Psi_{LMK} \\
&\hat{L}_{\zeta}\Psi_{LMK} = M\hbar\Psi_{LMK} \\
&\hat{L}_{z}\Psi_{LMK} = K\hbar\Psi_{LMK} \\
\end{aligned}
\right.	% 注意这个点不要拉下\right.	
\end{equation}
能量本征值为
\begin{equation}
E_{L|K|} = \frac{1}{2I}L(L+1) +\frac{1}{2}\left(\frac{1}{I'}-\frac{1}{I}\right)K^{2}\hbar^{2}
\end{equation}
能量简并度
\begin{equation}
f_{LK}=
\left\{		% 大括号公式的写法
\begin{aligned}
&2L+1 \quad &k=0 \\
&2(2L+1) \quad &k \ne 0
\end{aligned}
\right.	% 注意这个点不要拉下\right.	
\end{equation}
~\\
~\\


\textbf{3 \quad 给定算符$\hat{a}^{\dag}$,$\hat{a}$,$\hat{n}=\hat{a}^{\dag}\hat{a}$,满足$\displaystyle (\hat{a}^{\dag})^{\dag}=\hat{a}$,$\displaystyle (\hat{a})^{\dag}=\hat{a}^{\dag}$,及$\{\hat{a}^{\dag},\hat{a}^{\dag}\}=0$,$\{\hat{a}^{\dag},\hat{a}\}=1$。\\
(1) 试证$\hat{n}^{2} = \hat{n}$,由此求出的本征值,并回答这些算符适用于描述什么类型的粒子。\\
(2) 在$\hat{n}$的自身表象下,给出$\hat{a}^{\dag}$,$\hat{a}$和$\hat{n}$的矩阵表示。}\\
~\\
解:(1) $\displaystyle \{\hat{a}^{\dag},\hat{a}^{\dag}\}=0 \Rightarrow \hat{a}^{\dag}\hat{a}^{\dag}=0$\\
$\therefore$
\begin{equation}
\hat{n}^{2} = \hat{a}^{\dag}\hat{a}\hat{a}^{\dag}\hat{a} = \hat{a}^{\dag}\left( 1-\hat{a}^{\dag}\hat{a} \right)\hat{a} = \hat{a}^{\dag}\hat{a}-\hat{a}^{\dag}\hat{a}^{\dag}\hat{a}\hat{a} = \hat{a}^{\dag}\hat{a} = \hat{n}
\end{equation}
$\therefore$
\begin{equation}
\hat{n}^{2}\left|n\right> = \hat{n}\left|n\right> \Rightarrow n^{2}\left|n\right> = n\left|n\right> \Rightarrow n=0,1
\end{equation}
显然$\hat{n}$用来描述费米子。\\
(2) 在$\hat{n}$的自身表象下
\begin{align}
\left<m\left|\hat{n}\right|n\right> &= \left<m\left|n\right|n\right>=n\left<m|n\right> =n\delta_{mn} \notag \\
\left<m\left|\hat{a}^{\dag}\right|n\right> &= \left<m\left|\sqrt{1-n}\right|n+1\right> =\sqrt{1-n}\left<m|n+1\right> = \sqrt{1-n}\delta_{m,n+1} \notag \\
\left<m\left|\hat{a}\right|n\right> &= \left<m\left|\sqrt{n}\right|n-1\right> =\sqrt{n}\left<m|n-1\right> = \sqrt{n}\delta_{m,n-1} \notag 
\end{align}

$\therefore$
\begin{equation}\nonumber 		% \nonumber使该公式不编号
\hat{n}=\begin{bmatrix}
0 & 0\\
0 & 1
\end{bmatrix} \quad
\hat{a}^{\dag}=\begin{bmatrix}
0 & 0\\
1 & 0
\end{bmatrix} \quad
\hat{a}=\begin{bmatrix}
0 & 1\\
0 & 0
\end{bmatrix}
\end{equation}
~\\
~\\
\textbf{4 \quad 写出自由粒子的$Dirac$相对论性运动方程,进而考察轨道角动量$\bm{L}$所满足的$Heisenberg$运动方程,由此说明$Dirac$方程所描述的粒子必有内禀自旋。给出相应的自旋算符,指出其本征值和相应的量子数是多少。}\\
解:自由电子$Dirac$方程为
\begin{equation}
i\hbar\frac{\partial}{\partial t}\bm\Psi = \left(c\bm{\alpha \cdot p} +mc^{2}\beta\right)\bm\Psi = \left(-i\hbar c\bm{\alpha \cdot \nabla} +mc^{2}\beta\right)\bm\Psi
\end{equation}
其中$\hat{H} = c\bm{\alpha \cdot p} +mc^{2}\beta$,轨道角动量$\bm{L}$所满足的$Heisenberg$运动方程为
\begin{equation}
\frac{\dif \bm{L}}{\dif t} = \frac{i}{\hbar}\left[ \hat{H} , \bm{L} \right]
\end{equation}
其中
\begin{equation}
\left[ \hat{H} , \bm{L} \right] = \left[ \left( c\bm{\alpha \cdot p} +mc^{2}\beta \right) , \bm{r} \times \bm{p} \right] = -i\hbar c\bm{\alpha} \times \bm{p}
\end{equation}
$\therefore$ 轨道角动量$\bm{L}$所满足的$Heisenberg$运动方程为
\begin{equation}
\frac{\dif \bm{L}}{\dif t} = c\bm{\alpha} \times \bm{p}
\end{equation}
但对于自由粒子而言空间各向同性,理应有角动量守恒,所以除轨道角动量之外粒子必有内禀的自旋角动量。引入$\bm\Sigma$使之满足$\left[\bm\Sigma,\beta\right]=0$且$\left[\Sigma_{i},\alpha_{j}\right]=2i\epsilon_{ijk}\alpha_{k}$。\\
$\therefore$ 
\begin{align}
\left[ \hat{H} , \bm\Sigma \right] &= \left[ \left( c\bm{\alpha \cdot p} +mc^{2}\beta \right) , \bm\Sigma \right] \equiv c\left[ \bm{\alpha \cdot p} , \bm\Sigma \right] \notag \\
&= 2ic\bm{\alpha} \times \bm{p} \notag 
\end{align}
令$\displaystyle \bm{S} = \frac{\hbar}{2}\bm\Sigma$,则有$\displaystyle \left[ \hat{H} , \bm{S} \right] = i\hbar c\bm{\alpha} \times \bm{p}$,使总角动量$\bm{J}$满足$\displaystyle \left[ \hat{H} , \bm{S} \right] = 0$。\\
$\therefore$ 自由电子自旋算符为$\displaystyle \bm{S} = \frac{\hbar}{2}\bm\Sigma$,本征值为$\displaystyle \pm\frac{\hbar}{2}$,对应量子数为$\displaystyle \frac{1}{2}$。
~\\
~\\
\textbf{5 \quad 写出传播子$K(x''t'',x't')$的定义式。设体系哈密顿量$\hat{H}$不显含时间,有本征方程$\hat{H}\psi_{n}(x)=E\psi_{n}(x)$,试导出传播子在能量表象中的表达式,并就$t''=t'=t$的情况进行讨论。}\\
解:由$\hat{H}\psi_{n}(x)=E\psi_{n}(x)$
\begin{equation}
\begin{aligned}
K(x''t'',x't') &= \sum_{n}\left<x''\left|e^{-\frac{i}{\hbar}\hat{H}(t''-t')}\right|\psi_{n}(x)\right> \left<\psi_{n}(x)\Big|x'\right> \\
&= \sum_{n}e^{-\frac{i}{\hbar}E_{n}(t''-t')}\psi_{n}(x'')\psi^{*}_{n}(x') \\
&= \sum_{n}\psi_{n}(x'',t'')\psi^{*}_{n}(x',t')
\end{aligned}
\end{equation}
若$t''=t'=t$,则有
\begin{equation}
K(x''t'',x't') = \sum_{n}\psi_{n}(x'')\psi^{*}_{n}(x') = \delta(x''-x')
\end{equation}
~\\
~\\
\textbf{6 \quad }\\



\end{spacing} 		% 结束行距 
\end{document}  