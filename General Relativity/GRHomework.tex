% XeLaTeX can use any Mac OS X font. See the setromanfont command below.
% Input to XeLaTeX is full Unicode, so Unicode characters can be typed directly into the source.

% The next lines tell TeXShop to typeset with xelatex, and to open and save the source with Unicode encoding.

%!TEX TS-program = xelatex
%!TEX encoding = UTF-8 Unicode

\documentclass[12pt]{article}
\usepackage{geometry}                % See geometry.pdf to learn the layout options. There are lots.
\geometry{letterpaper}                   % ... or a4paper or a5paper or ... 
%\geometry{landscape}                % Activate for for rotated page geometry
%\usepackage[parfill]{parskip}    % Activate to begin paragraphs with an empty line rather than an indent
\usepackage{graphicx}
\usepackage{amssymb}
\usepackage{graphicx}
\usepackage{amsmath}           % 数学符号与特殊字符
\usepackage{ragged2e}          % 两端对齐
\usepackage{color}             	% 字体颜色
\usepackage{indentfirst}		% 首行缩进
\geometry{left=1.5cm,right=2.5cm,top=2.0cm,bottom=2.5cm}
% Will Robertson's fontspec.sty can be used to simplify font choices.
% To experiment, open /Applications/Font Book to examine the fonts provided on Mac OS X,
% and change "Hoefler Text" to any of these choices.

\usepackage{fontspec,xltxtra,xunicode}
\defaultfontfeatures{Mapping=tex-text}
\setromanfont[Mapping=tex-text]{Hoefler Text}
\setsansfont[Scale=MatchLowercase,Mapping=tex-text]{Gill Sans}
\setmonofont[Scale=MatchLowercase]{Andale Mono}

\title{\textbf{General Relativity Homework}}
\author{\emph{Tianxiao Liang} \quad 2018322003}
%\date{}                                           % Activate to display a given date or no date

\begin{document}
\maketitle
\setlength{\parindent}{0pt}	% 取消首行缩进,如果需要缩进,则将0pt设置为所需数量的pt,一般为2pt
% For many users, the previous commands will be enough.
% If you want to directly input Unicode, add an Input Menu or Keyboard to the menu bar 
% using the International Panel in System Preferences.
% Unicode must be typeset using a font containing the appropriate characters.
% Remove the comment signs below for examples.

% \newfontfamily{\A}{Geeza Pro}
% \newfontfamily{\H}[Scale=0.9]{Lucida Grande}
% \newfontfamily{\J}[Scale=0.85]{Osaka}
\newcommand*{\dif}{\mathop{}\!\mathrm{d}} 		% \mathop{}用来输出直立黑体的微分算子,如\mathop{d},为了简略我们使用\dif代替\mathop{d}

% 第一题
\textbf{1. Prove the torsion $\displaystyle\Gamma^{\lambda}_{[\mu\nu]}$ (the anti-symmetric part of an affine connection) is a tensor.} \\		% \top是转置符号
~\\
\textbf{Proof\,:} 	% \widetilde用来在字母上加~
\begin{equation}
\begin{aligned}
 \displaystyle \widetilde\Gamma^{\lambda}_{[\mu\nu]} &= \frac{1}{2}(\widetilde\Gamma^{\lambda}_{\mu\nu}-\widetilde\Gamma^{\lambda}_{\nu\mu}) \\
&= \frac{1}{2}(\Gamma^{\rho}_{\alpha\sigma}-\Gamma^{\rho}_{\sigma\alpha}) \frac{\partial x^{\alpha}}{\partial\widetilde x^{\mu}} \frac{\partial x^{\sigma}}{\partial \widetilde x^{\nu}} \frac{\partial\widetilde x^{\lambda}}{\partial x^{\rho}}+\frac{1}{2}(\frac{\partial^{2}x^{\rho}}{\partial\widetilde x^{\mu}\partial\widetilde x^{\nu}}\frac{\partial\widetilde x^{\lambda}}{\partial x^{\rho}} - \frac{\partial^{2}x^{\rho}}{\partial\widetilde x^{\nu}\partial\widetilde x^{\mu}}\frac{\partial\widetilde x^{\lambda}}{\partial x^{\rho}})\\
&= \Gamma^{\rho}_{[\alpha\sigma]}\frac{\partial x^{\alpha}}{\partial\widetilde x^{\mu}} \frac{\partial x^{\sigma}}{\partial \widetilde x^{\nu}} \frac{\partial\widetilde x^{\lambda}}{\partial x^{\rho}}\\
\end{aligned}
	\tag{1.1}
\end{equation}
From Equ.1.1 we can prove that the torsion $\displaystyle\Gamma^{\lambda}_{[\mu\nu]}$ is a tensor.\\
~\\
Q.E.D.
~\\
~\\
~\\
% 第二题
\textbf{2. We know that $\Phi$ is a scalar. Please verify that $\displaystyle A_{\mu}=\frac{\partial\Phi}{\partial x^{\mu}}$ is a covariant vector.} \\		
~\\
\textbf{Proof\,:} \begin{equation} 		% \widetilde用来在字母上加~
\displaystyle \widetilde{A}_{\mu}=\frac{\partial\Phi}{\partial\widetilde{x}^{\mu}}, A_{\nu}=\frac{\partial\Phi}{\partial x^{\nu}}
	\tag{2.1}
\end{equation}
then we have
\begin{equation}
\displaystyle \widetilde{A}_{\mu}=\frac{\partial\Phi}{\partial x^{\nu}}\frac{\partial x^{\nu}}{\partial\widetilde{x}^{\mu}}=A_{\nu}\frac{\partial x^{\nu}}{\partial\widetilde{x}^{\mu}}
	\tag{2.2}
\end{equation}
So $\displaystyle A_{\mu}=\frac{\partial\Phi}{\partial x^{\mu}}$ is a covariant vector.\\
~\\
Q.E.D.
~\\
~\\
~\\
% 第三题
\textbf{3. Prove $T^{\mu\nu}A_{\mu\nu}=0$ when $T^{\mu\nu}$ is a symmetric tensor and $A_{\mu\nu}$ is an antisymmetric tensor.} \\		
~\\
\textbf{Proof\,:} 
\begin{equation}
\displaystyle T^{\mu\nu}A_{\mu\nu}=T^{\nu\mu}A_{\nu\mu}=T^{\mu\nu}(-A_{\mu\nu})=-T^{\mu\nu}A_{\mu\nu}
	\tag{3.1}
\end{equation}
then we can get 
\begin{equation}
\displaystyle T^{\mu\nu}A_{\mu\nu}=-T^{\mu\nu}A_{\mu\nu}
	\tag{3.2}
\end{equation}
So we can say $T^{\mu\nu}A_{\mu\nu}=0$.\\
~\\
Q.E.D.
~\\
~\\
~\\
% 第四题
\textbf{4. Known that $g$ is the metric of $g_{\mu\nu}$. Try to verify that $\displaystyle \Gamma^{\mu}_{\alpha\mu}=\frac{1}{2}g^{\mu\nu}g_{\mu\nu,a}=\frac{\partial}{\partial x^{\alpha}}(\ln\sqrt{-g})$.} \\			% log和ln用\log和\ln表示	
~\\
\textbf{Proof\,:}
\begin{equation}
\begin{aligned}
\displaystyle 
\Gamma^{\mu}_{\alpha\beta} &= \frac{1}{2}g^{\mu\nu}(g_{\alpha\nu,\mu}+g_{\nu\mu,\alpha}-g_{\alpha\mu,\nu})\\
&=\frac{1}{2}g^{\mu\nu}g_{\mu\nu,\alpha}
\end{aligned}
	\tag{4.1}
\end{equation}
when
\begin{equation}
\displaystyle 
\frac{\partial}{\partial x^{\alpha}}(\ln\sqrt{-g})=-\frac{1}{\sqrt{-g}} \cdot \frac{1}{2}(-g)^{-1/2}\frac{\partial g}{\partial x^{\alpha}}=\frac{1}{2g}\frac{\partial g}{\partial x^{\alpha}}
	\tag{4.2}
\end{equation}
And because of the equation $\dif g=g \cdot g^{\mu\nu}\dif g^{\mu\nu}$, we can get $\displaystyle \frac{\partial g}{\partial x^{\alpha}}=g \cdot g^{\mu\nu}\frac{\partial g_{\mu\nu}}{\partial x^{\alpha}}$.\\
So we have $\displaystyle \Gamma^{\mu}_{\alpha\mu}=\frac{1}{2}g^{\mu\nu}g_{\mu\nu,a}=\frac{\partial}{\partial x^{\alpha}}(\ln\sqrt{-g})$.\\
~\\
Q.E.D.
~\\
~\\
~\\
% 第五题
\textbf{5. Known that $\displaystyle A_{\mu;\nu}=A_{\mu,\nu}-\Gamma^{\lambda}_{\mu\nu}A_{\lambda}$. Use coordinate differential relation $\displaystyle U_{;\mu}=U_{,\mu}$ and Leibnitzs law to prove $B^{\mu}_{;\nu}=B^{\mu}_{,\nu}+\Gamma^{\mu}_{\lambda\nu}B^{\lambda}$.} \\		
~\\
\textbf{Proof\,:}
From Leibnitzs law, we have
\begin{equation}
\displaystyle 
(A_{\mu}B^{\mu})_{;\nu}=A_{\mu;\nu}B^{\mu}+A_{\mu}B^{\mu}_{;\nu}, (A_{\mu}B^{\mu})_{,\nu}=A_{\mu,\nu}B^{\mu}+A_{\mu}B^{\mu}_{,\nu}
	\tag{5.1}
\end{equation}
while $\displaystyle U_{;\mu}=U_{,\mu}$ , we have
\begin{equation}
\displaystyle 
A_{\mu;\nu}B^{\mu}+A_{\mu}B^{\mu}_{;\nu}=A_{\mu,\nu}B^{\mu}+A_{\mu}B^{\mu}_{,\nu}
	\tag{5.2}
\end{equation}
Then substitute $\displaystyle A_{\mu;\nu}=A_{\mu,\nu}-\Gamma^{\lambda}_{\mu\nu}A_{\lambda}$ into Equ.5.2, we'll have
\begin{equation}
\displaystyle 
(A_{\mu;\nu}-\Gamma^{\lambda}_{\mu\nu}A_{\lambda})B^{\mu}+A_{\mu}B^{\mu}_{;\nu}=A_{\mu,\nu}B^{\mu}+A_{\mu}B^{\mu}_{,\nu}
	\tag{5.3}
\end{equation}
The formula can be obtained as
\begin{equation}
\displaystyle 
A_{\mu}B^{\mu}_{;\nu}=A_{\mu}B^{\mu}+A_{\mu}\Gamma^{\mu}_{\sigma\nu}B^{\sigma}
	\tag{5.4}
\end{equation}
it equals to $B^{\mu}_{;\nu}=B^{\mu}_{,\nu}+\Gamma^{\mu}_{\lambda\nu}B^{\lambda}$.\\
~\\
Q.E.D.
~\\
~\\
~\\
% 第六题
\textbf{6. Known that $\displaystyle \dif s^{2}=g_{\mu\nu}\dif x^{\mu}\dif x^{\nu}=-\dif\tau^{2}$. Derive the geodesic equation from variational principle $\displaystyle \delta\int_{A}^{B}\dif s=0$ or $\displaystyle \delta\int_{A}^{B}(\frac{\dif\tau}{\dif\lambda})^{2}\dif\lambda=0$.} \\	
~\\
\textbf{Answer\,:} We can get $\dif s=(g_{\alpha\beta}\dif x^{\alpha}\dif x^{\beta})^{1/2}$, and introduce scalar parameter $\lambda$.\\
Then we have
\begin{equation}
\displaystyle \dif s=(g_{\alpha\beta}\dot x^{\alpha}\dot x^{\beta})^{1/2}\dif\lambda
	\tag{6.1}
\end{equation}
and $\displaystyle \dot x^{\alpha}=\frac{\dif x^{\alpha}}{\dif\lambda}$, $\displaystyle \dot x^{\beta}=\frac{\dif x^{\beta}}{\dif\lambda}$.\\
So we have
\begin{equation}
\displaystyle 
\delta\int_{A}^{B}L\dif \lambda=0
	\tag{6.2}
\end{equation}
and $L=(g_{\alpha\beta}\dot x^{\alpha}\dot x^{\beta})^{1/2}$ is the Lagrangian.
From Lagrange equation $\displaystyle \frac{\partial L}{\partial x^{\nu}}-\frac{\dif}{\dif\lambda}\frac{\partial L}{\partial\dot x^{\nu}}=0$, we have :
\begin{equation}
\displaystyle 
\frac{1}{(g_{\alpha\beta}\dot x^{\alpha}\dot x^{\beta})^{1/2}}\frac{\partial g_{\alpha\beta}}{\partial x^{\nu}}\dot x^{\alpha}\dot x^{\beta}-\frac{\dif}{\dif\lambda}\frac{g_{\alpha\nu}\dot x^{\alpha}+g_{\beta\nu}\dot x^{\beta}}{(g_{\alpha\beta}\dot x^{\alpha}\dot x^{\beta})^{1/2}}=0
	\tag{6.3}
\end{equation}
when we select	 $\lambda$ as $s$, we have
\begin{equation}
\displaystyle 
(g_{\alpha\beta}\dot x^{\alpha}\dot x^{\beta})^{1/2}=g_{\alpha\beta}\frac{\dif x^{\alpha}}{\dif\lambda}\frac{\dif x^{\beta}}{\dif\lambda}=g_{\alpha\beta}\frac{\dif x^{\alpha}}{\dif s}\frac{\dif x^{\beta}}{\dif s}=1
	\tag{6.4}
\end{equation}
Now we can rewrite the Lagrange equation as
\begin{equation}
\begin{aligned}
\displaystyle 
g_{\alpha\nu,\beta}\dot x^{\alpha}\dot x^{\beta}-\frac{\dif}{\dif s}(g_{\alpha\nu}\dot x^{\alpha})=0\\
g_{\alpha\nu}\frac{\dif^{2}x^{\alpha}}{\dif s^{2}}+(g_{\alpha\nu,\beta}-\frac{1}{2}g_{\alpha\beta,\nu})\frac{\dif x^{\alpha}}{\dif s}\frac{\dif x^{\beta}}{\dif s}=0
\end{aligned}
	\tag{6.5}
\end{equation}
and
\begin{equation}
\displaystyle 
g_{\alpha\nu,\beta}\frac{\dif x^{\alpha}}{\dif s}\frac{\dif x^{\beta}}{\dif s}=g_{\beta\nu,\alpha}\frac{\dif x^{\alpha}}{\dif s}\frac{\dif x^{\beta}}{\dif s}=g_{\alpha\beta,\nu}\frac{\dif x^{\beta}}{\dif s}\frac{\dif x^{\alpha}}{\dif s}
	\tag{6.6}
\end{equation}
so
\begin{equation}
\displaystyle 
\frac{\dif^{2}x^{\mu}}{\dif s^{2}}+\frac{1}{2}g^{\mu\nu}(g_{\alpha\nu,\beta}+g_{\beta\nu,\alpha}-g_{\alpha\beta,\nu})\frac{\dif x^{\alpha}}{\dif s}\frac{\dif x^{\beta}}{\dif s}=0
	\tag{6.7}
\end{equation}
Finally we can get
\begin{equation}
\displaystyle 
\frac{\dif^{2}x^{\mu}}{\dif s^{2}}+\Gamma^{\mu}_{\alpha\beta}\frac{\dif x^{\alpha}}{\dif s}\frac{\dif x^{\beta}}{\dif s}=0
	\tag{6.8}
\end{equation}
~\\
~\\
~\\
% 第七题
\textbf{7. If an ordinary spherical space is embedded in three - dimensional Euclidean space and spherical polar coordinate system is selected, it has line elements with the form $\dif s^{2}=a^{2}\dif\theta^{2}+a^{2}\sin^{2}\theta\dif\phi^{2}$ :} \\	
~\\
\textbf{(1) derive $g^{\mu\nu}$;}\\
\textbf{(2) derive all Christoffel connection $\Gamma^{\mu}_{\alpha\beta}$;}\\
\textbf{(3) derive all $R^{\alpha}_{\mu\nu\lambda}$;}\\
\textbf{(4) derive all $R_{\mu\nu}$;}\\
\textbf{(5) derive $R$;}\\
\textbf{(6) derive the geodesic equation of spherical space represented by the metric.}\\
~\\
\textbf{Answer\,:}\\
(1) From the definition we can get $\displaystyle g^{11}=\frac{1}{a^{2}}$, $\displaystyle g^{22}=\frac{1}{a^{2}\sin^{2}\theta}$, and $g^{12}=g^{21}=0$ . \\
~\\
(2) From (1), we can get
\begin{equation}
\displaystyle 
\Gamma^{1}_{11} = \frac{1}{2}g^{1\mu}(g_{\mu1,1}+g_{1\mu,1}-g_{11,\mu})=\frac{1}{2}g^{11}g_{11,1}=0\\
	\tag{7.1}
\end{equation}

\begin{equation}
\displaystyle 
\Gamma^{1}_{12} = \Gamma^{1}_{21} = \frac{1}{2}g^{11}(g_{11,2}+g_{21,1}-g_{12,1})=\frac{1}{2}g^{11}g_{11,2}=0\\
	\tag{7.2}
\end{equation}

\begin{equation}
\begin{aligned}
\displaystyle 
\Gamma^{1}_{22} &= \frac{1}{2}g^{11}(g_{12,2}+g_{21,2}-g_{22,1})\\
&=\frac{1}{2}g^{11}(-g_{22,1})=\frac{1}{2}\frac{1}{a^{2}}(-2a^{2}\sin\theta\cos\theta)=-\sin\theta\cos\theta
\end{aligned}
	\tag{7.2}
\end{equation}

\begin{equation}
\displaystyle 
\Gamma^{2}_{11} = \frac{1}{2}g^{22}(g_{21,1}+g_{12,1}-g_{11,2})=0 
	\tag{7.3}
\end{equation}

\begin{equation}
\begin{aligned}
\displaystyle 
\Gamma^{2}_{12} &= \Gamma^{2}_{21} = \frac{1}{2}g^{22}(g_{22,1}+g_{12,2}-g_{21,2})\\
&=\frac{1}{2}g^{22}g_{22,1}=\frac{1}{2}\frac{1}{a^{2}\sin^{2}\theta}(2a^{2}\sin\theta\cos\theta)=\cot\theta
\end{aligned}
	\tag{7.4}
\end{equation}

\begin{equation}
\displaystyle 
\Gamma^{2}_{22} = \frac{1}{2}g^{22}g_{22,2}=0 
	\tag{7.5}
\end{equation}
So we can finally get $\Gamma^{1}_{11}=\Gamma^{1}_{12}=\Gamma^{1}_{21}=\Gamma^{2}_{11}=\Gamma^{2}_{22}=0$, $\Gamma^{1}_{22}=-\sin\theta\cos\theta$, and $\Gamma^{2}_{12}=\cot\theta$.\\
~\\
(3) From (2), we can get
\begin{equation}
\begin{aligned}
\displaystyle 
R^{1}_{212} &= \Gamma^{1}_{22,1}- \Gamma^{1}_{21,2}+\Gamma^{1}_{\sigma1}\Gamma^{\sigma}_{22}-\Gamma^{1}_{\sigma2}\Gamma^{\sigma}_{21}\\
&= \Gamma^{1}_{22,1}-\Gamma^{1}_{22}\Gamma^{2}_{21}=\sin^{2}\theta
\end{aligned}
	\tag{7.6}
\end{equation}
and
\begin{equation}
\displaystyle 
R^{2}_{121} = g^{22}R_{2121} = g^{22}R_{1212}
	\tag{7.7}
\end{equation}
for $R^{2}_{212}=g^{11}R^{1}_{1212}$ , then we can get $\displaystyle R_{2121} = \frac{R^{1}_{212}}{g^{11}} = \frac{\sin^{2}\theta}{\frac{1}{a^{2}}}=a^{2}\sin^{2}\theta$.\\
So we have $\displaystyle R^{2}_{121} = \frac{1}{a^{2}\sin^{2}\theta} \cdot a^{2}\sin^{2}\theta=1$. We can finally get $R^{1}_{212} = \sin^{2}\theta, R^{2}_{121} = 1$.\\
~\\
(4) Because $R_{\mu\nu}=R^{\lambda}_{\mu\nu\lambda}=R^{1}_{\mu\nu1}+R^{2}_{\mu\nu2}$, we can get Equ.7.8 below :\\
\begin{equation}
\begin{aligned}
\displaystyle 
R_{11} &= R^{\lambda}_{11\lambda} = R^{1}_{111}+R^{2}_{112} = -1\\
R_{12} &= R^{\lambda}_{12\lambda} = R^{1}_{121}+R^{2}_{122} = 0\\
R_{22} &= R^{\lambda}_{22\lambda} = R^{1}_{221}+R^{2}_{222} = -\sin^{2}\theta\\
\end{aligned}
	\tag{7.8}
\end{equation}
~\\
(5) $\displaystyle R=g^{\mu\nu}R_{\mu\nu}=g^{11}R_{11}+g^{22}R_{22}=-\frac{1}{a^{2}}-\frac{1}{a^{2}}=-\frac{2}{a^{2}}$.\\
~\\
(6) The geodesic equation in triangular rectangular coordinates system is $\displaystyle \frac{\dif^{2}x^{\mu}}{\dif\tau^{2}}+\Gamma^{\mu}_{\alpha\beta}\frac{\dif x^{\alpha}}{\dif\tau}\frac{\dif x^{\beta}}{\dif\tau}=0$.\\
Then we can transform it into spherical polar coordinate system :
\begin{equation}
\displaystyle 
\frac{\dif^{2}\theta}{\dif\tau^{2}}+(-\sin\theta\cos\theta)\frac{\dif\phi}{\dif\tau}\frac{\dif\phi}{\dif\tau}=0
	\tag{7.9}
\end{equation}
and we will finally get
\begin{equation}
\displaystyle 
\frac{\dif^{2}\phi}{\dif\tau^{2}}+2\cot\theta\frac{\dif\theta}{\dif\tau}\frac{\dif\phi}{\dif\tau}=0
	\tag{7.10}
\end{equation}
Equ.7.10 is the geodesic equation we want.
~\\
~\\
~\\
% 第八题
\textbf{8. Prove Einstein field equation $\displaystyle R_{\mu\nu}-\frac{1}{2}g_{\mu\nu}R=\kappa T_{\mu\nu}$ can be rewritten as $\displaystyle R_{\mu\nu}=\kappa(T_{\mu\nu}-\frac{1}{2}g_{\mu\nu}T)$.} \\		
~\\
\textbf{Proof\,:} Transform $\displaystyle R_{\mu\nu}-\frac{1}{2}g_{\mu\nu}R=\kappa T_{\mu\nu}$ into 
\begin{equation}
\displaystyle
g^{\mu\nu}R_{\mu\nu}-\frac{1}{2}g^{\mu\nu}g_{\mu\nu}R=\kappa g^{\mu\nu}T_{\mu\nu}
	\tag{8.1}
\end{equation}
then we'll get
\begin{equation}
\displaystyle
R-\frac{1}{2} \cdot 4R=\kappa T \quad \Rightarrow  \quad R=-\kappa T	% \Longrightarrow和\Longrightarrow为长箭头,\Rightarrow和\rightarrow为短箭头
	\tag{8.2}
\end{equation}
So we have
\begin{equation}
\begin{aligned}
\displaystyle
R_{\mu\nu}-\frac{1}{2}g_{\mu\nu}R=R_{\mu\nu}-\frac{1}{2}g_{\mu\nu}\kappa T=\kappa T_{\mu\nu}\\
\Rightarrow R_{\mu\nu}=-\kappa(\frac{1}{2}g_{\mu\nu}T-T_{\mu\nu})=\kappa(T_{\mu\nu}-\frac{1}{2}g_{\mu\nu}T)
\end{aligned}
	\tag{8.3}
\end{equation}
~\\
Q.E.D.
~\\
~\\
~\\
% 第九题
\textbf{9. Under the linear approximation of a weak gravitational field, the metric can be written as $g_{\mu\nu}=\eta_{\mu\nu}+h_{\mu\nu}$. Find the form of a linearized Einstein field equation.} \\	
~\\
\textbf{Answer\,:} Under weak gravitational field we can represent metric as $g_{\mu\nu}=h_{\mu\nu}+\eta_{\mu\nu}$ , and we make $|h_{\mu\nu}|\ll1$ . In linear approximation theory, we just keep the linear terms for $h_{\mu\nu}$ , so we have 	% \ll表示远小于,对应有表示\gg远大于
\begin{equation}
\begin{aligned}
\Gamma^{\mu}_{\alpha\beta} &= \frac{1}{2}\eta_{\mu\nu}(h_{\alpha\nu,\beta}+h_{\beta\nu,\alpha}-h_{\alpha\beta,\nu}) \\
&=\frac{1}{2}(h_{\alpha,\beta}^{\mu}+h_{\beta,\alpha}^{\mu}-h_{\alpha\beta}^{,\mu})
\end{aligned}
		\tag{9.1}
\end{equation}
And we have linearized Ricci tensor 
\begin{equation}
\begin{aligned}
R_{\mu\nu} &= \Gamma^{\lambda}_{\mu\lambda,\,\nu}-\Gamma^{\lambda}_{\mu\nu,\,\lambda} \\
&\equiv \frac{1}{2}(h^{,\alpha}_{\mu\nu}+h_{,\mu\,,\nu}-h^{\alpha}_{\mu,\nu,\alpha}-h^{\alpha}_{\nu,\mu,\alpha})
\end{aligned}
		\tag{9.2}
\end{equation}
for $h\equiv h^{\alpha}_{\alpha}=\eta_{\alpha\beta}h_{\alpha\beta}$ .\\
~\\
We define 
\begin{equation}
\bar h_{\mu\nu}\equiv h_{\mu\nu}-\frac{1}{2}\eta_{\mu\nu}h
		\tag{9.3}
\end{equation}
and its inverse transformation is 
\begin{equation}
 {\bar{\bar h}}_{\mu\nu} \equiv \bar h_{\mu\nu}-\frac{1}{2}\eta_{\mu\nu}\bar h=h_{\mu\nu}
		\tag{9.4}
\end{equation}
which can be easily proved. \\
With the help of Equ.9.3 and Equ.9.4, we'll get linearized field equation
\begin{equation}
\bar{R}_{\mu\nu} \equiv R_{\mu\nu}-\frac{1}{2}\eta_{\mu\nu}R=-8\pi GT_{\mu\nu}
		\tag{9.5(a)}
\end{equation}
The specific form is
\begin{equation}
\bar h^{,\alpha}_{\mu\nu,\alpha}+\eta_{\mu\nu} \bar{h}^{,\alpha,\beta}_{\alpha\beta}-\bar h^{,\alpha}_{\mu\alpha,\nu}-\bar h^{,\alpha}_{\nu\alpha,\mu}=-16\pi GT_{\mu\nu}
		\tag{9.5(b)}
\end{equation}
Consider the harmonic condition, then we have
\begin{equation}
\bar h^{,\alpha}_{\mu\alpha}=0
		\tag{9.6}
\end{equation}
So we can finally get the simplified field equation
\begin{equation}
\bar h^{,\alpha}_{\mu\nu,\alpha}=-16\pi GT_{\mu\nu}
		\tag{9.7}
\end{equation}
Equ.9.7 is the answer we want to get.
~\\
~\\
~\\
% 第十题
\textbf{10. Suppose $\dif s^{2}=-(x^{0})^{4}(\dif x^{0})^{2}+2e^{x^{1}}(\dif x^{1})^{2}+e^{-x^{2}}(\dif x^{2})^{2}+(\dif x^{3})^{2}$, and prove the space-time is flat.} \\		
~\\
\textbf{Proof\,:} Under the coordinate transformation in Equ.10.1
\begin{equation}
\left\{		% 大括号公式的写法
\begin{aligned}
\displaystyle 
t &=\frac{1}{3}(x_{0})^{3}\\
x &=2\sqrt{2}e^{x^{1}/2}\\
y &=-2e^{-x^{2}/2}\\
z &=x^{3}\\
\end{aligned}
\right.	% 注意这个点不要拉下\right.	
		\tag{10.1}
\end{equation}
We can get $\dif s^{2}=-\dif t^{2}+\dif x^{2}+\dif y^{2}+\dif z^{2}$, so the space-time is flat.\\
~\\
Q.E.D.
~\\
~\\
\end{document}  